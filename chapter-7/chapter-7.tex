\chapter{Fazit}
\label{sec:fazit}

\section{Zusammenfassung des methodischen Vorgehens}

Diese Arbeit untersucht, inwieweit Umweltindikatoren aus PEP-Ecopassport-
Dokumenten mit wenigen, allgemein verfügbaren Produktmerkmalen geschätzt werden
können, um auch für Produkte ohne PEP eine belastbare Größenordnungsabschätzung
zu ermöglichen. Im Fokus steht damit nicht die exakte Rekonstruktion einer
vollständigen Ökobilanz, sondern ein kompakter, praxistauglicher Ansatz zur
Vorhersage zentraler Umweltwirkungen.

Als Grundlage wurde eine automatisierte Pipeline aufgebaut, die heterogene
PEP-PDFs in ein einheitliches, maschinenlesbares Datenformat überführt. Die
Pipeline umfasst die Recherche und Erfassung relevanter PEPs, das PDF-Parsing
und die strukturierte Extraktion von Produkt- und Umweltinformationen sowie die
Normalisierung und Validierung der resultierenden Daten. Dadurch entsteht eine
Datenbasis, die eine systematische statistische Auswertung der
Indikatoren und die anschließende Modellierung ermöglicht.

Für die statistische Auswertung werden ausschließlich Datensätze berücksichtigt,
bei denen die jeweiligen Zielindikatoren sowie zentrale Produktmerkmale
(z.\,B.\ Gesamtgewicht und Stromverbrauch) plausibel und numerisch verfügbar
sind. Bereits die deskriptive Analyse zeigt, dass sämtliche Umweltindikatoren
stark rechtsschief verteilt sind und mehrere Größenordnungen abdecken.
Einzelne Produkte weisen sehr hohe Werte auf und prägen Mittelwerte und
RMSE-basierte Kennzahlen überproportional.

Um diese Skaleneffekte zu reduzieren und die Modellierung zu stabilisieren,
wurden für alle Indikatoren systematisch verschiedene Transformationen der
Zielvariablen verglichen. Je nach Indikator wurden keine Transformation,
\texttt{log1p} sowie Box-Cox-Transformationen getestet. In den meisten Fällen
führte eine logarithmische Transformation zu der besten oder zumindest
stabilsten Testgüte, während Box-Cox die Ergebnisse teilweise noch leicht
verbessern konnte. Die endgültig berichteten Fehlermaße werden nach Möglichkeit
nach Rücktransformation auf der Originalskala angegeben, damit sie in der
jeweiligen Einheit interpretierbar bleiben.

Als Modellansatz wurde eine lineare Regression mit bewusst kompaktem
Feature-Set gewählt, um die Zielsetzung der Arbeit zu verfolgen. 
Als erklärende Variablen dienen das log-transformierte
Produktgewicht, der log-transformierte, über die Lebensdauer aggregierte
Stromverbrauch sowie verdichtete Materialinformationen, die über eine PCA aus
dem Materialblock abgeleitet werden. Dadurch bleibt das Modell auch bei vielen
Materialspalten handhabbar und nutzt wenige, interpretierbare
Hauptkomponenten statt viele einzelne Materialinformationen.

Die Modellgüte wird mit einer streng getrennten Train/Test-Auswertung beurteilt.
Dazu werden die Daten einmalig in Trainings- und Testdaten aufgeteilt und das
Modell ausschließlich auf den Trainingsdaten angepasst. Hyperparameter, etwa
bei Ridge oder Lasso, werden über wiederholte innere Train/Valid-Splits im
Trainingsdatensatz bestimmt. Die abschließende Bewertung erfolgt genau einmal
auf dem unabhängigen Testset. Als Kennzahlen werden neben $R^2$ und RMSE auch
robuste Fehlermaße berichtet, insbesondere der Median des absoluten Fehlers
sowie Median und Mittelwert der relativen absoluten Fehler, um die typische
Modellleistung trotz Ausreißern nachvollziehbar zu charakterisieren.

\section{Einordnung der Ergebnisse}

Die Ergebnisse der Regressionmodelle zeigen, dass sie als
\emph{heuristische Abschätzung} zu verstehen sind und nicht als Ersatz für eine
vollständige Berechnung. Auch bei hoher erklärter Varianz
können einzelne Produkte deutlich verfehlt werden, insbesondere in Bereichen
mit sehr kleinen oder sehr großen Zielwerten und bei Nischenprodukten. Die
Ergebnisse liefern damit keine belastbare Grundlage, um PEP Werte präzise zu
replizieren oder einzelne Produkte auf Dezimalstellen genau zu vergleichen.

Der Mehrwert liegt vielmehr in der schnellen, datengetriebenen Einordnung von
Größenordnungen und in der Priorisierung. Das Modell kann beispielsweise
genutzt werden, um frühe Design und Beschaffungsentscheidungen grob zu
unterstützen, Produkte oder Varianten hinsichtlich erwarteter Umweltwirkungen
vorzusortieren, sowie Auffälligkeiten in PEP Angaben zu identifizieren, wenn
ein ausgewiesener Wert stark von der erwarteten Größenordnung abweicht. Damit
passt der Ansatz besonders zu Anwendungen, in denen nur wenige Merkmale
verfügbar sind, aber trotzdem eine erste Orientierung benötigt wird.


\section{Ausblick und zukünftiger Forschungsbedarf}

Ein zentraler Entwurfsgedanke war die Beschränkung auf Merkmale, die auch ohne
PEP typischerweise messbar oder zumindest plausibel abschätzbar sind. Es wäre allerdings
prinzipiell möglich, zusätzliche erklärende Variablen in das Modell aufzunehmen,
etwa die verwendete Berechnungsmethode und die verwendete Datenbank. 
Solche Informationen könnten die Genauigkeit erhöhen,
würden jedoch die Einfachheit und Übertragbarkeit des Ansatzes verringern, da
sie für neue Produkte häufig nicht verlässlich oder nicht in standardisierter
Form vorliegen. Damit stünde eine solche Erweiterung im Spannungsfeld zur
Zielsetzung, mit einem bewusst kompakten Feature Set robuste, breit nutzbare
Abschätzungen zu ermöglichen.

Eine naheliegende Erweiterung, die diese Zielsetzung weitgehend erhält, ist
die Modellierung getrennt nach Produktkategorien. Eine Kategorisierung lässt
sich meist aus Metadaten oder Produktbeschreibungen ableiten und führt
gleichzeitig zu homogeneren Teilmengen, in denen sich die Beziehungen zwischen
Gewicht, Stromverbrauch, Materialmix und Indikatoren stabiler verhalten
dürften. Dadurch könnte die Vorhersagegüte insbesondere für Indikatoren mit
mittlerer Stabilität deutlich steigen, ohne dass das Modellkonzept wesentlich 
komplizierter wird und damit die Zielsetzung verloren geht.