\chapter{Explorative Modellentwicklung}
\label{sec:exp_models}

Dieses Kapitel beschreibt die Wahl des Regressionsmodells. 
Im Mittelpunkt steht die
explorative Entwicklung und experimentelle Erprobung verschiedener
Modellvarianten, aus denen das in Abschnitt~\ref{sec:regression_co2}
eingesetzte Hauptmodell für die CO$_2$-Äquivalente sowie die übrigen
Indikatoren abgeleitet wurde.

Die Modellentwicklung erfolgte iterativ und datengetrieben: Unterschiedliche
Feature-Sets, verschiedene PCA-Varianten sowie alternative lineare
Schätzer wurden ausprobiert und anhand einheitlicher Gütemaße bewertet.
Ziel ist es nachvollziehbar zu machen,
welche Kombinationen sich in der Praxis als robust erwiesen haben und welche
Ansätze verworfen wurden.


\section{Experimentelle Fragestellungen}
\label{sec:exp_questions}

Ausgehend von der Zielsetzung, mit möglichst wenigen und robust erfassbaren 
Merkmalen brauchbare Vorhersagen zu erhalten, leiten sich für die explorative 
Modellentwicklung insbesondere folgende Fragestellungen ab:

\begin{enumerate}
  \item \textbf{Beitrag der Materialinformationen:}
        Verbessert sich die Vorhersagegüte gegenüber einem Modell, 
        das nur Gewicht und Stromverbrauch nutzt, wenn zusätzlich Materialinformationen 
        einbezogen werden?
  \item \textbf{Rohmaterialien vs.\ Material-PCA:}
        Ist es günstiger, die zahlreichen Materialspalten direkt zu verwenden,
      oder führt eine PCA des Materialblocks zu stabileren Modellen?
  \item \textbf{Wahl des Regressionsverfahrens und Regularisierung:}
        Unterscheiden sich OLS, \texttt{Ridge} und \texttt{Lasso} hinsichtlich 
        Stabilität und erzielbarer Gütemaße bei den vorhandenen PEP-Daten, 
        insbesondere bei vielen korrelierten Regressoren?
  \item \textbf{Wahl der Zieltransformation:}
        Führt eine Log-Transformation (\texttt{log1p}) oder eine Box-Cox-Transformation zu 
        besserer Vorhersagegüte und plausibleren Residuen?
\end{enumerate}

Die nachfolgenden Abschnitte stellen die dafür durchgeführten Experimente vor und 
leiten aus den beobachteten Unterschieden einfache Heuristiken für die weitere Modellierung ab.



\section{Vergleich der Feature-Sets}
\label{sec:exp_feature_sets}

Als erster Schritt der experimentellen Modellentwicklung wurde untersucht,
welchen Beitrag unterschiedliche Eingangsmerkmale zur Vorhersage von
\emph{Climate change (total)} leisten. Grundlage sind hierbei
$n = 173$ PEPs, für die Gesamtgewicht, Stromverbrauch
und die CO$_2$-Äquivalente vollständig vorliegen.

Verglichen wurden drei Feature-Sets, jeweils in einem linearen
Regressionsmodell auf der Transformationsskala
$\log(1+\text{CO2}_\text{total})$:

\begin{itemize}
  \item \textbf{Basis:} nur Gewicht und Stromverbrauch.
  \item \textbf{Basis + Rohmaterialien:} zusätzlich alle ausgewählten
        Materialien als separate Regressoren.
  \item \textbf{Basis + Rohmaterialien mit Minimalvorkommen:}
        Materialien, die in mehr als 10 PEPs vorkommen, als separate Regressoren.
  \item \textbf{Basis + PCA-Materialien:} statt der Rohmaterialspalten
        werden $k$ Hauptkomponenten aus einer PCA auf dem Materialblock
        verwendet (Varianzschwelle 90\,\%).
\end{itemize}

Für alle drei Varianten wurden dieselben Train/Test-Splits verwendet,
so dass die Gütemaße direkt vergleichbar sind. Tabelle~\ref{tab:reg_exp_feature_sets}
zeigt die erzielten Ergebnisse.

\begin{table}[h]
  \centering
  \caption{Vergleich verschiedener Feature-Sets für den Indikator
           \emph{Climate change (total)} auf der Skala
           $\log(1+\text{CO2}_\text{total})$.}
  \label{tab:reg_exp_feature_sets}
  \begin{tabular}{lccc}
    \toprule
    Modellvariante                     & $R^2_{\text{Train}}$ & \textbf{$R^2_{\text{Test}}$} & RMSE\textsubscript{Test} \\
    \midrule
    Basis (Gewicht, Strom)             & 0.817 & \textbf{0.770} & 1.904 \\
    Basis + Rohmaterialien             & 0.904 & \textbf{0.567} & 2.140 \\
    Basis + Rohmaterialien ($n >=10$)  & 0.842 & \textbf{0.832} & 1.152 \\
    Basis + PCA-Materialien ($n >=10$) & 0.887 & \textbf{0.882} & 1.095 \\
    PCA auf alle Variablen             & 0.822 & \textbf{0.738} & 1.579 \\
    \bottomrule
  \end{tabular}
\end{table}

Das Basismodell aus Gewicht und Stromverbrauch erklärt bereits einen
großen Anteil der Varianz der CO$_2$-Äquivalente. Die Erweiterung um
Rohmaterialanteile erhöht zwar das Trainings-$R^2$ deutlich, bringt
aber auf dem Testdatensatz keinen Mehrwert und verschlechtert $R^2_{\text{Test}}$ und den
RMSE\textsubscript{Test} leicht. Dies ist ein Hinweis auf Überanpassung durch die vielen,
teilweise korrelierten Materialvariablen.

Deutlich besser schneidet das Modell mit einem Minimalvorkommen an Materialien ab.
$R^2_{\text{Test}}$ ist hier durchscnittlich deutlich höher als im Basismodell.

Auffällig ist, dass die Variante \emph{PCA auf alle Variablen}
deutlich schlechter abschneidet als die material-spezifische PCA.
Eine PCA über alle Variablen scheint die starken
Basiseffekte von Gewicht und Strom mit den vielen, teils verrauschten
Materialvariablen zu vermischen und damit genau diese klaren Zusammenhänge
abzuschwächen. 

Die PCA-Variante mit Material-Hauptkomponenten erreicht das höchste Test-$R^2$ und 
einen deutlich geringeren Test-RMSE als die
anderen Modelle. Die Materialinformationen tragen also erkennbar
zur Vorhersage bei, müssen dafür aber in verdichteter Form (PCA) in das
Modell eingehen. Auf Basis dieser Experimente wurde das
„Basis + PCA-Materialien“-Feature-Set als Ausgangspunkt für die weitere
Modellentwicklung und den späteren SGD-Regressionsansatz gewählt.


\section{Vergleich der Regressionsverfahren}
\label{sec:exp_regressionsverfahren}

Die im vorherigen Abschnitt beschriebenen Experimente zum Vergleich der
Feature-Sets wurden für den Indikator \emph{Climate change (total)} zunächst
mit einer klassischen linearen Regression auf Basis der der \texttt{LinearRegression} aus
\texttt{scikit-learn} nachgebildet und in eine Pipeline mit
\texttt{StandardScaler}, optionaler PCA und Train/Test-Aufteilung eingebettet.

Im nächsten Schritt wird das lineare Modell anschließend mit 
\texttt{Ridge} und \texttt{Lasso}
implementiert,
die über einen Regularisierungsparameter $\alpha$ die Modellkomplexität steuern und 
dadurch insbesondere bei vielen, korrelierten Regressoren stabilere Schätzungen 
liefern können.

Bei festem Feature-Set
(Gewicht, Stromverbrauch, PCA-Materialkomponenten) lieferten
\texttt{Ridge} und \texttt{Lasso} über mehrere äußere
Wiederholungen des Experiments ähnliche $R^2$- und RMSE-Werte.
Beide Verfahren schnitten konstant besser ab als \texttt{LinearRegression}.
Die Wahl des Regressors war allerdins, wie die Zahlen zur Modellgüte 
in \ref{tab:vergleich_regressoren} 
zeigen, damit deutlich weniger bedeutend als der Einfluss des Feature-Sets.

\begin{table}[h]
  \centering
  \caption{Vergleich der Regressionsverfahren auf dem festgelegten Feature-Set.}
  \label{tab:vergleich_regressoren}
  \begin{tabular}{lccc}
    \toprule
    Regressionsverfahren                     & $R^2_{\text{Train}}$ & \textbf{$R^2_{\text{Test}}$} & RMSE\textsubscript{Test} \\
    \midrule
    LinearRegression                   & 0.887 & \textbf{0.882} & 77565.84 \\
    Ridge                              & 0.906 & \textbf{0.896} & 59635.57 \\
    Lasso                              & 0.906 & \textbf{0.896} & 59636.03 \\
    \bottomrule
  \end{tabular}
\end{table}

Für die weiteren Analysen wurde das \texttt{Ridge} Modell verwendet,
da es sich über mehrere Wiederholungen mit verschiedenen zufälligen 
Test-/Trainingssplits als minimal robuster und konstanter erwies.


