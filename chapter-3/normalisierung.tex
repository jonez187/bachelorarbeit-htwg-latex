\section{Normalisierung der Daten}
\label{subsec:normalisierung}
Nach der Extraktion lag ein heterogener Datensatz mit uneinheitlichen Bezeichnungsformen
für Länder, Materialien, Lebenszyklusphasen, Energiequellen und Einheiten vor.
Um eine konsistente Auswertung zu ermöglichen, wurden sämtliche Schreibweisen vereinheitlicht.
Ziel war es, strukturell vergleichbare Werte zu schaffen und gruppierte Analysen über
mehrere PEPs hinweg zu ermöglichen.

Zur systematischen Erfassung der vorhandenen Begriffe wurde ein Hilfsskript
entwickelt, das die in den JSON-Dateien vorkommenden
Rohwerte systematisch zusammenfasst. Für jedes relevante Feld werden die Häufigkeiten einzelner
Strings erfasst und als Übersichtstabellen ausgegeben. 

Für jede Datendomäne wurde darauf basierend eine Zu\-ord\-nungs\-tabelle 
(\emph{Mapping-Datei}) erstellt, die reguläre Ausdrücke den vereinheitlichten
Standardbegriffen zuordnet.
Vereinheitlicht wurden:
\begin{itemize}
  \item \textbf{Einheiten} (z.\,B.\ \emph{kg~CO$_2$-eq}, \emph{kg~CO$_2$e} und \emph{kg~CO$_2$~equiv} zu \emph{kg~CO$_2$~eq}),
  \item \textbf{Materialien} (z.\,B.\ \emph{Paper}, \emph{Cardboard} und \emph{Carton} zu \emph{Paper}),
  \item \textbf{Lebenszyklusphasen} (z.\,B.\ \emph{A1 bis A3} und \emph{Manufacturing} zu \texttt{manufacturing}),
  \item \textbf{Strommixe} (z.\,B.\ \emph{France grid mix}, \emph{France Mix} und \emph{French grid} zu \texttt{FR}).
\end{itemize}
Diese Mappings wurden schrittweise verfeinert, bis alle Rohwerte
abgedeckt waren. 

Bei der Vereinheitlichung wurden einige pragmatische Entscheidungen getroffen:
\emph{Steel} und \emph{Iron} wurden beispielsweise zur Kategorie \emph{Steel} zusammengefasst,
\emph{Carton} und \emph{Cardboard} werden zu \emph{Paper} zusammengefasst,
da beide ähnliche Materialeigenschaften aufweisen. 
Für Kunststoffe wurde eine vereinfachte Zusammenführung vorgenommen.

Ein Python-Skript wendet diese Mapping-Dateien auf alle extrahierten JSON-Dateien an.

Doppelte Materialeinträge, die durch das Mapping innerhalb eines Produkts entstanden sind, 
werden, sofern sie identische Bezeichnungen aufweisen, mithilfe eines weiteren Python-Skripts zusammengeführt. 
Dabei werden Prozentangaben und Gewichtsangaben bei Dubletten addiert.

Durch die Vo\-kabular\-analyse und an\-schließen\-de Normalisierung entsteht so ein
standardisierter Datensatz mit konsistenten Schreibweisen und eindeutiger
Begriffssystematik. Diese Vereinheitlichung bildet die methodische Grundlage
für die nachfolgende quantitative Auswertung.

