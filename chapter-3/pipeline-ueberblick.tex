\section{Überblick über die Pipeline}

Ziel der entwickelten Pipeline ist die automatisierte Extraktion strukturierter Daten
aus PEP-Ecopassport-Dokumenten im PDF-Format. Die PEPs bilden die zentrale Quelle für
produktbezogene Umweltinformationen, enthalten jedoch uneinheitlich formatierte
Tabellen und Textblöcke, die eine direkte Auswertung erschweren.

Die Pipeline wandelt die heterogenen PDF-Dokumente in ein einheitliches,
maschinenlesbares Datenformat um. Als Input dienen die PEP-PDFs, der Output ist
eine strukturierte CSV-Datei, die sämtliche relevanten Variablen zu Produkt,
Materialien, Energieverbrauch und Umweltindikatoren enthält. Der Prozess umfasst
mehrere aufeinanderfolgende Schritte:

\begin{itemize}
  \item \textbf{Recherche und Erfassung:} Recherche, Speicherung und Bewertung der
        verfügbaren PEP-Dokumente mit Gebäudeautomatisierungsbezug aus der öffentlichen PEP-Datenbank \cite{PEPDB}. 
  \item \textbf{Extraktion:} Umwandlung der PDF-Dateien in Rohtext und Tabelleninhalte
        mittels Dokumentenparser, Layout- und Tabellenstrukturen werden erkannt.
  \item \textbf{Interpretation:} Zuordnung der erkannten Inhalte zu einem definierten
        Schema mithilfe regelbasierter und LLM-gestützter Methoden.
  \item \textbf{Normalisierung:} Harmonisierung von Einheiten, Materialnamen und
        Energie­modellen zur Sicherstellung der Vergleichbarkeit.
  \item \textbf{Export:} Zusammenführung aller Informationen in eine flache,
        analysierbare CSV-Datei als Grundlage der nachfolgenden statistischen
        Auswertung.
\end{itemize}

Abbildung~\ref{fig:pipeline_overview} zeigt den groben schematischen Aufbau des
Gesamtprozesses von der Rohdatenerfassung bis zur strukturierten Datenbasis. 

\begin{figure}[h]
  \centering
  \includegraphics[width=1.0\textwidth]{images/Pipeline_Grafik.png}
  \caption{Schematischer Aufbau der Pipeline: von der PEP-Erfassung bis zur
  strukturierten Datenbasis. [Eigene Darstellung]}
  \label{fig:pipeline_overview}
\end{figure}