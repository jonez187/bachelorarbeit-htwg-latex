\chapter{Pipeline und Datenbasis (Methodik)}
\label{chapter:methodik}

Dieses Kapitel beschreibt den Aufbau der Datenpipeline, die Extraktion der relevanten
Variablen aus PEP-Ecopassport-Dokumenten sowie die Struktur und Aufbereitung der
resultierenden Datenbasis.

\FloatBarrier
\section{Überblick über die Pipeline}

Ziel der entwickelten Pipeline ist die automatisierte Extraktion strukturierter Daten
aus PEP-Ecopassport-Dokumenten im PDF-Format. Die PEPs bilden die zentrale Quelle für
produktbezogene Umweltinformationen, enthalten jedoch uneinheitlich formatierte
Tabellen und Textblöcke, die eine direkte Auswertung erschweren.

Die Pipeline wandelt die heterogenen PDF-Dokumente in ein einheitliches,
maschinenlesbares Datenformat um. Als Input dienen die PEP-PDFs, der Output ist
eine strukturierte CSV-Datei, die sämtliche relevanten Variablen zu Produkt,
Materialien, Energieverbrauch und Umweltindikatoren enthält. Der Prozess umfasst
mehrere aufeinanderfolgende Schritte:

\begin{itemize}
  \item \textbf{Recherche und Erfassung:} Recherche, Speicherung und Bewertung der
        verfügbaren PEP-Dokumente mit Gebäudeautomatisierungsbezug aus der öffentlichen PEP-Datenbank \cite{PEPDB}. 
  \item \textbf{Extraktion:} Umwandlung der PDF-Dateien in Rohtext und Tabelleninhalte
        mittels Dokumentenparser, Layout- und Tabellenstrukturen werden erkannt.
  \item \textbf{Interpretation:} Zuordnung der erkannten Inhalte zu einem definierten
        Schema mithilfe regelbasierter und LLM-gestützter Methoden.
  \item \textbf{Normalisierung:} Harmonisierung von Einheiten, Materialnamen und
        Energie­modellen zur Sicherstellung der Vergleichbarkeit.
  \item \textbf{Export:} Zusammenführung aller Informationen in eine flache,
        analysierbare CSV-Datei als Grundlage der nachfolgenden statistischen
        Auswertung.
\end{itemize}

Abbildung~\ref{fig:pipeline_overview} zeigt den groben schematischen Aufbau des
Gesamtprozesses von der Rohdatenerfassung bis zur strukturierten Datenbasis. 

\begin{figure}[h]
  \centering
  \includegraphics[width=1.0\textwidth]{images/Pipeline_Grafik.png}
  \caption{Schematischer Aufbau der Pipeline: von der PEP-Erfassung bis zur
  strukturierten Datenbasis. [Eigene Darstellung]}
  \label{fig:pipeline_overview}
\end{figure}

\FloatBarrier
\section{Datenerhebung und PEP-Recherche}

Ziel der Datenerhebung ist die Identifikation und Extraktion von PEP-Ecopassport-Dokumenten,
die sich auf Geräte der Gebäudeautomatisierung oder IoT Komponenten beziehen. Die PEP-Datenbank (\cite{PEPDB}) bildet dabei
die Quelle der Untersuchung. Für jedes gefundene Dokument wurden Produktinformationen,
Material- und Energiedaten sowie Metadaten in strukturierter Form erfasst.

Im ersten Schritt wurden die Dokumente automatisiert recherchiert, die zugehörigen PDFs
heruntergeladen, analysiert und nach Relevanz für den Smart-Home- bzw. IoT-Bereich klassifiziert. 
Das Ergebnis wurde in CSV-Formaten gespeichert und diente als Grundlage für die weitergehende Analyse.

Zur Ermittlung der verfügbaren PEP-Dokumente wurde zunächst die Suchfunktion der
PEP-Datenbank analysiert. Über die Browser-Entwicklertools konnten die zugrunde liegenden
Netzwerkanfragen identifiziert werden. 
Dadurch konnte mithilfe von \emph{JavaScript} HTML-Snippets der Produkte 
mit den gesuchten Daten abgefragt, geparst und in CSV-Dateien gespeichert werden.
Um dabei gezielt IoT relevante Produkte zu erfassen, wurden die Abfragen mit spezifischen
Suchbegriffen in den Parametern erweitert (z.\,B.\ \emph{controller}, \emph{sensor}, \emph{gateway},
\emph{wifi}, \emph{knx}, \emph{zigbee}, \emph{cloud}). 
Anschließend erfolgte eine manuelle Prüfung und Klassifikation
der Ergebnisse, da die Suchbegriffe im Produktitel nicht direkt 
auf Komponenten der Gebäudeautomation schließen können.
Zudem wendet die Suchfunktion der PEP-Plattform nicht alle Filter korrekt
an und es treten Überschneidungen zwischen den Seiten auf.

Die ermittelten Produkte wurden in einer Excel-Datei manuell kategorisiert.
Die Zuordnung erfolgte nach folgenden Kriterien:
\begin{itemize}
  \item \textbf{Gebäudeautomatisierung:} Geräte mit klarer Konnektivität (z.\,B.\ ZigBee, WiFi, KNX) oder Cloud-Anbindung,
        wie Gateways, smarte Sensoren oder Steuerungen.
  \item \textbf{Keine IoT-Relevanz (Aussortiert):} Produkte ohne Kommuni\-kations\-fähigkeit (z.\,B.\ Kabel, Trafos, LED-Panels).
\end{itemize}

Zusätzlich zur halbautomatisierten Suche wurden IoT-relevante Unternehmen gezielt
identifiziert (z.\,B.\ ABB, Siemens, Schneider Electric, Legrand, Somfy, Daikin,
Bosch, Honeywell). Deren PEP-Dokumente wurden manuell durchsucht und ergänzt. 
Dieses Vorgehen ist aufwändiger als die automatisierte Suche über Netzwerkabfragen,
konnte die Datenbasis aber enorm vergrößern.

\FloatBarrier
\section{PDF-Parsing und Extraktion}

Die Extraktion strukturierter Daten aus PEP-PDFs stellt einen technisch anspruchsvollen
Teil der Arbeit dar. Ziel ist es, aus den heterogenen Dokumenten eine konsistente,
maschinenlesbare Repräsentation der Umweltindikatoren, Gerätedaten und
Metadaten zu erzeugen, welche entsprechend der Zielsetzung der Arbeit analysiert werden können. 
Die finale Lösung kombiniert eine robuste Layoutanalyse mit
Docling und eine LLM-basierte, schema­\-gesteuerte Inhalts\-interpretation.

Zu Beginn wurde eine auf \texttt{pdfplumber} basierende Pipeline eingesetzt, 
die auf der von Selg \cite{Selg2025} entwickelten Pipeline aufbaut.
Sie erkennt mithilfe von Regex- und Textheuristiken Tabellen und Materiallisten. 
Obwohl dieser Ansatz für einzelne PDFs funktionierte, erwies sich die Übertragbarkeit als unzureichend.
Ursache waren typische Strukturprobleme von PDF-Dateien: eine verzerrte
Zeilen- und Wortreihenfolge im Textlayer, stark variierende Layouts, Tabellen als
Rasterbilder sowie uneinheitliche Bezeichnungs- und Einheitenformate. Bereits kleine
Abweichungen in Tabellenköpfen führten zu fehlerhaften Zuordnungen von Indikatoren oder
Spalten. 

Die Vielzahl individueller Ausnahmen entwickelte sich zu einem unübersichtlichen Netz von abzufangenden Ausnahmefällen, 
das neue Konflikte zwischen bestehenden und neu hinzugefügten
Layouts verursachte.
Auch die manuelle Ergänzung einzelner Werte ist bei der erforderlichen Datenmenge in dieser Arbeit nicht mehr praktikabel.
Eine vollständige Generalisierung des pdfplumber-Parsers war im Rahmen
der Arbeit nicht realistisch umsetzbar.

Diese Limitierungen führten zur Entwicklung einer neuen, modularen Pipeline, die auf
dem Open-Source-Framework \emph{Docling} von IBM basiert. Docling erlaubt die
strukturierte Segmentierung von PDF-Inhalten in Absätze, Tabellen, Listen und Bilder
und exportiert diese in Markdown oder JSON. Dadurch konnte die textuelle Logik vom
Layout entkoppelt und die Zuverlässigkeit der Verarbeitung deutlich
verbessert werden.

Die Pipeline trennt klar zwischen Layoutanalyse und Inhaltsinterpretation:
\begin{itemize}
  \item \textbf{Docling-Konvertierung:} PDF-Dateien werden in eine Markdown-Struktur
        überführt. OCR und Bildbeschreibung sind deaktiviert, um Laufzeit und
        Speicherverbrauch zu reduzieren. Tabellen- und Abschnittsgrenzen bleiben
        erhalten.
  \item \textbf{Regelbasierter Filter:}
      Um Kontextverluste des nachgelagerten Sprachmodells zu vermeiden,
      wurde ein regelbasierter Python-Filter auf die aus Docling generierten
      Markdown-Dateien angewendet. Irrelevante Segmente (z.\,B.\ Kopf- und Fußzeilen, Unternehmensinformationen, Titelblätter)
      werden über eine Blacklist entfernt.
  \item \textbf{LLM-basierte Extraktion:} Der kon\-ver\-tierte Text wird in Ab\-schnit\-ten
        an ein Sprachmodell übergeben, das definierte Variablen extrahiert und im
        JSON-Format zurückgibt. Die Prompt\-struktur erzwingt strikte Datentypen und
        klare Feldbezeichnungen. 
\end{itemize}

Für die semantische Extraktion wurde \emph{GPT-5} verwendet, angesprochen über die
\emph{Responses-API}. Diese neue Schnittstelle unterstützt strukturierte
Ausgaben und optional eine Schema-Validierung. Durch einen Aufruf im
\texttt{response\_format=json\_object}-Modus, werden nur gültige JSON-Formate geliefert, 
was den Post-Processing-Aufwand erheblich senkt. GPT-5 konnte
Numerische Werte mit zugehörigen Einheiten stabil erkennen und Module (A1–A3, A4, A5,
B*, C*, D) zuverlässig zuordnen. 
Die Temperatur des LLMs wurde auf 0 gesetzt, um Zufälligkeit und Kreativität möglichst zu vermeiden.

Die Kombination aus Docling und GPT-5 führte somit zu einem 
skalierbaren Verfahren, das auch bei komplexen Layouts konsistente Ergebnisse liefert.

Die neue Pipeline konnte die Anzahl fehlerhafter oder unvollständiger Einträge deutlich
reduzieren.
Für PDFs mit reinen Rastertabellen bleibt jedoch eine 
Einschränkung bestehen, da ohne OCR keine
Inhalts­extraktion möglich ist. 
Der Einsatz eines LLMs führt aufgrund der stochastischen Modellkomponenten zudem zu einer eingeschränkten
Reproduzierbarkeit und Transparenz. 
Obwohl das Risiko minimiert wurde, könnte es vorkommen, dass identische Eingaben
aufgrund von Halluzinationen nicht immer identische Ausgaben liefern.
Diese Einschränkungen sind angesichts der PDF-Heterogenität nicht zu umgehen und methodisch vertretbar.


\FloatBarrier
\section{Normalisierung der Daten}
\label{subsec:normalisierung}
Nach der Extraktion lag ein heterogener Datensatz mit uneinheitlichen Bezeichnungsformen
für Länder, Materialien, Lebenszyklusphasen, Energiequellen und Einheiten vor.
Um eine konsistente Auswertung zu ermöglichen, wurden sämtliche Schreibweisen vereinheitlicht.
Ziel war es, strukturell vergleichbare Werte zu schaffen und gruppierte Analysen über
mehrere PEPs hinweg zu ermöglichen.

Zur systematischen Erfassung der vorhandenen Begriffe wurde ein Hilfsskript
entwickelt, das die in den JSON-Dateien vorkommenden
Rohwerte inventorisiert. Für jedes relevante Feld (z.\,B. Indikatoreinheiten,
Materialbezeichnungen oder Energiequellen) werden die Häufigkeiten einzelner
Strings erfasst und als Übersichtstabellen ausgegeben. 

Für jede Datendomäne wurde darauf basierend eine Zu\-ord\-nungs\-tabelle 
(\emph{Mapping-Datei}) erstellt, die reguläre Ausdrücke den vereinheitlichten
Standardbegriffen zuordnet.
Folgende Datendomänen wurden vereinheitlicht:
Einheiten (z.\,B. \emph{„kg~CO$_2$-eq“}, \emph{„kg~CO2e“} und \emph{„kg~CO2~equiv“} zu \emph{„kg~CO2~eq“}),
Materialien (z.\,B. \emph{„Paper“}, \emph{„Cardboard“} und \emph{„Carton“} zu \emph{„Paper“}),
Lebenszyklusphasen (z.\,B. \emph{A1-A3} und \emph{Manufacturing} zu \emph{manufacturing})
und Strommixe (z.\,B. \emph{France grid mix}, \emph{France Mix}, \emph{French grid} zu \emph{FR}).
Diese Mappings wurden schrittweise verfeinert, bis alle identifizierten
Abweichungen abgedeckt waren. 

Bei der Vereinheitlichung wurden einige pragmatische Entscheidungen getroffen:
\emph{Steel} und \emph{Iron} wurden beispielsweise zur Kategorie \emph{Steel} zusammengefasst,
\emph{Carton} und \emph{Cardboard} werden zu \emph{Paper} zusammengefasst,
da beide ähnliche Materialeigenschaften aufweisen. 
Für Kunststoffe wurde eine vereinfachte Zusammenführung vorgenommen.

Ein Python-Skript wendet diese Mapping-Dateien auf alle extrahierten JSON-Dateien an.

Doppelte Materialeinträge, die durch das Mapping innerhalb eines Produkts entstanden sind, 
werden, sofern sie identische Bezeichnungen aufweisen, mithilfe eines weiteren Python-Skripts zusammengeführt. 
Dabei werden Prozentangaben und Gewichtsangaben bei Dubletten addiert.

Durch die Vo\-kabular\-analyse und an\-schließen\-de Normalisierung entsteht so ein
standardisierter Datensatz mit konsistenten Schreibweisen und eindeutiger
Begriffssystematik. Diese Vereinheitlichung bildet die methodische Grundlage
für die nachfolgende quantitative Auswertung.



\FloatBarrier
\section{Datenbereinigung und Validierung}

Nach der automatischen Extraktion und Normalisierung werden zuerst
fehlerhafte oder unvollständige
Datensätze mithilfe eines Python-Skripts ausgeschlossen. 
Ein Datensatz galt als unbrauchbar, wenn sämtliche
Umweltindikatoren fehlten oder ausschließlich Nullwerte enthielten, oder wenn
die zentralen Felder \texttt{total\_weight}, \texttt{electricity\_consumption},
\texttt{material\_composition} und \texttt{energy\_model} gleichzeitig leer waren.
Diese Kriterien führten zur Aussonderung von 8 der insgesamt 242 Datensätze.
Diese PEPs enthielten zwar Metadaten, jedoch keine quantitativen Werte
und wurden daher nicht in die Analyse einbezogen.

\begin{table}[h]
\centering
\caption{Anzahl der PEP-Dokumente von der Recherche bis zum Analysedatensatz.}
\label{tab:pep_flow}
\begin{tabular}{p{9cm}r}
\toprule
Schritt & Anzahl \\
\midrule
Automatisierte Keyword-Suche (identifiziert) & $+$ 184 \\
Aussortiert ohne Gebäudeautomatisierungsrelevanz & $-$ 77 \\
Manuell ergänzte PEPs & $+$ 145 \\
\midrule
Gesamtzahl recherchierter PEPs & 252 \\
Aussortiert als unbrauchbar oder fehlerhaft & $-$ 18 \\
\midrule
Finaler Analysedatensatz & 234 \\
\bottomrule
\end{tabular}
\end{table}


Die Datenbereinigung wurde durch automatisierte Prüfmechanismen begleitet.
Dazu zählten Validierungen auf fehlende oder ungültige Einheiten, numerische
Typfehler (z.\,B. String statt numerischer Wert) sowie Plausibilitätsprüfungen,
etwa auf Null- oder Extremwerte bei zentralen Variablen.
Ein zusätzlicher Kontrolllauf identifizierte Datensätze mit nicht plausiblen
Summen oder Flatline-Indikatoren (identische Werte in allen Phasen), die von der
weiteren Analyse ausgeschlossen wurden.

Mehrere alternative Ansätze wurden im Verlauf der Datenbereinigung geprüft und
bewusst verworfen. Die Aktivierung von OCR für alle PDFs hätte den Aufwand und
die Laufzeit erheblich erhöht, ohne die Datenqualität signifikant zu verbessern.
Ebenso wurde auf ein vollautomatisches Mapping über Sprachmodelle verzichtet,
da dieses zu unkontrollierten Korrekturen führte. 
Eine Hierarchisierung der Materialien (z.\,B. \emph{Iron} als Unterkategorie von
\emph{Metals}) oder eine gesonderte Behandlung von Verpackungsmaterialien wurde
aus Gründen der Vergleichbarkeit nicht umgesetzt.

Die bereinigten JSON-Dateien werden anschließend
in ein flaches, ta\-bel\-len\-ba\-sier\-tes Format überführt. 
Während die ursprünglichen JSON-Strukturen sowohl menschen- als auch maschinenlesbar
angelegt waren, wurde das Format nun in eine einheitliche, analysierbare
Datenstruktur überführt, die sich für statistische Auswertungen und Visualisierungen
eignet.




Nach Abschluss der Bereinigung standen 234  
strukturierte Datensätze zur Verfügung. Diese bilden die Grundlage für die
nachfolgende statistische Analyse. Die zentrale Datenbasis umfasst vereinheitlichte
Material-, Energie- und Länderschreibweisen sowie geprüfte Indikatorwerte, wodurch
eine zuverlässige quantitative Auswertung der Umweltwirkungen ermöglicht wird.