\section{Datenbereinigung und Validierung}

Nach der automatischen Extraktion und Normalisierung werden zuerst
fehlerhafte oder unvollständige
Datensätze mithilfe eines Python-Skripts ausgeschlossen. 
Ein Datensatz galt als unbrauchbar, wenn sämtliche
Umweltindikatoren fehlten oder ausschließlich Nullwerte enthielten, oder wenn
die zentralen Felder \texttt{total\_weight}, \texttt{electricity\_consumption},
\texttt{material\_composition} und \texttt{energy\_model} gleichzeitig leer waren.
Diese Kriterien führten zur Aussonderung von 8 der insgesamt 242 Datensätze.
Diese PEPs enthielten zwar Metadaten, jedoch keine quantitativen Werte
und wurden daher nicht in die Analyse einbezogen.

\begin{table}[h]
\centering
\caption{Anzahl der PEP-Dokumente von der Recherche bis zum Analysedatensatz.}
\label{tab:pep_flow}
\begin{tabular}{p{9cm}r}
\toprule
Schritt & Anzahl \\
\midrule
Automatisierte Keyword-Suche (identifiziert) & $+$ 184 \\
Aussortiert ohne Gebäudeautomatisierungsrelevanz & $-$ 77 \\
Manuell ergänzte PEPs & $+$ 145 \\
\midrule
Gesamtzahl recherchierter PEPs & 252 \\
Aussortiert als unbrauchbar oder fehlerhaft & $-$ 18 \\
\midrule
Finaler Analysedatensatz & 234 \\
\bottomrule
\end{tabular}
\end{table}


Die Datenbereinigung wurde durch automatisierte Prüfmechanismen begleitet.
Dazu zählten Validierungen auf fehlende oder ungültige Einheiten, numerische
Typfehler (z.\,B. String statt numerischer Wert) sowie Plausibilitätsprüfungen,
etwa auf Null- oder Extremwerte bei zentralen Variablen.
Ein zusätzlicher Kontrolllauf identifizierte Datensätze mit nicht plausiblen
Summen oder Flatline-Indikatoren (identische Werte in allen Phasen), die von der
weiteren Analyse ausgeschlossen wurden.

Mehrere alternative Ansätze wurden im Verlauf der Datenbereinigung geprüft und
bewusst verworfen. Die Aktivierung von OCR für alle PDFs hätte den Aufwand und
die Laufzeit erheblich erhöht, ohne die Datenqualität signifikant zu verbessern.
Ebenso wurde auf ein vollautomatisches Mapping über Sprachmodelle verzichtet,
da dieses zu unkontrollierten Korrekturen führte. 
Eine Hierarchisierung der Materialien (z.\,B. \emph{Iron} als Unterkategorie von
\emph{Metals}) oder eine gesonderte Behandlung von Verpackungsmaterialien wurde
aus Gründen der Vergleichbarkeit nicht umgesetzt.

Die bereinigten JSON-Dateien werden anschließend
in ein flaches, ta\-bel\-len\-ba\-sier\-tes Format überführt. 
Während die ursprünglichen JSON-Strukturen sowohl menschen- als auch maschinenlesbar
angelegt waren, wurde das Format nun in eine einheitliche, analysierbare
Datenstruktur überführt, die sich für statistische Auswertungen und Visualisierungen
eignet.


