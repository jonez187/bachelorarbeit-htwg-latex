\section{Datenerhebung und PEP-Recherche}

Ziel der Datenerhebung ist die Identifikation und Extraktion von PEP-Ecopassport-Dokumenten,
die sich auf Geräte der Gebäudeautomatisierung oder IoT Komponenten beziehen. Die PEP-Datenbank (\cite{PEPDB}) bildet dabei
die Quelle der Untersuchung. Für jedes gefundene Dokument wurden Produktinformationen,
Material- und Energiedaten sowie Metadaten in strukturierter Form erfasst.

Im ersten Schritt wurden die Dokumente automatisiert recherchiert, die zugehörigen PDFs
heruntergeladen, analysiert und nach Relevanz für den Smart-Home- bzw. IoT-Bereich klassifiziert. 
Das Ergebnis wurde in CSV-Formaten gespeichert und diente als Grundlage für die weitergehende Analyse.

Zur Ermittlung der verfügbaren PEP-Dokumente wurde zunächst die Suchfunktion der
PEP-Datenbank analysiert. Über die Browser-Entwicklertools konnten die zugrunde liegenden
Netzwerkanfragen identifiziert werden. 
Dadurch konnte mithilfe von \emph{JavaScript} HTML-Snippets der Produkte 
mit den gesuchten Daten abgefragt, geparst und in CSV-Dateien gespeichert werden.
Um dabei gezielt IoT relevante Produkte zu erfassen, wurden die Abfragen mit spezifischen
Suchbegriffen in den Parametern erweitert (z.\,B.\ \emph{controller}, \emph{sensor}, \emph{gateway},
\emph{wifi}, \emph{knx}, \emph{zigbee}, \emph{cloud}). 
Anschließend erfolgte eine manuelle Prüfung und Klassifikation
der Ergebnisse, da die Suchbegriffe im Produktitel nicht direkt 
auf Komponenten der Gebäudeautomation schließen können.
Zudem wendet die Suchfunktion der PEP-Plattform nicht alle Filter korrekt
an und es treten Überschneidungen zwischen den Seiten auf.

Die ermittelten Produkte wurden in einer Excel-Datei manuell kategorisiert.
Die Zuordnung erfolgte nach folgenden Kriterien:
\begin{itemize}
  \item \textbf{Gebäudeautomatisierung:} Geräte mit klarer Konnektivität (z.\,B.\ ZigBee, WiFi, KNX) oder Cloud-Anbindung,
        wie Gateways, smarte Sensoren oder Steuerungen.
  \item \textbf{Keine IoT-Relevanz (Aussortiert):} Produkte ohne Kommuni\-kations\-fähigkeit (z.\,B.\ Kabel, Trafos, LED-Panels).
\end{itemize}

Zusätzlich zur halbautomatisierten Suche wurden IoT-relevante Unternehmen gezielt
identifiziert (z.\,B.\ ABB, Siemens, Schneider Electric, Legrand, Somfy, Daikin,
Bosch, Honeywell). Deren PEP-Dokumente wurden manuell durchsucht und ergänzt. 
Dieses Vorgehen ist aufwändiger als die automatisierte Suche über Netzwerkabfragen,
konnte die Datenbasis aber enorm vergrößern.