\chapter{Abstract}
\setheader{Abstract}

Diese Arbeit entwickelt eine durch\-gängi\-ge Pipe\-line zur auto\-ma\-ti\-sier\-ten Auf\-be\-rei\-tung von
PEP-Ecopassport-Deklarationen und nutzt die daraus gewonnene Datenbasis zur heuristischen
Abschätzung zentraler Umweltindikatoren für Produkte der Gebäudeautomation. PEPs liefern
standardisierte Umweltkennzahlen, liegen in der Praxis jedoch als heterogene PDF-Dokumente
vor und sind daher nur eingeschränkt automatisiert auswertbar.

Auf Basis des aufbereiteten Datensatzes wird ein Re\-gres\-sions\-an\-satz entwickelt, der
ausschließlich Merkmale nutzt, die auch ohne PEP typischerweise verfügbar oder abschätzbar
sind. Dazu zählen insbesondere Gesamtgewicht, über die Lebensdauer aggregierter
Stromverbrauch und verdichtete Materialinformationen, die durch eine PCA aus dem
verwendeten Material abgeleitet werden. Die Modellgüte wird mit strikt getrennten Trainings- und
Testdaten und zusätzlichen robusten Fehlermaßen bewertet.

Die stabilsten Ergebnisse werden für \emph{Climate change (total)} erzielt. Für mehrere
weitere Indikatoren ergeben sich ebenfalls gute Schätzungen, während \emph{Eutrophication
(freshwater)}, \emph{Eutrophication (marine)} und \emph{Radioactive waste disposed} nur
eingeschränkt erklärbar sind. Methodische Unterschiede zwischen den PEPs, wie verschiedene Berechnungsmethoden, 
begrenzen die Vergleichbarkeit und wirken
als Rauschen. Insgesamt ermöglicht der Ansatz eine schnelle, datengetriebene Einordnung
für neue Produkte ohne PEP und kann frühe Entscheidungen in Entwicklung und
Beschaffung unterstützen.
