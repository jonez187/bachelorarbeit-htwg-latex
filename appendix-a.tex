% appendix-a.tex

\chapter{Weitere Regressionsmodelle}
\label{app:other_regressions}

Die folgenden Abschnitte ergänzen die im Haupttext berichteten Ergebnisse um zusätzliche
Indikatoren. Dargestellt sind jeweils die beste Zieltransformation, die Testgüte sowie
Streudiagramme und QQ Plots zur Einordnung der Fehlerstruktur.

\FloatBarrier
\section{Regression des Indikators Hazardous waste disposed}
\label{sec:reg_hazardous_waste}
\FloatBarrier

Für den Indikator \emph{Hazardous waste disposed} konnten $n = 168$ PEPs verwendet werden.
Es wurden keine Transformation, \texttt{log1p} und eine Box-Cox-Transformation verglichen.
Die beste Testgüte wird mit \texttt{log1p} erreicht, daher wird dieses Modell berichtet.
Die Zielvariable ist damit
$\text{log\_hwd} = \log(1 + \text{hazardous\_waste\_disposed}_{\text{total}})$.
Tabelle~\ref{tab:reg_hazardous_waste} fasst die Testleistung nach Rücktransformation zusammen.

\begin{table}[h]
  \centering
  \caption{Gütekennzahlen des linearen Regressionsmodells
           (\texttt{Hazardous waste disposed} als Zielvariable).}
  \label{tab:reg_hazardous_waste}
  \begin{tabular}{lc}
    \toprule
    Größe & Wert (Test) \\
    \midrule
    $R^2_{\text{Test}}$   & $0.813$ \\
    $\mathrm{RMSE}_{\text{Test}}$ & $18489.74\,\mathrm{kg}$ \\
    \midrule
    Median absoluter Fehler & $176.7890\,\mathrm{kg}$ \\
    $\mathrm{MdARE}_{\text{Test}}$ (Median rel. Fehler) & $0.6639$ \\
    $\mathrm{MARE}_{\text{Test}}$ (Mittelwert rel. Fehler) & $13.1431$ \\
    \bottomrule
  \end{tabular}
\end{table}

Das Modell erreicht eine gute Testgüte, allerdings ist die Fehlerverteilung stark schief.
Der RMSE wird durch wenige große Abweichungen dominiert, während der Medianfehler deutlich kleiner ausfällt.
Abbildung~\ref{fig:reg_hazardous_waste_scatter} zeigt die insgesamt gute Trendabbildung bei mittleren und großen Werten,
während kleine Zielwerte zu großen relativen Fehlern führen können.

\begin{figure}[h]
  \centering
  \includegraphics[width=0.9\textwidth]{images/regression_h_w_d.png}
  \caption{Vorhergesagte gegenüber tatsächlichen Werten des Indikators
           \emph{Hazardous waste disposed}. Beide Achsen sind logarithmisch skaliert.}
  \label{fig:reg_hazardous_waste_scatter}
\end{figure}

Der QQ Plot in Abbildung~\ref{fig:reg_hazardous_waste_qq} zeigt eine näherungsweise Linearität im Zentrum,
mit Abweichungen in den Randbereichen, was auf schwere Verteilungsschwänze hinweist.

\begin{figure}[h]
  \centering
  \includegraphics[width=0.95\textwidth]{images/regression_hwd_qq_new.png}
  \caption{QQ Plot der Schätzfehler des \emph{Hazardous waste disposed} Modells auf der Transformationsskala.}
  \label{fig:reg_hazardous_waste_qq}
\end{figure}


\FloatBarrier
\section{Regression des Indikators Water use}
\label{sec:reg_water_use}
\FloatBarrier

Für den Indikator \emph{Water use} wird dieselbe Regressionspipeline verwendet.
Es wurden keine Transformation, \texttt{log1p} und eine Box-Cox-Transformation verglichen.
Die beste Testgüte wird mit Box-Cox erreicht, daher wird dieses Modell berichtet.
Tabelle~\ref{tab:reg_water_use} fasst die Testleistung nach Rücktransformation zusammen.

\begin{table}[h]
  \centering
  \caption{Gütekennzahlen des linearen Regressionsmodells
           (\texttt{Water use} als Zielvariable).}
  \label{tab:reg_water_use}
  \begin{tabular}{lc}
    \toprule
    Größe & Wert (Test) \\
    \midrule
    $R^2_{\text{Test}}$   & $0{,}726$ \\
    $\mathrm{RMSE}_{\text{Test}}$ & $28682{,}36$ \\
    \midrule
    Median absoluter Fehler & $1353{,}5625$ \\
    $\mathrm{MdARE}_{\text{Test}}$ (Median rel. Fehler) & $0{,}9297$ \\
    $\mathrm{MARE}_{\text{Test}}$ (Mittelwert rel. Fehler) & $2{,}3278$ \\
    \bottomrule
  \end{tabular}
\end{table}

Die Testgüte ist moderat, mit deutlich größerer Streuung als bei den stärksten Indikatoren.
Abbildung~\ref{fig:reg_water_use_scatter} zeigt eine tendenzielle Unterschätzung im oberen Wertebereich.
Die Fehlerverteilung ist durch Ausreißer geprägt, was sich auch im QQ Plot widerspiegelt.

\begin{figure}[h]
  \centering
  \includegraphics[width=0.9\textwidth]{images/regression_water_use.png}
  \caption{Vorhergesagte gegenüber tatsächlichen Werten des Indikators
           \emph{Water use}. Beide Achsen sind logarithmisch skaliert.}
  \label{fig:reg_water_use_scatter}
\end{figure}

\begin{figure}[h]
  \centering
  \includegraphics[width=0.95\textwidth]{images/regression_water_use_qq.png}
  \caption{QQ Plot der Residuen des \emph{Water use} Modells auf der Transformationsskala.}
  \label{fig:reg_water_use_qq}
\end{figure}


\FloatBarrier
\section{Regression des Indikators Photochemical ozone formation (HH)}
\label{sec:reg_pof_hh}
\FloatBarrier

Für den Indikator \emph{Photochemical ozone formation, human health}
(\texttt{photochemical\_ozone\_formation\_hh}) wurden keine Transformation, \texttt{log1p}
und Box-Cox verglichen. Die beste Testgüte wird mit Box-Cox erreicht, daher wird dieses Modell berichtet.
Es konnten $n = 171$ PEPs verwendet werden.
Tabelle~\ref{tab:reg_pof_hh} fasst die Testleistung nach Rücktransformation zusammen.

\begin{table}[h]
  \centering
  \caption{Gütekennzahlen des linearen Regressionsmodells
           (\texttt{Photochemical ozone formation, human health} als Zielvariable).}
  \label{tab:reg_pof_hh}
  \begin{tabular}{lc}
    \toprule
    Größe & Wert (Test) \\
    \midrule
    $R^2_{\text{Test}}$   & $0{,}802$ \\
    $\mathrm{RMSE}_{\text{Test}}$ & $109{,}1708\,\mathrm{kg\ C_2H_4}$ \\
    \midrule
    Median absoluter Fehler & $0{,}0255\,\mathrm{kg\ C_2H_4}$ \\
    $\mathrm{MdARE}_{\text{Test}}$ (Median rel. Fehler) & $0{,}6665$ \\
    $\mathrm{MARE}_{\text{Test}}$ (Mittelwert rel. Fehler) & $1{,}2811$ \\
    \bottomrule
  \end{tabular}
\end{table}

Die Vorhersagen erfassen den Haupttrend gut, mit größerer Streuung bei kleinen Werten
und einer leichten Unterschätzung großer Zielwerte, vgl. Abbildung~\ref{fig:reg_pof_hh_scatter}.
Die Residuen zeigen in den Randbereichen deutliche Abweichungen von Normalität, vgl. Abbildung~\ref{fig:reg_pof_hh_qq}.

\begin{figure}[h]
  \centering
  \includegraphics[width=0.9\textwidth]{images/regression_pof.png}
  \caption{Vorhergesagte gegenüber tatsächlichen Werten des Indikators
           \emph{Photochemical ozone formation, human health}. Beide Achsen sind logarithmisch skaliert.}
  \label{fig:reg_pof_hh_scatter}
\end{figure}

\begin{figure}[h]
  \centering
  \includegraphics[width=0.95\textwidth]{images/regression_pof_qq.png}
  \caption{QQ Plot der Residuen des \emph{Photochemical ozone formation, human health} Modells auf der Transformationsskala.}
  \label{fig:reg_pof_hh_qq}
\end{figure}


\FloatBarrier
\section{Regression des Indikators Resource use, fossils}
\label{sec:reg_ruf}
\FloatBarrier

Für den Indikator \emph{Resource use, fossils} wurden keine Transformation, \texttt{log1p}
und Box-Cox verglichen. Die beste Testgüte wird mit \texttt{log1p} erreicht, daher wird dieses Modell berichtet.
Tabelle~\ref{tab:reg_ru_f} fasst die Testleistung nach Rücktransformation zusammen.

\begin{table}[h]
  \centering
  \caption{Gütekennzahlen des linearen Regressionsmodells
           (\texttt{Resource use, fossils} als Zielvariable).}
  \label{tab:reg_ru_f}
  \begin{tabular}{lc}
    \toprule
    Größe & Wert (Test) \\
    \midrule
    $R^2_{\text{Test}}$   & $0{,}871$ \\
    $\mathrm{RMSE}_{\text{Test}}$ & $1119662{,}28\,\mathrm{MJ}$ \\
    \midrule
    Median absoluter Fehler & $14460{,}95\,\mathrm{MJ}$ \\
    $\mathrm{MdARE}_{\text{Test}}$ (Median rel. Fehler) & $0{,}6370$ \\
    $\mathrm{MARE}_{\text{Test}}$ (Mittelwert rel. Fehler) & $11{,}1394$ \\
    \bottomrule
  \end{tabular}
\end{table}

Die Testgüte ist hoch, der RMSE wird jedoch durch wenige große Abweichungen dominiert.
Abbildung~\ref{fig:reg_ruf_scatter} zeigt eine enge Punktwolke entlang der Diagonalen
mit einzelnen starken Überschätzungen. Der QQ Plot weist auf schwere Verteilungsschwänze hin, vgl. Abbildung~\ref{fig:reg_ruf_qq}.

\begin{figure}[h]
  \centering
  \includegraphics[width=0.9\textwidth]{images/regression_ruf.png}
  \caption{Vorhergesagte gegenüber tatsächlichen Werten des Indikators
           \emph{Resource use, fossils}. Beide Achsen sind logarithmisch skaliert.}
  \label{fig:reg_ruf_scatter}
\end{figure}

\begin{figure}[h]
  \centering
  \includegraphics[width=0.95\textwidth]{images/regression_ruf_qq.png}
  \caption{QQ Plot der Residuen des \emph{Resource use, fossils} Modells auf der Transformationsskala.}
  \label{fig:reg_ruf_qq}
\end{figure}


\FloatBarrier
\section{Regression des Indikators Eutrophication (terrestrial)}
\label{sec:reg_eutrophication_terr}
\FloatBarrier

Für den Indikator \emph{Eutrophication, terrestrial} konnten $n = 107$ PEPs verwendet werden.
Es wurden keine Transformation, \texttt{log1p} und Box-Cox verglichen. Die beste Testgüte wird mit Box-Cox erreicht.
Tabelle~\ref{tab:reg_eutrophication_t} fasst die Testleistung nach Rücktransformation zusammen.

\begin{table}[h]
  \centering
  \caption{Gütekennzahlen des linearen Regressionsmodells
           (\texttt{Eutrophication, terrestrial} als Zielvariable).}
  \label{tab:reg_eutrophication_t}
  \begin{tabular}{lc}
    \toprule
    Größe & Wert (Test) \\
    \midrule
    $R^2_{\text{Test}}$   & $0{,}793$ \\
    $\mathrm{RMSE}_{\text{Test}}$ & $99{,}33\,\mathrm{mol\ N}$ \\
    \midrule
    Median absoluter Fehler & $0{,}0854\,\mathrm{mol\ N}$ \\
    $\mathrm{MdARE}_{\text{Test}}$ (Median rel. Fehler) & $0{,}5611$ \\
    $\mathrm{MARE}_{\text{Test}}$ (Mittelwert rel. Fehler) & $1{,}1653$ \\
    \bottomrule
  \end{tabular}
\end{table}

Das Modell erreicht eine solide Testgüte, mit erhöhter Streuung bei kleinen Werten.
Im oberen Wertebereich ist eine leichte Unterschätzung sichtbar, vgl. Abbildung~\ref{fig:reg_et_scatter}.
Der QQ Plot zeigt Abweichungen in den Randbereichen, vgl. Abbildung~\ref{fig:reg_et_qq}.

\begin{figure}[h]
  \centering
  \includegraphics[width=0.9\textwidth]{images/regression_et.png}
  \caption{Vorhergesagte gegenüber tatsächlichen Werten des Indikators
           \emph{Eutrophication, terrestrial}. Beide Achsen sind logarithmisch skaliert.}
  \label{fig:reg_et_scatter}
\end{figure}

\begin{figure}[h]
  \centering
  \includegraphics[width=0.95\textwidth]{images/regression_et_qq.png}
  \caption{QQ Plot der Residuen des \emph{Eutrophication, terrestrial} Modells auf der Transformationsskala.}
  \label{fig:reg_et_qq}
\end{figure}


\FloatBarrier
\section{Regression des Indikators Ozone depletion}
\label{sec:reg_ozone_depletion}
\FloatBarrier

Für den Indikator \emph{Ozone depletion} wurden keine Transformation, \texttt{log1p} und Box-Cox verglichen.
Die beste Testgüte wird mit Box-Cox erreicht, daher wird dieses Modell berichtet.
Es konnten $n = 170$ PEPs benutzt werden.
Tabelle~\ref{tab:reg_ozone_depletion} fasst die Testleistung nach Rücktransformation zusammen.

\begin{table}[h]
  \centering
  \caption{Gütekennzahlen des linearen Regressionsmodells
           (\texttt{Ozone depletion} als Zielvariable).}
  \label{tab:reg_ozone_depletion}
  \begin{tabular}{lc}
    \toprule
    Größe & Wert (Test) \\
    \midrule
    $R^2_{\text{Test}}$ & $0.858$ \\
    $\mathrm{RMSE}_{\text{Test}}$ & $0.0029$ \\
    \midrule
    Median absoluter Fehler & $0.0000$ \\
    $\mathrm{MdARE}_{\text{Test}}$ (Median rel. Fehler) & $0.8635$ \\
    $\mathrm{MARE}_{\text{Test}}$ (Mittelwert rel. Fehler) & $1.0411$ \\
    \bottomrule
  \end{tabular}
\end{table}

Der Indikator liegt auf einer sehr kleinen Skala, daher erscheinen absolute Fehler in gerundeter Darstellung teilweise als Null.
Relativ betrachtet zeigen sich dennoch Ausreißer, insbesondere bei sehr kleinen Zielwerten.
Abbildung~\ref{fig:reg_ozone_depletion_scatter} und Abbildung~\ref{fig:reg_ozone_depletion_qq} verdeutlichen die Streuung
und eine rechtsschiefe Fehlerstruktur in den oberen Quantilen.

\begin{figure}[h]
  \centering
  \includegraphics[width=0.9\textwidth]{images/regression_od.png}
  \caption{Vorhergesagte gegenüber tatsächlichen Werten von \texttt{Ozone depletion}. Beide Achsen sind logarithmisch skaliert.}
  \label{fig:reg_ozone_depletion_scatter}
\end{figure}

\begin{figure}[h]
  \centering
  \includegraphics[width=0.95\textwidth]{images/regression_od_qq.png}
  \caption{QQ Plot der Schätzfehler des \emph{Ozone depletion} Modells auf der Transformationsskala.}
  \label{fig:reg_ozone_depletion_qq}
\end{figure}


\FloatBarrier
\section{Regression des Indikators Resource use, minerals and metals}
\label{sec:reg_ru_mm}
\FloatBarrier

Für den Indikator \emph{Resource use, minerals and metals} wurden keine Transformation, \texttt{log1p} und Box-Cox verglichen.
Die beste Testgüte wurde mit Box-Cox erreicht, daher wird dieses Modell berichtet.
Tabelle~\ref{tab:reg_ru_mm} fasst die Testleistung nach Rücktransformation auf die Originalskala zusammen.

\begin{table}[h]
  \centering
  \caption{Gütekennzahlen des linearen Regressionsmodells
           (\texttt{Resource use, minerals and metals} als Zielvariable).}
  \label{tab:reg_ru_mm}
  \begin{tabular}{lc}
    \toprule
    Größe & Wert (Test) \\
    \midrule
    $R^2_{\text{Test}}$ & $0.866$ \\
    $\mathrm{RMSE}_{\text{Test}}$ (Originalskala) & $0.7218$ \\
    \midrule
    Median absoluter Fehler & $0.0010$ \\
    $\mathrm{MdARE}_{\text{Test}}$ (Median rel. Fehler) & $0.7798$ \\
    $\mathrm{MARE}_{\text{Test}}$ (Mittelwert rel. Fehler) & $1.0003$ \\
    \bottomrule
  \end{tabular}
\end{table}

Die Testgüte ist hoch und der Haupttrend wird gut abgebildet.
Einzelne Ausreißer bleiben bestehen, vgl. Abbildung~\ref{fig:reg_ru_mm_scatter}, und die Residuen zeigen schwere Schwänze, vgl. Abbildung~\ref{fig:reg_ru_mm_qq}.

\begin{figure}[h]
  \centering
  \includegraphics[width=0.9\textwidth]{images/regression_rumm.png}
  \caption{Vorhergesagte gegenüber tatsächlichen Werten des Indikators
           \emph{Resource use, minerals and metals}. Beide Achsen sind logarithmisch skaliert.}
  \label{fig:reg_ru_mm_scatter}
\end{figure}

\begin{figure}[h]
  \centering
  \includegraphics[width=0.95\textwidth]{images/regression_rumm_qq.png}
  \caption{QQ Plot der Residuen des Modells für \emph{Resource use, minerals and metals} auf der Transformationsskala.}
  \label{fig:reg_ru_mm_qq}
\end{figure}
