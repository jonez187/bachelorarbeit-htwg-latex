% appendix-a.tex
\chapter{Anhang}

\section{Visualisierung weiterer Regressionsmodelle (Hohe Modellgüte)}
\label{app:other_regressions}

Die folgenden Abschnitte ergänzen die im Haupttext berichteten Ergebnisse um zusätzliche
Indikatoren. Dargestellt sind jeweils
Streudiagramme und QQ Plots zur Einordnung der Fehlerstruktur.

\FloatBarrier
\subsection{Regression des Indikators Hazardous waste disposed}
\label{sec:reg_hazardous_waste}
\FloatBarrier

Das Modell erreicht eine gute Testgüte, allerdings ist die Fehlerverteilung stark schief.
Der RMSE wird durch wenige große Abweichungen dominiert, während der Medianfehler deutlich kleiner ausfällt.
Abbildung~\ref{fig:reg_hazardous_waste_scatter} zeigt die insgesamt gute Trendabbildung bei mittleren und großen Werten,
während kleine Zielwerte zu großen relativen Fehlern führen können.

\begin{figure}[h]
  \centering
  \includegraphics[width=0.9\textwidth]{images/regression_h_w_d.png}
  \caption{Vorhergesagte gegenüber tatsächlichen Werten des Indikators
           \emph{Hazardous waste disposed}. Beide Achsen sind logarithmisch skaliert.}
  \label{fig:reg_hazardous_waste_scatter}
\end{figure}

Der QQ Plot in Abbildung~\ref{fig:reg_hazardous_waste_qq} zeigt eine näherungsweise Linearität im Zentrum,
mit Abweichungen in den Randbereichen, was auf schwere Verteilungsschwänze hinweist.

\begin{figure}[h]
  \centering
  \includegraphics[width=0.95\textwidth]{images/regression_hwd_qq_new.png}
  \caption{QQ Plot der Schätzfehler des \emph{Hazardous waste disposed} Modells auf der Transformationsskala.}
  \label{fig:reg_hazardous_waste_qq}
\end{figure}

Die zugehörigen Gütekennzahlen und die gewählte Zieltransformation sind in Tabelle~\ref{tab:reg_other_overview} zusammengefasst.

\FloatBarrier
\subsection{Regression des Indikators Water use}
\label{sec:reg_water_use}
\FloatBarrier

Die Testgüte ist moderat, mit deutlich größerer Streuung als bei den stärksten Indikatoren.
Abbildung~\ref{fig:reg_water_use_scatter} zeigt eine tendenzielle Unterschätzung im oberen Wertebereich.
Die Fehlerverteilung ist durch Ausreißer geprägt, was sich auch im QQ Plot widerspiegelt.

\begin{figure}[h]
  \centering
  \includegraphics[width=0.9\textwidth]{images/regression_water_use.png}
  \caption{Vorhergesagte gegenüber tatsächlichen Werten des Indikators
           \emph{Water use}. Beide Achsen sind logarithmisch skaliert.}
  \label{fig:reg_water_use_scatter}
\end{figure}

\begin{figure}[h]
  \centering
  \includegraphics[width=0.95\textwidth]{images/regression_water_use_qq.png}
  \caption{QQ Plot der Schätzfehler des \emph{Water use} Modells auf der Transformationsskala.}
  \label{fig:reg_water_use_qq}
\end{figure}

Die zugehörigen Gütekennzahlen und die gewählte Zieltransformation sind in Tabelle~\ref{tab:reg_other_overview} zusammengefasst.

\FloatBarrier
\subsection{Regression des Indikators Photochemical ozone formation (HH)}
\label{sec:reg_pof_hh}
\FloatBarrier

Die Vorhersagen erfassen den Haupttrend gut, mit größerer Streuung bei kleinen Werten
und einer leichten Unterschätzung großer Zielwerte, vgl. Abbildung~\ref{fig:reg_pof_hh_scatter}.
Die Schätzfehler zeigen in den Randbereichen deutliche Abweichungen von Normalität, vgl. Abbildung~\ref{fig:reg_pof_hh_qq}.

\begin{figure}[h]
  \centering
  \includegraphics[width=0.9\textwidth]{images/regression_pof.png}
  \caption{Vorhergesagte gegenüber tatsächlichen Werten des Indikators
           \emph{Photochemical ozone formation, human health}. Beide Achsen sind logarithmisch skaliert.}
  \label{fig:reg_pof_hh_scatter}
\end{figure}

\begin{figure}[h]
  \centering
  \includegraphics[width=0.95\textwidth]{images/regression_pof_qq.png}
  \caption{QQ Plot der Schätzfehler des \emph{Photochemical ozone formation, human health} Modells auf der Transformationsskala.}
  \label{fig:reg_pof_hh_qq}
\end{figure}

Die zugehörigen Gütekennzahlen und die gewählte Zieltransformation sind in Tabelle~\ref{tab:reg_other_overview} zusammengefasst.

\FloatBarrier
\subsection{Regression des Indikators Resource use, fossils}
\label{sec:reg_ruf}
\FloatBarrier

Abbildung~\ref{fig:reg_ruf_scatter} zeigt eine enge Punktwolke entlang der Diagonalen
mit einzelnen starken Überschätzungen. Der QQ Plot weist auf schwere Verteilungsschwänze hin, vgl. Abbildung~\ref{fig:reg_ruf_qq}.

\begin{figure}[h]
  \centering
  \includegraphics[width=0.9\textwidth]{images/regression_ruf.png}
  \caption{Vorhergesagte gegenüber tatsächlichen Werten des Indikators
           \emph{Resource use, fossils}. Beide Achsen sind logarithmisch skaliert.}
  \label{fig:reg_ruf_scatter}
\end{figure}

\begin{figure}[h]
  \centering
  \includegraphics[width=0.95\textwidth]{images/regression_ruf_qq.png}
  \caption{QQ Plot der Schätzfehler des \emph{Resource use, fossils} Modells auf der Transformationsskala.}
  \label{fig:reg_ruf_qq}
\end{figure}

Die zugehörigen Gütekennzahlen und die gewählte Zieltransformation sind in Tabelle~\ref{tab:reg_other_overview} zusammengefasst.

\FloatBarrier
\subsection{Regression des Indikators Eutrophication (terrestrial)}
\label{sec:reg_eutrophication_terr}
\FloatBarrier

Das Modell erreicht eine solide Testgüte, mit erhöhter Streuung bei kleinen Werten.
Im oberen Wertebereich ist eine leichte Unterschätzung sichtbar, vgl. Abbildung~\ref{fig:reg_et_scatter}.
Der QQ Plot zeigt Abweichungen in den Randbereichen, vgl. Abbildung~\ref{fig:reg_et_qq}.

\begin{figure}[h]
  \centering
  \includegraphics[width=0.9\textwidth]{images/regression_et.png}
  \caption{Vorhergesagte gegenüber tatsächlichen Werten des Indikators
           \emph{Eutrophication, terrestrial}. Beide Achsen sind logarithmisch skaliert.}
  \label{fig:reg_et_scatter}
\end{figure}

\begin{figure}[h]
  \centering
  \includegraphics[width=0.95\textwidth]{images/regression_et_qq.png}
  \caption{QQ Plot der Schätzfehler des \emph{Eutrophication, terrestrial} Modells auf der Transformationsskala.}
  \label{fig:reg_et_qq}
\end{figure}

Die zugehörigen Gütekennzahlen und die gewählte Zieltransformation sind in Tabelle~\ref{tab:reg_other_overview} zusammengefasst.

\FloatBarrier
\subsection{Regression des Indikators Ozone depletion}
\label{sec:reg_ozone_depletion}
\FloatBarrier


Der Indikator liegt auf einer sehr kleinen Skala, daher erscheinen absolute Fehler in gerundeter Darstellung teilweise als Null.
Relativ betrachtet zeigen sich dennoch Ausreißer, insbesondere bei sehr kleinen Zielwerten.
Abbildung~\ref{fig:reg_ozone_depletion_scatter} und Abbildung~\ref{fig:reg_ozone_depletion_qq} verdeutlichen die Streuung
und eine rechtsschiefe Fehlerstruktur in den oberen Quantilen.

\begin{figure}[h]
  \centering
  \includegraphics[width=0.9\textwidth]{images/regression_od.png}
  \caption{Vorhergesagte gegenüber tatsächlichen Werten von \texttt{Ozone depletion}. Beide Achsen sind logarithmisch skaliert.}
  \label{fig:reg_ozone_depletion_scatter}
\end{figure}

\begin{figure}[h]
  \centering
  \includegraphics[width=0.95\textwidth]{images/regression_od_qq.png}
  \caption{QQ Plot der Schätzfehler des \emph{Ozone depletion} Modells auf der Transformationsskala.}
  \label{fig:reg_ozone_depletion_qq}
\end{figure}

Die zugehörigen Gütekennzahlen und die gewählte Zieltransformation sind in Tabelle~\ref{tab:reg_other_overview} zusammengefasst.

\FloatBarrier
\subsection{Regression des Indikators Resource use, minerals and metals}
\label{sec:reg_ru_mm}
\FloatBarrier


Die Testgüte ist hoch und der Haupttrend wird gut abgebildet.
Einzelne Ausreißer bleiben bestehen, vgl. Abbildung~\ref{fig:reg_ru_mm_scatter}, und die Schätzfehler
zeigen schwere Schwänze, vgl. Abbildung~\ref{fig:reg_ru_mm_qq}.

\begin{figure}[h]
  \centering
  \includegraphics[width=0.9\textwidth]{images/regression_rumm.png}
  \caption{Vorhergesagte gegenüber tatsächlichen Werten des Indikators
           \emph{Resource use, minerals and metals}. Beide Achsen sind logarithmisch skaliert.}
  \label{fig:reg_ru_mm_scatter}
\end{figure}

\begin{figure}[h]
  \centering
  \includegraphics[width=0.95\textwidth]{images/regression_rumm_qq.png}
  \caption{QQ Plot der Schätzfehler des Modells für \emph{Resource use, minerals and metals} auf der Transformationsskala.}
  \label{fig:reg_ru_mm_qq}
\end{figure}

Die zugehörigen Gütekennzahlen und die gewählte Zieltransformation sind in Tabelle~\ref{tab:reg_other_overview} zusammengefasst.


\section{Visualisierung weiterer Regressionsmodelle (Geringe Modellgüte)}
\label{app:other_regressions_weak}

\FloatBarrier
\subsection{Regression des Indikators Eutrophication (freshwater)}
\label{sec:reg_eutrophication_freshwater}
\FloatBarrier

Für den Indikator \emph{Eutrophication (freshwater)} wird die zuvor aufgebaute
Regressionspipeline analog angewendet.
Trotz des einheitlichen Modellansatzes fällt die Vorhersagegüte deutlich geringer aus als
bei den besser erklärbaren Zielgrößen.
Auf dem Testdatensatz wird nur ein begrenzter Anteil der Varianz erklärt
($R^2_{\text{Test}} = 0.434$ bei $n = 133$).

Die robusten Fehlermaße verdeutlichen die Instabilität der Vorhersagen.
Der Median der relativen Fehler liegt bei $\mathrm{MdARE}\approx 5.04$,
der Mittelwert bei $\mathrm{MARE}\approx 208.43$.
Damit treten sehr große relative Abweichungen auf.

Insgesamt ist \emph{Eutrophication (freshwater)} mit dem gewählten Feature-Set aus Gewicht,
Stromverbrauch und Material-PCA nur eingeschränkt erklärbar.

Zur Veranschaulichung zeigt Abbildung~\ref{fig:reg_efw_scatter} das Streudiagramm der vorhergesagten
gegenüber den tatsächlichen Werten für \emph{Eutrophication (freshwater)} auf logarithmischen Achsen.
Es ist erkennbar, dass das Modell große Teile der Zielwerte nur in einem relativ engen Band vorhersagt.
Sehr kleine tatsächliche Werte werden deutlich überschätzt, während sehr große Werte tendenziell
unterschätzt werden. Damit bildet das Modell eher einen mittleren Größenbereich ab, als die volle
Spannweite der Daten, was zur geringen erklärten Varianz und zu den hohen relativen Fehlern passt.

\begin{figure}[h]
  \centering
  \includegraphics[width=0.9\textwidth]{images/regression_efw.png}
  \caption{Vorhergesagte gegenüber tatsächlichen Werten des Indikators
           \emph{Eutrophication (freshwater)}. Beide Achsen sind logarithmisch skaliert.}
  \label{fig:reg_efw_scatter}
\end{figure}

Abbildung~\ref{fig:reg_efw_qq} zeigt den QQ Plot der Residuen auf der Transformationsskala.
Ein Teil der Punkte folgt im Zentrum näherungsweise der Referenzgeraden. Die meisten weichen jedoch, 
insbesondere in den äußeren Quantilen, deutlich ab. Dies spricht gegen eine
Normalverteilung der Fehler und deutet auf schwere Verteilungsschwänze sowie ausgeprägte Ausreißer hin.

\begin{figure}[h]
  \centering
  \includegraphics[width=0.95\textwidth]{images/regression_efw_qq.png}
  \caption{QQ Plot der Schätzfehler des Modells für \emph{Eutrophication (freshwater)}
           auf der Transformationsskala.}
  \label{fig:reg_efw_qq}
\end{figure}


\FloatBarrier
\subsection{Regression des Indikators Eutrophication Marine}
\label{sec:reg_eutrophication_marine}
\FloatBarrier

Für den Indikator \emph{Eutrophication (marine)} wurde dieselbe Regressionspipeline wie zuvor
angewendet. Auch hier bleibt die Testgüte niedrig, und die relativen Fehler sind extrem.
Tabelle~\ref{tab:reg_em} fasst die Gütemaße zusammen.

\begin{table}[h]
  \centering
  \caption{Gütekennzahlen des linearen Regressionsmodells
           (\texttt{Eutrophication (marine)} als Zielvariable).}
  \label{tab:reg_em}
  \begin{tabular}{lc}
    \toprule
    Größe & Wert (Test) \\
    \midrule
    $R^2_{\text{Test}}$   & $0.322$ \\
    $\mathrm{RMSE}_{\text{Test}}$  & $25.0197$ \\
    Rel.\ RMSE ($\mathrm{RMSE}/\overline{y}$) & $1.7841$ \\
    \midrule
    $\mathrm{MdARE}_{\text{Test}}$ (Median rel.\ Fehler) & $7.4875$ \\
    $\mathrm{MARE}_{\text{Test}}$ (Mittelwert rel.\ Fehler) & $186.5045$ \\
    RMSE ohne Top 20 & $1.2260$ \\
    \bottomrule
  \end{tabular}
\end{table}

Die geringe erklärte Varianz zeigt, dass Gewicht, Stromverbrauch und die verdichteten
Materialinformationen die Streuung dieses Indikators nur unzureichend abbilden.
Die sehr hohen relativen Fehler deuten zusätzlich auf starke Instabilität hin,
insbesondere bei kleinen Zielwerten.
Dass der RMSE ohne die Top 20 Beobachtungen deutlich kleiner ausfällt,
spricht dafür, dass wenige extreme Fälle die Fehlerstatistik dominieren.

Abbildung~\ref{fig:reg_em_scatter} zeigt, dass das Modell den groben Trend nur schwach trifft
und viele Punkte weit von der Ideallinie entfernt liegen, selbst auf logarithmischer Skala.
Der QQ Plot in Abbildung~\ref{fig:reg_em_qq} bestätigt eine deutliche Abweichung von der
Normalverteilung auf der Transformationsskala, was auf schwere Schwänze und systematische
Modellfehler hinweist.

\begin{figure}[h]
  \centering
  \includegraphics[width=0.9\textwidth]{images/regression_em.png}
  \caption{Vorhergesagte gegenüber tatsächlichen Werten des Indikators
           \emph{Eutrophication (marine)}. Beide Achsen sind logarithmisch skaliert.}
  \label{fig:reg_em_scatter}
\end{figure}

\begin{figure}[h]
  \centering
  \includegraphics[width=0.95\textwidth]{images/regression_em_qq.png}
  \caption{QQ Plot der Schätzfehler des Modells für \emph{Eutrophication (marine)}
           auf der Transformationsskala.}
  \label{fig:reg_em_qq}
\end{figure}
