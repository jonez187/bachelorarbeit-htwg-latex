\chapter{Einleitung}

\section{Motivation und Relevanz}
\label{sec:motivation}

Gebäudeautomation und IoT-Komponenten gewinnen stark an Bedeutung. Sensoren, Gateways
und Steuerungen werden in Gebäuden zunehmend eingesetzt, um Energieflüsse zu optimieren,
Komfort zu erhöhen und Prozesse zu automatisieren. Mit der wachsenden Verbreitung solcher
Geräte steigt auch ihr Anteil an Materialverbrauch, Energieeinsatz und Abfallaufkommen
entlang des Lebenszyklus.

Der Gebäudesektor zählt neben Verkehr und der industriellen Produktion zu den wesentlichen
Verursachern von CO$_2$-Emissionen. 
In einer Studie von Bitkom e.V., dem Branchenverband der 
deutschen Informations- und Telekommunikationsbranche,
wird ein Anteil von etwa einem Drittel dem Ge\-bäu\-de\-sektor zugerechnet \cite{studie-einleitung}. 
Di\-gi\-tale Ge\-bäu\-de\-tech\-no\-lo\-gien werden in dieser Studie als relevanter Hebel betrachtet, 
durch bedarfsgerechte, intelligente Steuerung und fortgeschrittenem Monitoring 
den Energieverbrauch und die Emissionen von Gebäuden zu senken.

Neben eines erhöhten Komforts können Gebäudeautomation und IoT-Komponenten somit tatsächlich 
ökonomischen und ökologischen Nutzen liefern.

Die Größen\-ordnung dieses Potenzials lässt sich anhand einer Szenariorechnung aus der
Bitkom-Studie verdeutlichen. Abbildung~\ref{fig:bitkom_potential} zeigt die modellierten
CO$_2$-Min\-derungs\-poten\-ziale di\-gi\-taler Ge\-bäu\-de\-tech\-no\-logi\-en im Ge\-bäu\-de\-sek\-tor in Deutschland,
wobei die Ge\-bäu\-de\-au\-to\-mation gemäß DIN~EN~15232 in Ef\-fi\-zienz\-klassen eingeteilt wird.
Klasse~D dient dabei als Referenz und entspricht Gebäuden ohne automatische Regelung. 

\begin{figure}[ht]
  \centering
  \includegraphics[width=\textwidth]{images/studie-grafik.png}
  \caption{Einsparungen thermischer Energie nach Automationsklassen, relativ zur Klasse D (Zahlen aus der Studie \cite{studie-einleitung}).}
  \label{fig:bitkom_potential}
\end{figure}

Gleichzeitig sind die eingesetzten Geräte und die dadurch ermöglichten
Einsparungen ökologisch nicht „gratis“. Herstellung,
Materialeinsatz, Transport sowie die Nutzung verursachen eigene
Treibhausgasemissionen und weitere Umweltwirkungen. Damit entsteht ein Zielkonflikt:
Einerseits besteht ein Einsparpotenzial auf Gebäudeebene, andererseits müssen die
zusätzlichen Umweltwirkungen der Hardware transparent und vergleichbar quantifiziert
werden, um ökologische Hotspots zu identifizieren und fundierte Ent\-schei\-dungen in
Entwicklung, Beschaffung und Port\-folio\-manage\-ment zu er\-mög\-lichen. Besonders in frühen
Phasen liegen jedoch oft nur wenige robuste Produktmerkmale vor, während vollständige
Ökobilanzen typischerweise aufwendig sind und detaillierte Annahmen erfordern.

PEP-Ecopassport-Dokumente liefern standardisierte Umweltindikatoren und stellen damit
eine wertvolle Datenquelle dar. In der Praxis sind sie jedoch heterogen aufgebaut und
sind als PDF-Dateien veröffentlicht. Dadurch werden automatisierte Auswertung, Skalierung auf
größere Datenmengen und konsistente Vergleichbarkeit erschwert. Es entsteht eine Lücke
zwischen dem Bedarf an schnellen, nachvollziehbaren Umweltabschätzungen und der
tatsächlich verfügbaren Datengrundlage.

Eine kürzlich abgeschlossene Bachelorarbeit von Selg \cite{Selg2025} entwickelte eine
Extraktionspipeline, mit der relevante Daten aus PEP-Ecopassport-PDF-Dateien automatisiert
ausgelesen und strukturiert gespeichert werden können, und führte erste statistische
Analysen durch. Die vorliegende Arbeit knüpft an diese Vorarbeit an, erweitert die
Pipeline für eine größere und robustere Datenbasis und nutzt die extrahierten Variablen
für eine vertiefte quantitative Analyse und einer Modellentwicklung. 


\FloatBarrier
\section{Zielsetzung der Arbeit}
\label{sec:intro_ziele}

Ziel dieser Arbeit ist der Aufbau einer durchgängigen Pipeline, die PEP-Ecopassport-
Dokumente automatisiert verarbeitet und in eine strukturierte, analysierbare Datenbasis
überführt. Die Pipeline wird so ausgelegt, dass sie heterogene PDF-Layouts robust
verarbeitet und eine skalierbare Datengrundlage für die nachfolgende Analyse
und Modellentwicklung bereitstellt.

Im Mittelpunkt steht die Ableitung eines robusten und in\-ter\-pretier\-baren Modells zur
heu\-ris\-tischen Ab\-schätzung von Um\-welt\-aus\-wir\-kungen, ins\-besondere der CO$_2$-Emissionen.
Dazu werden in den PEP-Daten wiederkehrende Muster zwischen wenigen, allgemein
verfügbaren Produktmerkmalen und den ausgewiesenen Umweltindikatoren identifiziert und
modelliert. Der Ansatz ist so gewählt, dass das Modell auch auf neue Produkte ohne
verfügbare PEPs übertragbar ist und eine erste Einordnung der zu erwartenden
Umweltauswirkungen ermöglicht.

Die Genauigkeit des Modells soll analysiert und bewertet werden.
Ergänzend sollen die Grenzen dieser Herangehensweise diskutiert, insbesondere im Hinblick
auf Datenheterogenität, Ausreißer sowie die eingeschränkte Übertragbarkeit auf einzelne
Produktkategorien und Nutzungsszenarien werden.




\section{Wissenschaftliche Fragestellung}
\label{sec:intro_research_questions}

Aus der Zielsetzung ergibt sich die folgenden zentrale Fragestellung:

\begin{quote}
\emph{Wie lassen sich Umweltwirkungen von Produkten der Gebäudeautomatisierung auf Basis von PEP-Ecopassport-
Deklarationen systematisch extrahieren und statistisch analysieren, und inwieweit lassen sich daraus
robuste, in\-ter\-pre\-tier\-bare Modelle ableiten, die Um\-welt\-in\-di\-ka\-to\-ren auch für Produkte ohne PEP
heuristisch abschätzen können?}      
\end{quote}

Die Beantwortung dieser Fragestellung soll wissenschaftliche Erkenntnisse über die zentralen 
Umweltindikatoren innerhalb einer heterogenen Produktgruppe liefern und zeigen,
welche Produktmerkmale die größten Erklärungsbeiträge leisten. Gleichzeitig wird untersucht, unter
welchen Voraussetzungen vereinfachte, interpretierbare Regressionsmodelle eine praktikable
Erstabschätzung ermöglichen, und wo die Grenzen liegen. 
Damit werden sowohl methodische Grundlagen für eine skalierbare Auswertung von
PEP-Daten als auch praktische Anhaltspunkte für datenbasierte Nachhaltigkeitsbewertungen in frühen
Entscheidungsphasen geschaffen


\section{Aufbau der Arbeit}
\label{sec:intro_structure}

Zur Beantwortung der Forschungsfrage wird ein methodischer Ablauf umgesetzt, der von der
theoretischen Einordnung über den Aufbau einer belastbaren Datenbasis bis zur Modellierung
und Diskussion der Ergebnisse reicht. Die Arbeit ist in fünf Kapitel gegliedert.

Kapitel~\ref{chapter:theo} beschreibt die theo\-re\-tischen Grund\-lagen. Zu\-nächst werden der PEP-Eco\-pass\-port
Standard und der typische Aufbau von PEP-Dokumenten erläutert. Anschließend werden
Herausforderungen der strukturierten Informationsextraktion aus PDF-Dokumenten sowie
geeignete Extraktionsansätze diskutiert. Das Kapitel schließt mit statistischen Grundlagen,
die für die spätere explorative Analyse, Transformationen und Regressionsmodelle benötigt
werden.

Kapitel~\ref{chapter:methodik} stellt die entwickelte Pipeline und die erzeugte Datenbasis dar. Es beschreibt die
Recherche und Auswahl geeigneter PEP-Dokumente, das PDF-Parsing und die Extraktion
strukturierter Inhalte, die Normalisierung zentraler Begriffe und Einheiten sowie die
Datenbereinigung und Validierung. Ziel ist eine konsistente, analysierbare Datenbasis als
Grundlage der Modellierung.

Kapitel~\ref{chapter:analyse} analysiert den erarbeiteten Datensatz und leitet daraus Modellentscheidungen ab.
Zunächst werden Vollständigkeit und Verteilungen der Input-Variablen sowie zentrale
Umweltindikatoren deskriptiv ausgewertet. Darauf aufbauend folgt eine explorative
Modellentwicklung, in der Feature-Sets, PCA-Varianten und lineare Regressionsverfahren
verglichen werden. Zusätzlich wird die Material-PCA beschrieben und interpretiert. Abschließend
werden Regressionsmodelle für \emph{Climate change (total)} sowie für weitere 
Indikatoren geschätzt und hinsichtlich Vorhersagegüte und Fehlerstruktur bewertet.

Kapitel~\ref{chapter:diskussion} fasst die zentralen Ergebnisse zusammen und ordnet sie im Kontext der
Forschungsfrage ein. Darüber hinaus werden Limitationen der Datenbasis und des Modellansatzes,
methodische und technische Grenzen sowie die Übertragbarkeit der heuristischen Schätzung
diskutiert. Das Kapitel schließt mit einem Ausblick auf mögliche Erweiterungen der Pipeline
und weiterführende Forschungsfragen.
