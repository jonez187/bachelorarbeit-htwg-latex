\chapter{Einleitung}

\section{Motivation und Relevanz}
\label{sec:motiation}

Gebäudeautomation und IoT Komponenten gewinnen stark an Bedeutung.
Sensoren, Gateways und Steuerungen werden in Gebäuden zunehmend eingesetzt, um
Energieflüsse zu optimieren, Komfort zu erhöhen und Prozesse zu automatisieren.
Mit der wachsenden Verbreitung solcher Geräte steigt auch ihr Anteil an
Materialverbrauch, Energieeinsatz und Abfallaufkommen entlang des Lebenszyklus.

Parallel dazu rücken Nachhaltigkeit, Berichtspflichten und vergleichbare
Umweltkennzahlen stärker in den Fokus. Unternehmen benötigen für Entscheidungen in
Entwicklung, Beschaffung und Portfolio Management zunehmend belastbare Indikatoren,
um ökologische Hotspots zu identifizieren und Maßnahmen zu priorisieren.
In frühen Phasen sind jedoch oft nur wenige Produktmerkmale bekannt, während eine
vollständige Ökobilanz typischerweise aufwendig ist und detaillierte Annahmen
erfordert.

PEP Ecopassport Dokumente liefern standardisierte Umweltindikatoren und stellen
damit eine wertvolle Datenquelle dar. In der Praxis sind sie jedoch heterogen
aufgebaut und liegen meist als PDF vor, wodurch eine automatisierte Auswertung und
eine konsistente Vergleichbarkeit erschwert werden. Damit entsteht eine Lücke
zwischen dem Bedarf an schnellen, nachvollziehbaren Umweltabschätzungen und der
tatsächlich verfügbaren Datengrundlage.

Diese Arbeit adressiert diese Lücke durch eine automatisierte Pipeline zur
Extraktion strukturierter Umweltindikatoren aus PEP Dokumenten und durch eine
Modellierung, die prüft, wie weit sich zentrale Indikatoren mit wenigen, leicht
messbaren Produktmerkmalen approximieren lassen. Ziel ist dabei eine robuste
heuristische Einordnung und nicht die exakte Rekonstruktion einzelner PEP Werte.


\section{Zielsetzung der Arbeit}
\label{sec:intro_objectives}

Ziel dieser Arbeit ist der Aufbau einer durchgängigen Pipeline, die PEP
Ecopassport Dokumente automatisiert verarbeitet und daraus eine strukturierte,
analysierbare Datenbasis erzeugt. Auf der Datenbasis aufbauend soll untersucht werden, in
welchem Umfang zentrale Umweltindikatoren aus wenigen, grundsätzlich auch ohne
PEP verfügbaren Produktmerkmalen erklärbar sind.

Im Zentrum steht also die Entwicklung eines Modells für
eine robuste heuristische Abschätzung, die als Orientierung und Einordnung
der Umweltauswirkungen von Smart-Home Produkten dienen kann. 
Ergänzend werden die Grenzen dieser
Herangehensweise sowie die Rolle von Datenheterogenität und Ausreißern
herausgearbeitet.


\section{Wissenschaftliche Fragestellung}
\label{sec:intro_research_questions}

Aus der Zielsetzung ergeben sich die folgenden Fragestellungen.

\begin{enumerate}
  \item Wie lassen sich heterogene PEP PDF Dokumente automatisiert in ein
        konsistentes, maschinenlesbares Datenformat überführen, das
        Umweltindikatoren, Materialinformationen und technische Merkmale
        vergleichbar macht?
  \item In welchem Ausmaß lassen sich die in PEPs ausgewiesenen
        Umweltindikatoren. insbesondere \texttt{Climate change (total)}. durch
        wenige Produktmerkmale wie Gewicht, Stromverbrauch und verdichtete
        Materialinformationen erklären.
  \item Welche Indikatoren sind mit dem gewählten Feature Set gut vorhersagbar
        und bei welchen Indikatoren bleibt die Modellgüte gering. Welche
        Fehlerstrukturen treten auf und welche Konsequenzen ergeben sich daraus
        für Modellannahmen und Interpretation.
\end{enumerate}



\section{Aufbau der Arbeit}
\label{sec:intro_structure}

Die Arbeit gliedert sich wie folgt. Kapitel~\ref{chapter:methodik} beschreibt
die Datenbasis und den Aufbau der Pipeline. von der PEP Recherche über Parsing,
LLM gestützte Extraktion bis zur Normalisierung und Bereinigung der Daten.
Kapitel~\ref{sec:regression_co2} untersucht anhand des Indikators
\texttt{Climate change (total)} die Regressionspipeline, Transformationen und
Fehlerdiagnostik. Kapitel~\ref{sec:reg_other_indicators} überträgt die
Herangehensweise auf weitere Umweltindikatoren und vergleicht die
resultierende Modellgüte. Kapitel~\ref{sec:discussion} ordnet die Ergebnisse
ein, diskutiert Unsicherheiten sowie Grenzen der Daten und Modelle. Das Fazit
fasst die zentralen Erkenntnisse zusammen und skizziert praktikable
Weiterentwicklungen.