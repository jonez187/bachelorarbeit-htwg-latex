\chapter{Theoretische Grundlagen}

Hier werden die theoretischen Grundlagen für die vorliegende Arbeit gelegt. 
Ausgangspunkt ist die Beobachtung, dass Smart-Home-/IoT-Produkte Umweltwirkungen nicht nur in der Nutzungsphase (Stromverbrauch), 
sondern ebenso durch Materialzusammensetzung, Fertigung, Distribution und Entsorgung verursachen. 
Für die standardisierte Berichterstattung solcher Wirkungen existieren deklarative Formate wie die PEP Ecopassports, 
die Indikatoren entlang des Lebenszyklus ausweisen. Damit diese Angaben für quantitative Analysen nutzbar werden, 
sind konsistente Begriffe, Einheiten und Moduldefinitionen ebenso erforderlich wie ein Verständnis zentraler 
statistischer Verfahren zur Muster- und Zusammenhangsanalyse. Dieses Kapitel führt daher zunächst in Struktur 
und Inhalte von PEP-Deklarationen ein und skizziert anschließend die methodischen Bausteine 
(u. a. Lineare Regression und PCA), die in den folgenden Kapiteln zur Reduktion von Variablen, 
zur Erklärung von Indikatorvarianz und zur Ableitung einer praxistauglichen Heuristik für Produkte ohne PEP eingesetzt werden.

\section{PEP-Ecopassport}
Was ist PEP-Ecopassport, was steht drin, was ist interessant für mich

\subsection{PEP-Standard}

Der \emph{PEP~Ecopassport\textsuperscript{\textregistered}} ist ein international anerkanntes Programm zur Erstellung standardisierter Umweltproduktdeklarationen für elektrische, 
elektronische sowie Heizungs-, Lüftungs-, Klima- und Kälteprodukte (HVAC). Träger des Programms ist die \emph{P.E.P.~Association}, eine gemeinnützige
Organisation, deren Ziel es ist, ein gemeinsames und verlässliches Referenzsystem für Umweltinformationen dieser Produktkategorien bereitzustellen. 
Das Programm versteht sich als Branchenspezialisierung innerhalb des Rahmens der \emph{Environmental Product Declarations (EPD)} gemäß ISO~14025 und der 
Lebenszyklusnormen nach ISO~14040, und basiert somit auf international festgelegten Normen. \cite{PEP}

Ein \emph{PEP~Ecopassport} ist somit eine \emph{Typ~III-Umweltdeklaration} im Sinne der ISO~14025. Diese Deklarationen basieren auf quantitativen Ergebnissen einer
Lebenszyklusanalyse (\emph{Life Cycle Assessment, LCA}) und dienen der vergleichenden Bewertung von Produkten mit identischer Funktion. 
Die Datenerhebung und Berechnung erfolgt nach vordefinierten Parametern, die in sogenannten \emph{Product~Category~Rules~(PCR)} und bei Bedarf in 
\emph{Product~Specific~Rules~(PSR)} festgelegt sind. Jede PEP-Deklaration unterliegt einer unabhängigen Überprüfung der angewandten Methodik und der zugrunde liegenden LCA-Daten. \cite{Hassanzadeh2013}

Das Programm zielt auf Transparenz und Vergleichbarkeit ab. Hersteller erhalten ein einheitliches Verfahren, um ökologische Leistungskennwerte ihrer 
Produkte objektiv und nachvollziehbar zu kommunizieren. Für Anwender, Beschaffer und Energieberater stellen die PEP-Daten eine 
verlässliche Grundlage für ökologische Bewertungen und Beschaffungsentscheidungen dar.

Die Teilnahme am PEP-Programm ist freiwillig, gewinnt jedoch in der Praxis an Bedeutung, da Umweltproduktdeklarationen zunehmend als Nachweis 
oder Auswahlkriterium in Ausschreibungen und Produktbewertungen herangezogen werden. Eine gesetzliche Verpflichtung zur Erstellung besteht bislang nur in Einzelfällen, 
beispielsweise in Frankreich, wenn ein Hersteller aktiv mit Umweltvorteilen wirbt.

Das PEP-Programm unterscheidet sich klar von unternehmensbezogenen Treibhausgas-Bilanzierungen: 
Es erfasst ausschließlich produktspezifische Umweltwirkungen entlang des Lebenszyklus und folgt dabei den methodischen Vorgaben der ISO~14040-Reihe. 
Für umfassende \emph{GHG-Assessments} auf Organisationsebene sind PEP-Daten daher nicht geeignet.
\cite{PEP}

\subsection{Aufbau typischer PEP-Dokumente}

Ein vollständiges PEP umfasst typischerweise etwa zehn Seiten und gliedert sich in mehrere inhaltlich definierte Abschnitte.

\paragraph{Titel- und Metadatenblatt}
Das Deckblatt enthält grundlegende Angaben zum Produkt (Name, Version, Sprache, Hersteller), zum Veröffentlichungs- und Revisionsdatum
sowie zum Status der Erklärung (z.\,B. \emph{in~review} oder \emph{verified}). Darüber hinaus sind Kontaktinformationen, Firmenadresse und 
Registrierungsnummer enthalten. 

\paragraph{Allgemeine Produktinformationen}
Dieser Abschnitt beschreibt die funktionale Einheit (\emph{functional~unit}) und die Referenzlebensdauer, hier meist 10~bis~20~Jahre. 
Weiterhin werden die Produktfunktion, Anwendungsbereiche und gegebenenfalls weitere Varianten aufgeführt. 

\paragraph{Materialzusammensetzung}
Die Zusammensetzung des Produkts wird tabellarisch nach Hauptgruppen ausgewiesen, z.\,B.\ Kunststoffe, Metalle und weitere Materialien 
(Elektronik, Sonstiges). Im Beispiel des ABB~EQ~Meter entfallen 28\,\% auf Kunststoffe, 53\,\% auf Metalle und 19\,\% auf weitere Komponenten. 
Diese Angaben ermöglichen eine spätere Aggregation der Stoffanteile in harmonisierten Datenstrukturen.

\paragraph{Szenarien und Lebenszyklusphasen}
PEP-Dokumente sind entlang der Phasen des Produktlebenszyklus strukturiert, die den Vorgaben der EN~15804 entsprechen:
\begin{itemize}
    \item \textbf{Herstellung (A1--A3)}: Produktion und Vormaterialien, modelliert mit landesspezifischem Strommix (z.\,B.\ italienischer Grid~Mix).
    \item \textbf{Distribution (A4)}: Transport vom Werk zum Markt; häufig standardisierte Annahmen (z.\,B.\ 1\,000~km Schiff, 3\,300~km Lkw).
    \item \textbf{Installation (A5)}: Montage, meist nur Verpackungsabfälle berücksichtigt.
    \item \textbf{Nutzungsphase (B)}: Betrieb des Geräts mit angegebenem Energieverbrauch, z.\,B.\ 126~kWh über 20~Jahre, basierend auf europäischem Netzstrommix.
    \item \textbf{End-of-Life (C1--C4)}: Entsorgungsszenario gemäß PCR-Vorgaben (Recycling-, Deponie-, Transportanteile).
    \item \textbf{Optionale Phase (D)}: Rückgewinnung und Wiederverwendung außerhalb des Systemgrenzenmodells.
\end{itemize}
In der weiteren Datenaufbereitung werden diese Phasen zu den Kategorien \emph{manufacturing}, \emph{distribution}, \emph{installation}, 
\emph{use} und \emph{end\_of\_life} zusammengefasst.

\paragraph{Umweltindikatoren}
Die Umweltwirkungen werden für jede Lebenszyklusphase sowie als Gesamtwert angegeben.  
Die für diese Arbeit relevanten Indikatoren sind in der Tabelle~\ref{tab:indicators_description} aufgeführt.

\begin{table}[h!]
\centering
\caption{Umweltindikatoren}
\label{tab:indicators_description}
\resizebox{\textwidth}{!}{
\begin{tabular}{ll}
\hline
\textbf{Indikator} & \textbf{Beschreibung} \\
\hline
Acidification & Versauerung von Böden und Gewässern durch säurebildende Emissionen \\
Climate Change (Fossil) & Treibhauspotenzial durch fossile CO$_2$-Emissionen \\
Climate Change (Land Use and Land Use Change) & Treibhauspotenzial infolge von Landnutzungsänderungen (LULUC) \\
Climate Change (Total) & Gesamtes Treibhauspotenzial aus allen Quellen \\
Eutrophication (Freshwater) & Nährstoffanreicherung in Binnengewässern \\
Eutrophication (Marine) & Nährstoffanreicherung in marinen Ökosystemen \\
Eutrophication (Terrestrial) & Nährstoffanreicherung in terrestrischen Ökosystemen \\
Hazardous Waste Disposed & Entsorgung gefährlicher Abfälle \\
Ozone Depletion & Abbau der stratosphärischen Ozonschicht durch FCKW-Emissionen \\
Photochemical Ozone Formation (Human Health) & Bildung von bodennahem Ozon (Sommersmog) \\
Radioactive Waste Disposed & Entsorgung radioaktiver Abfälle \\
Resource Use (Fossils) & Nutzung fossiler Energieressourcen \\
Resource Use (Minerals and Metals) & Verbrauch abiotischer Ressourcen (Metalle und Mineralien) \\
Water Use & Entnahme und Verbrauch von Frischwasser \\
\hline
\end{tabular}
}
\end{table}




\paragraph{Verifikations- und Anhangsangaben}
Im abschließenden Teil werden die angewendeten Regelwerke und Datenquellen genannt, z.\,B.\ \emph{PCR-ed3-EN-2015\_04\_02} und \emph{PSR-0005-ed2-EN-2016\_03\_29}, 
die eingesetzte Software (z.\,B.\ SimaPro~9.3 mit Ecoinvent~3.8) sowie die Verifizierungsstelle und deren Akkreditierungsnummer. 
Darüber hinaus enthält dieser Abschnitt Angaben zum \emph{Materialaufbau} und zum verwendeten \emph{Energie\-modell}. 
Die Materialzusammensetzung wird in der Regel als prozentuale Massenanteile nach Hauptgruppen (Kunststoffe, Metalle, Elektronik, Sonstiges) dargestellt,
teils in Tabellenform, teils grafisch als Kreisdiagramm. 
Das Energiemodell beschreibt die angenommenen Strommixe und Spannungsniveaus je Lebenszyklusphase, beispielsweise den nationalen Grid~Mix 
für die Herstellung und den europäischen Durchschnittsmix für die Nutzungsphase. 

Obwohl der inhaltliche Mindestumfang und die zu berichtenden Umweltindikatoren durch die zugrundeliegenden ISO- und PCR-Vorgaben festgelegt sind, besteht keine feste formale Struktur. 
Das Layout, die grafische Aufbereitung und die Anordnung der Tabellen können je nach Hersteller, Software und Version variieren. 
So enthalten einige PEPs tabellarische Aufstellungen sämtlicher Indikatoren, während andere ergänzend oder teilweise ausschließlich Diagramme und grafische Vergleichsdarstellungen beinhalten. 


\section{Datenextraktion aus PDF-Dokumenten}

\subsection{Das Portable Document Format (PDF)}

Das \emph{Portable Document Format (PDF)} ist eines der beliebtesten elektronischen Dokumentenformate.  
Das PDF-Format ist primär ein \emph{layoutbasiertes Format}. 
Es wurde entwickelt, um das Erscheinungsbild der Originaldokumente plattform- und anwendungsübergreifend zu bewahren. \cite{Lovegrove1995}
Das Format beschreibt Objekte auf einer niedrigen Strukturebene und legt die \emph{Positionen und Schriftarten der einzelnen Zeichen} fest, aus
denen der sichtbare Text zusammengesetzt ist. 
Zu den beschriebenen Objekten gehören:
\begin{itemize}
    \item Gruppen von Zeichen (Textobjekte)
    \item Linien, Kurven und Bilder
    \item Stilattribute wie Schriftart, Farbe, Strichführung, Füllung und geometrische Formen.
\end{itemize}
\cite{Bast2017}

Obwohl PDF die visuelle Darstellung eines Dokuments zuverlässig bewahrt, fehlt den meisten Dateien eine explizite logische Struktur auf höherer Ebene. 
Die folgenden semantischen Einheiten sind im Format \emph{nicht direkt enthalten} und nur durch die oben genannte niedrige Strukturebene zusammengesetzt:
\begin{itemize}
    \item logische Komponenten wie Wörter, Textzeilen, Absätze, Tabellen oder Abbildungen \cite{Chao2004}
    \item Informationen über die \emph{semantischen Rollen} des Textes (z.\,B.\ Haupttext, Fußnote oder Bildunterschrift), \cite{Bast2017}
    \item eine eindeutige Lese- und Wortreihenfolge, insbesondere bei mehrspaltigen Layouts oder eingebetteten Elementen. \cite{Bast2017}
\end{itemize}

Hinzuzufügen ist, dass PDF-Dokumente mit semantischen Informationen \emph{getaggt} werden können. In der Praxis sind diese zusätzlichen Informationen selten gegeben.
Die für diese Arbeit relevanten PEP-Ecopassport-PDFs sind alle nicht getaggt. \cite{Bast2017}

Das Fehlen dieser semantischen Informationen erschwert die Wiederverwendung, Bearbeitung oder Modifikation des Layouts und Inhalts erheblich. \cite{Corrêa2017}
Die automatische Extraktion dieser Metadaten und Textinhalte ist daher eine zentrale, aber fehleranfällige Aufgabe, da es keine 
allgemein verbindlichen Standards für die Strukturierung solcher Informationen in PDF-Dokumenten gibt. \cite{Lipinski2013}





\subsection{Herausforerungen bei der automatisierten Extraktion}

Die Rekonstruktion des Textflusses und der semantischen Einheiten aus den Positionen einzelner Zeichen ist komplex.

\subparagraph{1. Wortidentifikation}
Die korrekte Bestimmung von Wortgrenzen ist nicht trivial:
\begin{itemize}
    \item \emph{Abstände:} Die Abstände zwischen Zeichen können innerhalb einer Zeile variieren, sodass keine feste Regel existiert, 
    um Wortgrenzen ausschließlich anhand der Zeichenpositionen zu bestimmen. \cite{Bast2017}
    \item \emph{Silbentrennung:} In mehrspaltigen Layouts getrennte Wörter müssen korrekt wieder zusammengeführt werden. \cite{Bast2017}
    \item \emph{Ligaturen:} Zeichenkombinationen wie „fl“ oder „fi“ werden im PDF oft als einzelnes Zeichen gespeichert und 
    müssen beim Extrahieren in mehrere Zeichen aufgelöst werden. \cite{Lipinski2013}
    \item \emph{Diakritische Zeichen:} Buchstaben mit Diakritika (z.\,B.\ à, ã) können als zwei separate Zeichen gespeichert sein
    und müssen beim Parsing zu einem Zeichen zusammengeführt werden. \cite{Bast2017}
\end{itemize}

\subparagraph{2. Lesereihenfolge (Reading Order)}
Die korrekte Lesereihenfolge ist entscheidend für die Verständlichkeit des Textes und der weiterführenden Interpretation.  \cite{Bast2017}
In mehrspaltigen Layouts sind Textzeilen im PDF häufig in einer verschränkten Reihenfolge gespeichert. 
Ohne Korrekturmechanismen führt dies zu unleserlichem, inhaltlich falsch zusammengesetztem Text.\cite{Lovegrove1995}

\subparagraph{3. Absatzgrenzen (Paragraph Boundaries)}
Die Erkennung von Absatzanfängen und -enden ist besonders schwierig:
\begin{itemize}
    \item \emph{Unterbrechungen:} Text, der zu einem Absatz gehört, kann durch Formeln, Tabellen oder Abbildungen unterbrochen und später auf derselben Seite 
    fortgesetzt werden.
    \item \emph{Seitenumbrüche:} Absätze können am Seiten- oder Spaltenende abgeschnitten und auf der folgenden Seite fortgeführt werden, ohne dass dies
    im PDF strukturell kenntlich gemacht wird.
\end{itemize}
\cite{Bast2017}

\subparagraph{4. Technologische und Layout-Herausforderungen}
\begin{itemize}
    \item \emph{Überlagerungen (Overlays):} 
    In grafisch komplexen Dokumenten können Text- und Bildelemente überlappen, etwa wenn Beschriftungen in Abbildungen eingebettet sind. 
    Dies erschwert die korrekte Segmentierung. \cite{Chao2004}
    \item \emph{Segmentierungsfehler:} 
    Bei Tabellen, Karten oder Diagrammen kann Text aus unterschiedlichen logischen Einheiten fälschlicherweise in dieselbe Gruppe aggregiert werden. \cite{Chao2004}
    \item \emph{Type-3-Fonts:} 
    Manche Zeichen (insbesondere Ligaturen und Sonderzeichen) werden im PDF nicht als Textobjekte, sondern als Vektorgrafiken gespeichert. 
    Solche Elemente sind mit herkömmlicher Textextraktion nicht identifizierbar und erfordern erweiterte, teils OCR- oder ML-basierte Verfahren. \cite{Bast2017}
\end{itemize}


\subsection{Extraktionsansätze}
Da diese Probleme weit verbreitet und bekannt sind, ergeben sich mehrere Extraktionsansätze, die das Ziel PDF-Dateien in ein strukturierts Format zu bringen,
um sie anschließend weiter zu analysieren. 

\paragraph{Klassische Verfahren (regelbasierte Parser)}
Ein Beispiel für ein klassisches, regelbasiertes Extraktionstool ist die Open-Source-Bibliothek \emph{pdfplumber}. 
Sie ist vollständig in Python implementiert und baut auf der weit verbreiteten Bibliothek \emph{pdfminer.six} auf. 
Das Werkzeug wurde speziell für die Text- und Tabellenextraktion aus PDF-Dokumenten entwickelt und gilt als eine 
der benutzerfreundlichsten Lösungen im Python-Ökosystem. \cite{Adhikari2025}

\emph{pdfplumber} konvertiert beim Einlesen einer PDF-Datei deren Inhalt in ein analysierbares Python-Objekt, das sämtliche Seiteninformationen
wie Text, Linien, Rechtecke und Bilder enthält. Jede Seite wird dabei als Sammlung geometrischer Objekte behandelt, deren Koordinaten und Stilattribute 
(z.\,B.\ Schriftart, Farbe, Position) präzise erfasst sind. \cite{Yang2025} 
Diese Informationen werden über Objektlisten verfügbar gemacht. 
Für die Erkennung und Extraktion von Tabellen nutzt das Tool einfache visuelle Heuristiken: Standardmäßig werden horizontale und vertikale Linien
einer Seite als potenzielle Zellgrenzen interpretiert.  \cite{Yang2025}
Über Parameter wie \texttt{table\_settings} oder \texttt{snap\_tolerance} lässt sich die Erkennung an unterschiedliche Layouts anpassen. 
So können beispielsweise die Toleranzen für Linienabstände verändert werden, um verschobene Spalten oder leere Zellen zu korrigieren. \cite{Yang2025}

Der regelbasierte Ansatz von \emph{pdfplumber} führt bei klar strukturierten, editierbaren PDF-Dokumenten zu guten Ergebnissen. 
In Studien zur Leistungsbewertung von Extraktionstools erzielte es in Domänen wie juristischen oder technischen
Dokumenten hohe F1-Scores (beispielsweise 0{,}98 im Bereich \emph{Law}). \cite{Adhikari2025}

Die Grenzen des Werkzeugs zeigen sich jedoch vor allem bei komplexen oder unregelmäßig formatierten PDF-Dateien. 
Insbesondere wissenschaftliche Dokumente mit mathematischen Ausdrücken, eingebetteten Formeln oder verschachtelten Tabellen führen zu deutlichen Leistungseinbußen. 
In der Kategorie \emph{Scientific} sank der F1-Score auf 0{,}76, was vor allem auf unvollständige Tabellenerkennung und fehlerhafte Segmentierung zurückzuführen ist. 
Auch bei Patenten oder Dokumenten mit grafischen Strukturen (z.\,B.\ chemischen Formeln oder Bauzeichnungen) stößt der regelbasierte Ansatz an seine Grenzen. \cite{Adhikari2025}

Aus sicht der zuvor beschriebenen Herausforderungen adressiert \emph{pdfplumber} also Probleme auf der Ebene der Zeichen- und Worterkennung weitgehend.
Die Wiederherstellung der Lesereihenfolge erfolgt allerdings rein geometrisch, ohne semantisches Verständnis, wodurch Textpassagen aus mehrspaltigen Layouts7
häufig in falscher Reihenfolge extrahiert werden. 
Absatzgrenzen, semantische Rollen (z.\,B.\ Überschriften, Fließtext, Bildunterschriften) und komplexe Tabellenstrukturen erkennt \emph{pdfplumber} nicht zuverlässig.

Damit steht \emph{pdfplumber} exemplarisch für klassische Extraktionstools, die ohne maschinelles Lernen oder tiefere Dokumentenverständnis-Modelle
arbeiten und deshalb bei komplexen, visuell strukturierten Dokumenten wie den PEP-Ecopassports an ihre methodischen Grenzen stoßen.



\paragraph{Erweiterte Verfahren (z.B. Docling)}
Ein modernes, KI-gestütztes Gegenbeispiel zu klassischen, regelbasierten Extraktionstools ist das Open-Source-Toolkit \emph{Docling}. 
Es wurde mit dem Ziel entwickelt, PDF-Dokumente und andere Formate in eine maschinell verarbeitbare, reich strukturierte Repräsentation zu überführen. 
Im Gegensatz zu Werkzeugen wie \emph{pdfplumber}, die auf geometrischen Heuristiken basieren, kombiniert \emph{Docling} klassische Parsing-Verfahren mit 
tiefen neuronalen Modellen für Layout- und Strukturerkennung. \cite{Auer2024}
Das Toolkit ist vollständig in Python implementiert, modular aufgebaut und kann lokal ohne Cloud-Anbindung ausgeführt werden, 
was es insbesondere für den Einsatz in sensiblen Datenumgebungen geeignet macht. \cite{Auer2025}

Technisch basiert \emph{Docling} auf einer linearen Verarbeitungs-Pipeline, die mehrere spezialisierte Komponenten kombiniert. 
Nach dem initialen Parsen durch ein PDF-Backend (z.\,B.\ \emph{qpdf} oder \emph{pypdfium}) werden für jede Seite Bitmap-Abbilder erzeugt, 
auf denen KI-Modelle für Layout- und Strukturerkennung ausgeführt werden. \cite{Auer2024}
Das zugrunde liegende Layout-Analysemodell \emph{DocLayNet} identifiziert auf Basis eines trainierten Objektdetektors verschiedene 
Seitenelemente und deren Begrenzungsrahmen – etwa Absätze, Überschriften, Listen, Abbildungen oder Tabellen.  \cite{Auer2024}
Diese visuellen Einheiten werden mit den extrahierten Text-Tokens verknüpft und zu konsistenten Dokumentstrukturen zusammengeführt. 
Für erkannte Tabellenobjekte kommt anschließend das Vision-Transformer-Modell \emph{TableFormer} zum Einsatz, das die logische Zeilen- und Spaltenstruktur 
einer Tabelle rekonstruiert und die Zellen semantisch klassifiziert (z.\,B.\ Kopf- oder Körperzellen).  
Für gescannte oder bildbasierte Dokumente steht optional eine OCR-Komponente auf Basis von \emph{EasyOCR} zur Verfügung. \cite{Auer2025}

Das Herzstück von \emph{Docling} bildet das Datenmodell \emph{DoclingDocument}, eine vereinheitlichte interne Repräsentation, die sämtliche 
Inhalte eines Dokuments (Text, Tabellen, Bilder, Layoutinformationen, Hierarchieebenen und Metadaten) in strukturierter Form abbildet. 
Nach Abschluss aller Erkennungsschritte werden die Ergebnisse zu einem vollständigen \emph{DoclingDocument} zusammengeführt und können in 
verschiedenen Formaten exportiert werden, darunter JSON, Markdown und HTML. 
Im Post-Processing ergänzt ein sprachsensitives Modell weitere Merkmale wie die Korrektur der Lesereihenfolge, die automatische Spracherkennung und die 
Extraktion zentraler Metadaten (Titel, Autoren, Referenzen).
\cite{Auer2024}

Durch diese Architektur adressiert \emph{Docling} mehrere der in Abschnitt~2.3 beschriebenen Extraktionsprobleme, die klassische Tools nur unzureichend lösen können. 
Es rekonstruiert eine konsistente Lesereihenfolge auch bei mehrspaltigen Layouts, erkennt logische Dokumentstrukturen und kann Tabellen semantisch 
interpretieren, anstatt sie rein geometrisch zu segmentieren.  \cite{Auer2025}
Darüber hinaus bietet es eine robuste Metadaten- und Inhaltsklassifizierung, die zwischen Fließtext, Überschriften, Listen, 
Bildunterschriften und Formeln unterscheidet. 
Die erzeugten Ausgaben sind reich strukturiert und dienen als Grundlage für weiterführende Analysen oder Datenpipelines, etwa zur Wissensextraktion, 
semantischen Suche oder automatisierten Inhaltsklassifikation. \cite{Auer2024}

Im Vergleich zu klassischen, regelbasierten Parsern wie \emph{pdfplumber} kombiniert \emph{Docling} geometrische und visuelle Merkmale mit semantischem Verständnis. 
Dadurch ermöglicht es eine qualitativ hochwertige, KI-gestützte Dokumentenkonvertierung, die sowohl schnelle 
als auch stabile Ergebnisse liefert und für komplexe Dokumente wie PEP-Ecopassports einen erheblichen Qualitätsgewinn in der Extraktion bietet. \cite{Auer2024}

Tabelle~\ref{tab:extraktion_vergleich} fasst die Extraktionsfähigkeiten der beiden Ansätze zusammen. 

\begin{table}[h!]
\centering
\caption{Vergleich der Extraktionsfähigkeiten von \textit{pdfplumber} und \textit{Docling}}
\label{tab:extraktion_vergleich}
\begin{tabular}{p{5cm}p{3cm}p{3cm}}
\hline
\textbf{Aspekt} & \textbf{pdfplumber} & \textbf{Docling} \\
\hline
Zeichen- und Worterkennung & gut & gut \\
Lesereihenfolge (Reading Order) & schlecht & gut \\
Absatz- und Textstruktur & teilweise & gut \\
Tabellenerkennung & teilweise & gut \\
Grafiken und eingebettete Objekte & nicht & teilweise \\
Metadatenextraktion & nicht & gut \\
Nicht-textuelle Inhalte (OCR) & nicht & gut \\
Komplexe Layouts (mehrspaltig, technisch) & schlecht & gut \\
Semantische Rollen (Caption, Footnote, etc.) & nicht & gut \\
\hline
\end{tabular}
\end{table}







\section{PDF-Datenextraktion (?)}
Theoretische Grundlagen PDFs und Daten (Aufbau Textlayer usw), gpt Modell für extraktion (Vor-/Nachteile)

\section{Statistische Methoden}
Welche statistischen Methoden werden eingesetzt und worauf basieren sie?

