\section{Datenextraktion aus PDF-Dokumenten}

\subsection{Das Portable Document Format (PDF)}

Das \emph{Portable Document Format (PDF)} ist eines der beliebtesten elektronischen Dokumentenformate.  
Das PDF-Format ist primär ein \emph{layoutbasiertes Format}. 
Es wurde entwickelt, um das Erscheinungsbild der Originaldokumente plattform- und anwendungsübergreifend zu bewahren. \cite{Lovegrove1995}
Das Format beschreibt Objekte auf einer niedrigen Strukturebene und legt die \emph{Positionen und Schriftarten der einzelnen Zeichen} fest, aus
denen der sichtbare Text zusammengesetzt ist. 
Zu den beschriebenen Objekten gehören:
\begin{itemize}
    \item Gruppen von Zeichen (Textobjekte)
    \item Linien, Kurven und Bilder
    \item Stilattribute wie Schriftart, Farbe, Strichführung, Füllung und geometrische Formen.
\end{itemize}
\cite{Bast2017}

Obwohl PDF die visuelle Darstellung eines Dokuments zuverlässig bewahrt, fehlt den meisten Dateien eine explizite logische Struktur auf höherer Ebene. 
Die folgenden semantischen Einheiten sind im Format \emph{nicht direkt enthalten} und nur durch die oben genannte niedrige Strukturebene zusammengesetzt:
\begin{itemize}
    \item logische Komponenten wie Wörter, Textzeilen, Absätze, Tabellen oder Abbildungen \cite{Chao2004}
    \item Informationen über die \emph{semantischen Rollen} des Textes (z.\,B.\ Haupttext, Fußnote oder Bildunterschrift), \cite{Bast2017}
    \item eine eindeutige Lese- und Wortreihenfolge, insbesondere bei mehrspaltigen Layouts oder eingebetteten Elementen. \cite{Bast2017}
\end{itemize}

Hinzuzufügen ist, dass PDF-Dokumente mit semantischen Informationen \emph{getaggt} werden können. In der Praxis sind diese zusätzlichen Informationen selten gegeben.
Die für diese Arbeit relevanten PEP-Ecopassport-PDFs sind alle nicht getaggt. \cite{Bast2017}

Das Fehlen dieser semantischen Informationen erschwert die Wiederverwendung, Bearbeitung oder Modifikation des Layouts und Inhalts erheblich. \cite{Corrêa2017}
Die automatische Extraktion dieser Metadaten und Textinhalte ist daher eine zentrale, aber fehleranfällige Aufgabe, da es keine 
allgemein verbindlichen Standards für die Strukturierung solcher Informationen in PDF-Dokumenten gibt. \cite{Lipinski2013}





\subsection{Hauptschwierigkeiten bei der automatisierten Extraktion}

Die Rekonstruktion des Textflusses und der semantischen Einheiten aus den Positionen einzelner Zeichen ist komplex.

\subparagraph{1. Wortidentifikation}
Die korrekte Bestimmung von Wortgrenzen ist nicht trivial:
\begin{itemize}
    \item \emph{Abstände:} Die Abstände zwischen Zeichen können innerhalb einer Zeile variieren, sodass keine feste Regel existiert, 
    um Wortgrenzen ausschließlich anhand der Zeichenpositionen zu bestimmen. \cite{Bast2017}
    \item \emph{Silbentrennung:} In mehrspaltigen Layouts getrennte Wörter müssen korrekt wieder zusammengeführt werden. \cite{Bast2017}
    \item \emph{Ligaturen:} Zeichenkombinationen wie „fl“ oder „fi“ werden im PDF oft als einzelnes Zeichen gespeichert und 
    müssen beim Extrahieren in mehrere Zeichen aufgelöst werden. \cite{Lipinski2013}
    \item \emph{Diakritische Zeichen:} Buchstaben mit Diakritika (z.\,B.\ à, ã) können als zwei separate Zeichen gespeichert sein
    und müssen beim Parsing zu einem Zeichen zusammengeführt werden. \cite{Bast2017}
\end{itemize}

\subparagraph{2. Lesereihenfolge (Reading Order)}
Die korrekte Lesereihenfolge ist entscheidend für die Verständlichkeit des Textes und der weiterführenden Interpretation.  \cite{Bast2017}
In mehrspaltigen Layouts sind Textzeilen im PDF häufig in einer verschränkten Reihenfolge gespeichert. 
Ohne Korrekturmechanismen führt dies zu unleserlichem, inhaltlich falsch zusammengesetztem Text.\cite{Lovegrove1995}

\subparagraph{3. Absatzgrenzen (Paragraph Boundaries)}
Die Erkennung von Absatzanfängen und -enden ist besonders schwierig:
\begin{itemize}
    \item \emph{Unterbrechungen:} Text, der zu einem Absatz gehört, kann durch Formeln, Tabellen oder Abbildungen unterbrochen und später auf derselben Seite 
    fortgesetzt werden.
    \item \emph{Seitenumbrüche:} Absätze können am Seiten- oder Spaltenende abgeschnitten und auf der folgenden Seite fortgeführt werden, ohne dass dies
    im PDF strukturell kenntlich gemacht wird.
\end{itemize}
\cite{Bast2017}

\subparagraph{4. Technologische und Layout-Herausforderungen}
\begin{itemize}
    \item \emph{Überlagerungen (Overlays):} 
    In grafisch komplexen Dokumenten können Text- und Bildelemente überlappen, etwa wenn Beschriftungen in Abbildungen eingebettet sind. 
    Dies erschwert die korrekte Segmentierung. \cite{Chao2004}
    \item \emph{Segmentierungsfehler:} 
    Bei Tabellen, Karten oder Diagrammen kann Text aus unterschiedlichen logischen Einheiten fälschlicherweise in dieselbe Gruppe aggregiert werden. \cite{Chao2004}
    \item \emph{Type-3-Fonts:} 
    Manche Zeichen (insbesondere Ligaturen und Sonderzeichen) werden im PDF nicht als Textobjekte, sondern als Vektorgrafiken gespeichert. 
    Solche Elemente sind mit herkömmlicher Textextraktion nicht identifizierbar und erfordern erweiterte, teils OCR- oder ML-basierte Verfahren. \cite{Bast2017}
\end{itemize}


