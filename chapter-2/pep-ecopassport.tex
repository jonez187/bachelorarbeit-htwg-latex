\section{PEP-Ecopassport}
Was ist PEP-Ecopassport, was steht drin, was ist interessant für mich

\subsection{PEP-Standard}

Der \emph{PEP~Ecopassport\textsuperscript{\textregistered}} ist ein international anerkanntes Programm zur Erstellung standardisierter Umweltproduktdeklarationen für elektrische, 
elektronische sowie Heizungs-, Lüftungs-, Klima- und Kälteprodukte (HVAC). Träger des Programms ist die \emph{P.E.P.~Association}, eine gemeinnützige
Organisation, deren Ziel es ist, ein gemeinsames und verlässliches Referenzsystem für Umweltinformationen dieser Produktkategorien bereitzustellen. 
Das Programm versteht sich als Branchenspezialisierung innerhalb des Rahmens der \emph{Environmental Product Declarations (EPD)} gemäß ISO~14025 und der 
Lebenszyklusnormen nach ISO~14040, und basiert somit auf international festgelegten Normen. \cite{PEP}

Ein \emph{PEP~Ecopassport} ist somit eine \emph{Typ~III-Umweltdeklaration} im Sinne der ISO~14025. Diese Deklarationen basieren auf quantitativen Ergebnissen einer
Lebenszyklusanalyse (\emph{Life Cycle Assessment, LCA}) und dienen der vergleichenden Bewertung von Produkten mit identischer Funktion. 
Die Datenerhebung und Berechnung erfolgt nach vordefinierten Parametern, die in sogenannten \emph{Product~Category~Rules~(PCR)} und bei Bedarf in 
\emph{Product~Specific~Rules~(PSR)} festgelegt sind. Jede PEP-Deklaration unterliegt einer unabhängigen Überprüfung der angewandten Methodik und der zugrunde liegenden LCA-Daten. \cite{PEP}

Das Programm zielt auf Transparenz und Vergleichbarkeit ab. Hersteller erhalten ein einheitliches Verfahren, um ökologische Leistungskennwerte ihrer 
Produkte objektiv und nachvollziehbar zu kommunizieren. Für Anwender, Beschaffer und Energieberater stellen die PEP-Daten eine 
verlässliche Grundlage für ökologische Bewertungen und Beschaffungsentscheidungen dar.

Die Teilnahme am PEP-Programm ist freiwillig, gewinnt jedoch in der Praxis an Bedeutung, da Umweltproduktdeklarationen zunehmend als Nachweis 
oder Auswahlkriterium in Ausschreibungen und Produktbewertungen herangezogen werden. Eine gesetzliche Verpflichtung zur Erstellung besteht bislang nur in Einzelfällen, 
beispielsweise in Frankreich, wenn ein Hersteller aktiv mit Umweltvorteilen wirbt.

Das PEP-Programm unterscheidet sich klar von unternehmensbezogenen Treibhausgas-Bilanzierungen: 
Es erfasst ausschließlich produktspezifische Umweltwirkungen entlang des Lebenszyklus und folgt dabei den methodischen Vorgaben der ISO~14040-Reihe. 
Für umfassende \emph{GHG-Assessments} auf Organisationsebene sind PEP-Daten daher nicht geeignet.


\subsection{Aufbau typischer PEP-Dokumente}

Ein vollständiges PEP umfasst typischerweise etwa zehn Seiten und gliedert sich in mehrere inhaltlich definierte Abschnitte.

\paragraph{Titel- und Metadatenblatt}
Das Deckblatt enthält grundlegende Angaben zum Produkt (Name, Version, Sprache, Hersteller), zum Veröffentlichungs- und Revisionsdatum
sowie zum Status der Erklärung (z.\,B. \emph{in~review} oder \emph{verified}). Darüber hinaus sind Kontaktinformationen, Firmenadresse und 
Registrierungsnummer enthalten. 

\paragraph{Allgemeine Produktinformationen}
Dieser Abschnitt beschreibt die funktionale Einheit (\emph{functional~unit}) und die Referenzlebensdauer, hier meist 10~bis~20~Jahre. 
Weiterhin werden die Produktfunktion, Anwendungsbereiche und gegebenenfalls weitere Varianten aufgeführt. 

\paragraph{Materialzusammensetzung}
Die Zusammensetzung des Produkts wird tabellarisch nach Hauptgruppen ausgewiesen, z.\,B.\ Kunststoffe, Metalle und weitere Materialien 
(Elektronik, Sonstiges). Im Beispiel des ABB~EQ~Meter entfallen 28\,\% auf Kunststoffe, 53\,\% auf Metalle und 19\,\% auf weitere Komponenten. 
Diese Angaben ermöglichen eine spätere Aggregation der Stoffanteile in harmonisierten Datenstrukturen.

\paragraph{Szenarien und Lebenszyklusphasen}
PEP-Dokumente sind entlang der Phasen des Produktlebenszyklus strukturiert, die den Vorgaben der EN~15804 entsprechen:
\begin{itemize}
    \item \textbf{Herstellung (A1--A3)}: Produktion und Vormaterialien, modelliert mit landesspezifischem Strommix (z.\,B.\ italienischer Grid~Mix).
    \item \textbf{Distribution (A4)}: Transport vom Werk zum Markt; häufig standardisierte Annahmen (z.\,B.\ 1\,000~km Schiff, 3\,300~km Lkw).
    \item \textbf{Installation (A5)}: Montage, meist nur Verpackungsabfälle berücksichtigt.
    \item \textbf{Nutzungsphase (B)}: Betrieb des Geräts mit angegebenem Energieverbrauch, z.\,B.\ 126~kWh über 20~Jahre, basierend auf europäischem Netzstrommix.
    \item \textbf{End-of-Life (C1--C4)}: Entsorgungsszenario gemäß PCR-Vorgaben (Recycling-, Deponie-, Transportanteile).
    \item \textbf{Optionale Phase (D)}: Rückgewinnung und Wiederverwendung außerhalb des Systemgrenzenmodells.
\end{itemize}
In der weiteren Datenaufbereitung werden diese Phasen zu den Kategorien \emph{manufacturing}, \emph{distribution}, \emph{installation}, 
\emph{use} und \emph{end\_of\_life} zusammengefasst.

\paragraph{Umweltindikatoren}
Die Umweltwirkungen werden für jede Lebenszyklusphase sowie als Gesamtwert angegeben.  
Die für diese Arbeit relevanten Indikatoren sind in der Tabelle~\ref{tab:indicators_description} aufgeführt.

\begin{table}[h!]
\centering
\caption{Umweltindikatoren}
\label{tab:indicators_description}
\resizebox{\textwidth}{!}{
\begin{tabular}{ll}
\hline
\textbf{Indikator} & \textbf{Beschreibung} \\
\hline
Acidification & Versauerung von Böden und Gewässern durch säurebildende Emissionen \\
Climate Change (Fossil) & Treibhauspotenzial durch fossile CO$_2$-Emissionen \\
Climate Change (Land Use and Land Use Change) & Treibhauspotenzial infolge von Landnutzungsänderungen (LULUC) \\
Climate Change (Total) & Gesamtes Treibhauspotenzial aus allen Quellen \\
Eutrophication (Freshwater) & Nährstoffanreicherung in Binnengewässern \\
Eutrophication (Marine) & Nährstoffanreicherung in marinen Ökosystemen \\
Eutrophication (Terrestrial) & Nährstoffanreicherung in terrestrischen Ökosystemen \\
Hazardous Waste Disposed & Entsorgung gefährlicher Abfälle \\
Ozone Depletion & Abbau der stratosphärischen Ozonschicht durch FCKW-Emissionen \\
Photochemical Ozone Formation (Human Health) & Bildung von bodennahem Ozon (Sommersmog) \\
Radioactive Waste Disposed & Entsorgung radioaktiver Abfälle \\
Resource Use (Fossils) & Nutzung fossiler Energieressourcen \\
Resource Use (Minerals and Metals) & Verbrauch abiotischer Ressourcen (Metalle und Mineralien) \\
Water Use & Entnahme und Verbrauch von Frischwasser \\
\hline
\end{tabular}
}
\end{table}




\paragraph{Verifikations- und Anhangsangaben}
Im abschließenden Teil werden die angewendeten Regelwerke und Datenquellen genannt, z.\,B.\ \emph{PCR-ed3-EN-2015\_04\_02} und \emph{PSR-0005-ed2-EN-2016\_03\_29}, 
die eingesetzte Software (z.\,B.\ SimaPro~9.3 mit Ecoinvent~3.8) sowie die Verifizierungsstelle und deren Akkreditierungsnummer. 
Darüber hinaus enthält dieser Abschnitt Angaben zum \emph{Materialaufbau} und zum verwendeten \emph{Energie\-modell}. 
Die Materialzusammensetzung wird in der Regel als prozentuale Massenanteile nach Hauptgruppen (Kunststoffe, Metalle, Elektronik, Sonstiges) dargestellt,
teils in Tabellenform, teils grafisch als Kreisdiagramm. 
Das Energiemodell beschreibt die angenommenen Strommixe und Spannungsniveaus je Lebenszyklusphase, beispielsweise den nationalen Grid~Mix 
für die Herstellung und den europäischen Durchschnittsmix für die Nutzungsphase. 

Obwohl der inhaltliche Mindestumfang und die zu berichtenden Umweltindikatoren durch die zugrundeliegenden ISO- und PCR-Vorgaben festgelegt sind, besteht keine feste formale Struktur. 
Das Layout, die grafische Aufbereitung und die Anordnung der Tabellen können je nach Hersteller, Software und Version variieren. 
So enthalten einige PEPs tabellarische Aufstellungen sämtlicher Indikatoren, während andere ergänzend oder teilweise ausschließlich Diagramme und grafische Vergleichsdarstellungen beinhalten. 

\paragraph{Einordnung für die Datenanalyse}
Die standardisierte, aber formal heterogene Gliederung der PEPs ermöglicht eine eindeutige Zuordnung der Informationen zu den Zieldatenfeldern der harmonisierten JSON-Datenbank. 
\emph{functional\_unit}, \emph{lifetime} und \emph{manufacturing\_country} bilden die Metadatenebene, 
\emph{material\_composition[]} enthält die Stoffanteile, 
\emph{indicators\{\}} die Umweltindikatoren pro Phase, 
und \emph{energy\_model[]} die modellierten Energiemixe je Lebenszyklusphase. 
Damit stellen PEP-PDFs eine normbasierte, inhaltlich konsistente, aber formal variable Datenquelle für die automatisierte Extraktion und statistische Auswertung dar.

