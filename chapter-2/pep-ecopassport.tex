\section{PEP-Ecopassport}
Die einzige Datenquelle dieser Arbeit bilden die 
\emph{PEP~Ecopassports\textsuperscript{\textregistered}}, 
welche ausschließlich im PDF-Format vorliegen. 
In diesem Kapitel werden die Standards, Inhalt und Struktur dieser Dokumente beschrieben,
um die spätere Datenerhebung und -verarbeitung nachvollziehbar zu machen.


\subsection{PEP-Standard}

Der \emph{PEP~Ecopassport} ist ein in\-ter\-national an\-er\-kanntes Programm 
für die Er\-stel\-lung stand\-ardi\-sier\-ter 
Um\-welt\-produkt\-de\-kla\-rationen für elek\-tri\-sche, 
elek\-tro\-nische sowie Heizungs-, Lüftungs-, Klima- und Kälteprodukte (HVAC). 
Träger des Programms ist die \emph{P.E.P.~Association}, eine gemeinnützige
Organisation, deren Ziel es ist, ein gemeinsames und verlässliches Referenzsystem für 
Umweltinformationen dieser Produktkategorien bereitzustellen. 
Das Programm versteht sich als Branchenspezialisierung innerhalb des Rahmens der 
\emph{Environmental Product Declarations (EPD)} gemäß ISO~14025 und der 
Lebenszyklusnormen nach ISO~14040, und basiert somit auf international festgelegten Normen \cite{PEP}. 

Die PEP-Deklarationen basieren auf quantitativen Ergebnissen einer
Lebenszyklusanalyse (\emph{Life Cycle Assessment, LCA}) und dienen der vergleichenden
Bewertung von Produkten mit identischer Funktion. 
Die Datenerhebung und Berechnung erfolgt nach vordefinierten Parametern,
die in sogenannten \emph{Product~Category~Rules~(PCR)} und bei Bedarf in 
\emph{Product~Specific~Rules~(PSR)} festgelegt sind. 
Jede PEP-Deklaration unterliegt einer unabhängigen Überprüfung der angewandten 
Methodik und der zugrunde liegenden LCA-Daten \cite{Hassanzadeh2013}. 

Die Teilnahme am PEP-Programm ist freiwillig, 
gewinnt jedoch in der Praxis an Bedeutung, da Umweltproduktdeklarationen zunehmend als Nachweis 
oder Auswahlkriterium in Ausschreibungen und Produktbewertungen herangezogen werden. 
Eine gesetzliche Verpflichtung zur Erstellung besteht bislang nur in Einzelfällen, 
beispielsweise in Frankreich, wenn ein Hersteller aktiv mit Umweltvorteilen wirbt \cite{PEP}. 

Das PEP-Programm unter\-schei\-det sich nach Angaben der \emph{P.E.P.~Association} klar von 
unter\-neh\-mens\-be\-zo\-ge\-nen Treib\-haus\-gas-Bilanzierungen. 
Es erfasst ausschließlich produktspezifische Umweltwirkungen entlang des Lebenszyklus und folgt 
dabei den methodischen Vorgaben der ISO~14040-Reihe. 
Für eine umfassende Treibhausgasbilanz auf Organisationsebene sind PEP-Daten daher nicht geeignet \cite{PEP}. 

\subsection{Aufbau typischer PEP-Dokumente}

Ein vollständiges PEP umfasst typischerweise etwa zehn Seiten und gliedert sich in mehrere inhaltlich definierte Abschnitte.

\paragraph{Titel- und Metadatenblatt}
Das Deckblatt enthält grundlegende Angaben zum Produkt (Name, Version, Sprache, Hersteller), 
zum Veröffentlichungs- und Revisionsdatum.
Darüber hinaus sind Kontaktinformationen, Firmenadresse und 
Registrierungsnummer enthalten. 

\paragraph{Allgemeine Produktinformationen}
Dieser Abschnitt beschreibt die funktionale Einheit (\emph{functional~unit}),
in welcher auch der Stromverbrauch dargestellt ist. 
Weiterhin werden Referenzlebensdauer, hier meist 10~bis~20~Jahre, die Produktfunktion,
Anwendungsbereiche und gegebenenfalls weitere Varianten aufgeführt. 

\paragraph{Materialzusammensetzung}
Die Zusammensetzung des Produkts wird teils in Tabellenform, teils grafisch als Kreisdiagramm 
nach Hauptgruppen ausgewiesen, 
z.\,B.\ Kunststoffe, Metalle und weitere Materialien 
(Papier/Karton, Elektronik, Sonstiges). 
In diesem Abschnitt ist meist auch das Gesamtgewicht des Produktes zu finden.

\paragraph{Szenarien und Lebenszyklusphasen}
PEP-Dokumente sind entlang der Phasen des Produktlebenszyklus strukturiert, die den Vorgaben der EN~15804 entsprechen:
\begin{itemize}
    \item \textbf{Herstellung (A1--A3)}: Produktion und Vormaterialien
    \item \textbf{Distribution (A4)}: Transport vom Werk zum Markt, häufig standardisierte Annahmen (z.\,B.\ 1\,000~km Schiff, 3\,300~km Lkw)
    \item \textbf{Installation (A5)}: Montage, meist nur Verpackungsabfälle berücksichtigt
    \item \textbf{Nutzungsphase (B)}: Betrieb des Geräts mit angegebenem Energieverbrauch, z.\,B.\ 126~kWh über 20~Jahre.
    \item \textbf{End-of-Life (C1--C4)}: Entsorgungsszenario gemäß PCR-Vorgaben (Recycling-, Deponie-, Transportanteile).
    \item \textbf{Optionale Phase (D)}: Rückgewinnung und Wiederverwendung außerhalb des Systemgrenzenmodells.
\end{itemize}
In der weiteren Datenaufbereitung werden diese Phasen zu den Kategorien \emph{manufacturing}, \emph{distribution}, \emph{installation}, 
\emph{use} und \emph{end\_of\_life} zusammengefasst.


\paragraph{Energiemodelle}
Zusätzlich werden die verwendeten Energiemodelle angegeben (z.\,B.\ \emph{France
Grid Mix}), welche die Herkunft und Zusammensetzung des im Lebenszyklus des Produkts
genutzten Stroms beschreiben.
Die Genauigkeit dieser Angaben variiert deutlich zwischen den Dokumenten.
In einigen Fällen ist jeder einzelnen Produktlebenszyklusphase ein spezifisches Land inklusive
des Jahres zugeordnet,
während andere PEPs für alle Phasen einen einheitlichen europäischen Strommix angeben.


\paragraph{Umweltindikatoren}
Die Umweltwirkungen werden für jede Lebenszyklusphase sowie als Gesamtwert angegeben.  
Die für diese Arbeit relevanten Indikatoren sind in der Tabelle~\ref{tab:indicators_description} aufgeführt.

\begin{table}[H]
\centering
\caption{Umweltindikatoren}
\label{tab:indicators_description}
\resizebox{\textwidth}{!}{
\begin{tabular}{ll}
\hline
\textbf{Indikator} & \textbf{Beschreibung} \\
\hline
Acidification & Versauerung von Böden und Gewässern durch säurebildende Emissionen \\
Climate Change (Total) & Gesamtes Treibhauspotenzial aus allen Quellen, CO$_2$ Äquivalente \\
Eutrophication (Freshwater) & Nährstoffanreicherung in Binnengewässern \\
Eutrophication (Marine) & Nährstoffanreicherung in marinen Ökosystemen \\
Eutrophication (Terrestrial) & Nährstoffanreicherung Böden \\
Hazardous Waste Disposed & Entsorgung gefährlicher Abfälle \\
Ozone Depletion & Abbau der stratosphärischen Ozonschicht durch FCKW-Emissionen \\
Photochemical Ozone Formation (Human Health) & Bildung von bodennahem Ozon (Sommersmog) \\
Radioactive Waste Disposed & Entsorgung radioaktiver Abfälle \\
Resource Use (Fossils) & Nutzung fossiler Energieressourcen \\
Resource Use (Minerals and Metals) & Verbrauch abiotischer Ressourcen (Metalle und Mineralien) \\
Water Use & Entnahme und Verbrauch von Frischwasser \\
\hline
\end{tabular}
}
\end{table}


\paragraph{Verifikations- und Anhangsangaben}
Im abschließenden Teil werden die angewendeten Regelwerke und Datenquellen genannt, 
z.\,B.\ \emph{PCR-ed3-EN-2015\_04\_02} und \emph{PSR-0005-ed2-EN-2016\_03\_29}, 
die eingesetzte Software (z.\,B.\ SimaPro~9.3 mit Ecoinvent~3.8) sowie die Verifizierungsstelle
und deren Akkreditierungsnummer. 
 

Obwohl der inhaltliche Mindestumfang und die zu berichtenden Umweltindikatoren durch die
zugrundeliegenden ISO- und PCR-Vorgaben festgelegt sind, besteht keine feste formale Struktur. 
Das Layout, die grafische Aufbereitung und die Anordnung der Tabellen können je nach Hersteller, 
Software und Version variieren. 
So enthalten einige PEPs tabellarische Aufstellungen sämtlicher Indikatoren, während andere ergänzend 
oder teilweise ausschließlich Diagramme und grafische Vergleichsdarstellungen beinhalten. 

Beispiele für registrierte PEPs, die für diese Arbeit relevant sind, sind zu finden in (TODO?)
https://register.pep-ecopassport.org/pep/consult/mbesqrsCBZbWbKJq6-kJ3oD8xj1w8Nr0yOA4adYakbU/mbesqrsCBZbWbKJq6-kJ3lQmBuGvAHsLUfQU9idjOpk
https://register.pep-ecopassport.org/pep/consult/mbesqrsCBZbWbKJq6-kJ3iYxqd7MZZUSeVvqE8K_428/mbesqrsCBZbWbKJq6-kJ3lQmBuGvAHsLUfQU9idjOpk
https://register.pep-ecopassport.org/pep/consult/mbesqrsCBZbWbKJq6-kJ3tllDPqEgPXR7-cNoARqDMg/mbesqrsCBZbWbKJq6-kJ3lQmBuGvAHsLUfQU9idjOpk