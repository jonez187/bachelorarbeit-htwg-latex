\section{Statistische Grundlagen}

Die in dieser Arbeit verwendeten statistischen Verfahren bilden 
die methodische Grundlage zur Analyse und Modellierung der aus PEP~Ecopassports extrahierten Daten. 
Dazu werden zunächst \emph{deskriptive und explorative} Verfahren eingesetzt, um Strukturen, 
Streuungen und Ausreißer in den Daten sichtbar zu machen. 
Die \emph{Hauptkomponentenanalyse} wird dafür verwendet die wichtigsten Merkmale zu identifizieren.
Darauf aufbauend wird die \emph{lineare Regression} als einfaches, interpretierbares Modell genutzt, 
um heuristische Beziehungen zwischen Einflussgrößen und den resultierenden Umweltindikatoren zu identifizieren. 
Diese Kombination ermöglicht eine robuste, nachvollziehbare und datengetriebene Einschätzung ökologischer Wirkzusammenhänge
im Datensatz.

\subsection{Deskriptive und explorative Statistik}
Die deskriptive und explorative Statistik bilden die Grundlage der quantitativen Datenanalyse. 
Beide dienen der Zusammenfassung, Beschreibung und Visualisierung von Datensätzen, 
um zentrale Merkmale einer Verteilung zu charakterisieren und potenzielle Muster oder Auffälligkeiten zu erkennen. \cite{Fisher2009}
Der Schwerpunkt liegt nicht auf Hypothesentests, sondern auf dem Verständnis der vorhandenen Daten. \cite{Dimić2019}
In dieser Arbeit werden die Verfahren auf aus PEP~Ecopassports extrahierte Kennwerte angewendet. 

\paragraph{Deskriptive Statistik}
Die deskriptive Statistik umfasst numerische und grafische Verfahren zur Beschreibung
(i) der \emph{zentralen Tendenz} und (ii) der \emph{Streuung} von Daten. \cite{Fisher2009}
Ziel ist die Verdichtung großer Datenmengen auf wenige aussagekräftige Kennzahlen. 
Zu den typischen Lagemaßen gehören \textbf{Mittelwert}, \textbf{Median} und \textbf{Modalwert}.
Der Mittelwert beschreibt die durchschnittliche Ausprägung, während der Median die geordnete Verteilung in zwei gleich große Hälften teilt. 
Der Median gilt als \emph{robustes Lagemaß}, da er – im Gegensatz zum Mittelwert – wenig durch Ausreißer beeinflusst wird. \cite{Dimić2019}
Für die Streuung werden Standardabweichung, Spannweite und insbesondere der \textbf{Interquartilsabstand (IQR)} verwendet.
Der IQR beschreibt die mittleren 50~\% der Daten und ist ein robustes Maß, das gegenüber Extremwerten stabil bleibt.
Für ordinale Merkmale ist der Median das geeignete Lagemaß; der IQR, ergänzt um Minimum und Maximum, quantifiziert die Streuung. \cite{Fisher2009}

\paragraph{Verteilungsformen und Schiefe}
Ein zentrales Merkmal numerischer Daten ist die Form ihrer Verteilung. 
In symmetrischen Verteilungen fallen Mittelwert, Median und Modus zusammen.
Bei \emph{rechtsschiefen} Verteilungen liegen einzelne hohe Werte weit über dem zentralen Bereich, sodass der Mittelwert größer als der Median ist; 
bei \emph{linksschiefen} Verteilungen gilt das umgekehrte Muster. \cite{Kaur2018}
Schiefe beeinflusst die Interpretation von Lage- und Streumaßen und motiviert den Einsatz robuster Kennwerte wie Median und IQR.
In der explorativen Praxis werden zudem log-transformierte Werte betrachtet, 
um stark asymmetrische Verteilungen zu symmetrisieren und visuell leichter interpretierbar zu machen.\cite{Marshall2010}

\paragraph{Log-Transformation und methodische Alternativen}
Wenn eine Verteilung deutlich von der Normalität abweicht, bestehen drei grundlegende Optionen:
(i) eine regelbasierte \emph{Ausreißerprüfung} mit dokumentierter Entfernung, \cite{Fisher2009}
(ii) eine \emph{Log-Transformation} zur Annäherung an Symmetrie und zur besseren Vergleichbarkeit in Visualisierungen \cite{Marshall2010},
oder (iii) die Anwendung \emph{nicht-parametrischer} Verfahren, die keine Normalverteilung voraussetzen \cite{Fisher2009}.
Parametrische Methoden setzen in der Regel Normalität voraus; die Entscheidung erfolgt im Rahmen der explorativen Analyse.

\subsection{Explorative Datenanalyse (EDA) und Visualisierungen}
Die explorative Datenanalyse ergänzt die deskriptive Statistik durch strukturentdeckende Verfahren. 
Sie dient der visuellen Erkundung und Bewertung von Mustern, Ausreißern oder Zusammenhängen zwischen Variablen, 
ohne dass zuvor Hypothesen formuliert werden müssen. 
Zentrale Visualisierungen sind Histogramme und Boxplots. \cite{Kaur2018}

\paragraph{Histogramme}
Histogramme stellen Häufigkeitsverteilungen kontinuierlicher Merkmale über Klassen dar. 
Sie erlauben Rückschlüsse auf Symmetrie, Schiefe und Mehrgipfligkeit und dienen zur
Prüfung von Verteilungsannahmen. \cite{Marshall2010}

\paragraph{Boxplots}
Boxplots visualisieren Median ($Q_2$), Quartile ($Q_1$, $Q_3$) und potenzielle Ausreißer.
Die sogenannten \emph{Whisker} markieren üblicherweise den Bereich bis zum 1{,}5-fachen Interquartilsabstand; 
Werte außerhalb gelten als potenzielle Ausreißer. \cite{Marshall2010} 
Diese Darstellungsform ermöglicht die Beurteilung von Streuung, Schiefe und Extremwerten
und eignet sich für den Vergleich mehrerer Merkmale. \cite{Kaur2018}

\subsection{Automatisierung, Reproduzierbarkeit und Datenqualität}
Eine \emph{skriptbasierte}, reproduzierbare Umsetzung (z.\,B.\ in~R) 
gewährleistet konsistente Analysen und Nachvollziehbarkeit aller Berechnungsschritte. 
Im Rahmen der explorativen Datenanalyse unterstützt sie die Datenqualitätsbewertung durch:
(i) Erkennung von Datenfehlern, Ausreißern und fehlenden Werten, 
(ii) Überwachung einfacher Profilierungsmaße (z.\,B.\ Anteil fehlender oder eindeutiger Werte) 
zur Bewertung von Vollständigkeit und Eindeutigkeit, sowie
(iii) Aufdeckung semantischer Inkonsistenzen und Formatabweichungen.
Diese systematische, nachvollziehbare Vorgehensweise bildet die Grundlage 
für die Qualitätssicherung entlang der Pipeline \emph{PDF~$\rightarrow$~JSON} 
und schafft Transparenz über den Zustand der analysierten Daten.


\subsection{Lineare Regression}

Die lineare Regression dient in dieser Arbeit als methodische Grundlage zur 
Modellierung der Umweltwirkungen von Produkten auf Basis quantitativer Einflussgrößen. 
Ziel ist es, Zusammenhänge zwischen erklärenden Variablen wie \emph{Produktgewicht}, \emph{Materialzusammensetzung}, \emph{Stromverbrauch} und \emph{verwendetem Energiemix}
und den resultierenden \emph{Umweltindikatoren} zu quantifizieren und zur Abschätzung unbekannter Werte nutzbar zu machen.

\paragraph{Modellstruktur}
Das Regressionsmodell beschreibt den linearen Zusammenhang zwischen einer abhängigen Variable \( y \) (z.\,B.\ einem Umweltindikator)
und mehreren unabhängigen Variablen \( x_1, x_2, \dots, x_k \) (z.\,B.\ Gewicht, Stromverbrauch, Materialanteile):
\[
y = \beta_0 + \beta_1 x_1 + \beta_2 x_2 + \dots + \beta_k x_k + \varepsilon.
\]
Dabei ist \( \beta_0 \) der Achsenabschnitt, \( \beta_i \) die Regressionskoeffizienten der jeweiligen Einflussgrößen 
und \( \varepsilon \) ein zufälliger Fehlerterm, der unerklärte Varianzanteile abbildet.
Die Koeffizienten \(\beta_i\) quantifizieren die Richtung und Stärke des Einflusses einzelner Variablen auf den Zielindikator. \cite{Montgomery2022}

\paragraph{Zentrale Annahmen}
Für die lineare Regression gelten folgende Grundannahmen:
\begin{itemize}
    \item \textbf{Linearität:} Die Beziehung zwischen abhängiger und unabhängigen Variablen ist näherungsweise linear.
    \item \textbf{Erwartungswert Null:} Die Fehlerterme haben einen Erwartungswert von null, \(E(\varepsilon)=0\).
    \item \textbf{Homoskedastizität:} Die Varianz der Fehler ist konstant und unabhängig von den Regressoren.
    \item \textbf{Unabhängigkeit:} Die Fehlerterme sind voneinander unkorreliert.
\end{itemize}
\cite{Su2012}
Diese Annahmen sichern die Unverzerrtheit und Effizienz der Parameterschätzungen. 
Für explorative Anwendungen, wie sie in dieser Arbeit verfolgt werden, steht jedoch die Strukturentdeckung im Vordergrund.
Moderate Abweichungen von den Idealannahmen sind daher akzeptabel, sofern sie dokumentiert werden.


%\paragraph{Parameterschätzung}
%Die Schätzung der Regressionskoeffizienten erfolgt nach der Methode der kleinsten Quadrate (\emph{Ordinary Least Squares}, OLS). 
%Dabei werden die Parameter so bestimmt, dass die Summe der quadrierten Abweichungen zwischen 
%beobachteten und modellierten Werten minimal wird:
%\[
%S(\boldsymbol{\beta}) = \sum_{i=1}^{n} (y_i - \hat{y}_i)^2.
%\]
%Die resultierenden Schätzer sind unter den genannten Modellannahmen unverzerrt 
%und besitzen die kleinste Varianz unter allen linearen, unverzerrten Schätzverfahren (Gauss-Markov-Eigenschaft). \cite{Su2012}

\paragraph{Modellinterpretation}
Die Koeffizienten \(\beta_i\) geben an, wie stark sich der Zielindikator \(y\) im Mittel verändert, 
wenn sich die Einflussgröße \(x_i\) um eine Einheit ändert, während alle anderen Variablen konstant bleiben. 
Das \emph{Bestimmtheitsmaß} \(R^2\) beschreibt den Anteil der Varianz des Zielindikators, der durch die 
erklärenden Variablen erklärt wird, und dient als zentrales Maß der Modellgüte. \cite{Montgomery2022}
Zur Bewertung der Modellangemessenheit werden Residuenanalysen eingesetzt, 
um Abweichungen von Linearität oder Homoskedastizität sichtbar zu machen. 

\paragraph{Anwendungsrahmen}
In dieser Arbeit wird die multiple lineare Regression verwendet, 
um Heuristiken zur Abschätzung der Umweltwirkungen von Elektro- und Elektronikprodukten zu entwickeln.
Das Modell dient der quantitativen Erfassung
von Zusammenhängen zwischen Produktmerkmalen und Umweltindikatoren und daraus schließend der möglichst präzisen Prognose 
der Umweltindikatoren anhand der Input-Variablen.
Damit bildet die lineare Regression eine nachvollziehbare, statistisch fundierte Basis für die 
Entwicklung eines vereinfachten Bewertungsmodells innerhalb der PEP-Datenanalyse.
