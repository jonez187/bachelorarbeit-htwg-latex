\section{Statistische Methoden}

\subsection{Deskriptive und explorative Statistik}

Die deskriptive und explorative Statistik bildet die Grundlage der Analyse der aus den PEP-Ecopassports extrahierten Daten. 
Sie dient der systematischen Aufbereitung, Verdichtung und visuellen Darstellung der Umweltindikatoren, 
Materialkompositionen und Energieverbrauchswerte. 
Ziel ist es, zentrale Muster, Streuungen und Ausreißer zu identifizieren, um sowohl die Datenqualität zu bewerten als auch
eine belastbare Basis für weiterführende statistische Auswertungen zu schaffen.

\paragraph{Grundlagen}
Die deskriptive Statistik befasst sich mit der Organisation und Beschreibung von Stichprobendaten.  \cite{Fisher2009}
Sie umfasst numerische und grafische Verfahren, um zentrale Tendenzen und Streuungen innerhalb einer Verteilung zu quantifizieren. 
Damit lassen sich große Datenmengen komprimiert darstellen und charakteristische Eigenschaften eines Datensatzes sichtbar machen. 
Da aus den PEP-Dokumenten extrahierte Umweltindikatoren häufig stark schiefe und ausreißerbehaftete Verteilungen aufweisen, 
werden in dieser Arbeit bevorzugt robuste Maße wie der Median und der Interquartilsabstand (IQR) verwendet. 
Diese Kennwerte sind unempfindlich gegenüber extremen Werten und beschreiben die mittleren 50 \% einer Verteilung. 
Zur besseren Vergleichbarkeit schief verteilter Daten werden ergänzend logarithmische Transformationen angewendet, 
um die Skalen zu stabilisieren und die visuelle Beurteilung der Daten zu erleichtern.

\paragraph{Visualisierung}
Zur Untersuchung der Verteilungen und zur Identifikation potenzieller Auffälligkeiten werden Histogramme und Boxplots eingesetzt. 
Histogramme ermöglichen eine schnelle Einschätzung der Verteilungsform und zeigen zentrale Tendenz, Streuung, Modalität und potenzielle Ausreißer. 
Boxplots stellen robuste Lage- und Streuungsmaße dar (Median und IQR) und verdeutlichen gleichzeitig Symmetrie, Schiefe und Extremwerte. 
Die parallele Darstellung mehrerer Boxplots erlaubt zudem den Vergleich zwischen unterschiedlichen Indikatoren oder Produktgruppen. 
Diese grafischen Verfahren werden in der explorativen Datenanalyse genutzt, um Muster oder Abweichungen ohne vorab definierte Hypothesen zu erkennen.

\paragraph{Automatisierung und Reproduzierbarkeit}
Die Auswertungen erfolgen reproduzierbar und skriptbasiert in R. 
Alle Berechnungsschritte, Transformationen und grafischen Ausgaben sind in modularen R-Skripten dokumentiert, 
sodass die Analyse vollständig nachvollziehbar und bei Bedarf automatisiert wiederholbar ist. 
Eine skriptbasierte Auswertung stellt sicher, dass auch nach Datenkorrekturen oder Erweiterungen identische Analyseschritte angewendet werden können.

\paragraph{Datenqualität und Plausibilitätsprüfung}
Die explorative Datenanalyse unterstützt zugleich die Bewertung der Datenqualität. 
Sie dient der Erkennung von Fehlern, Ausreißern, fehlenden oder inkonsistenten Werten 
und ermöglicht damit eine erste Plausibilitätsprüfung der extrahierten Datensätze. 
Durch die Beobachtung von Minimum-, Maximum- und Quartilswerten können unplausible Ausreißer oder fehlerhafte Einheiten erkannt werden. 
Somit leistet die EDA einen wesentlichen Beitrag zur Validierung der extrahierten PEP-Daten 
und zur Sicherstellung ihrer Eignung für nachgelagerte Analysen.
