\chapter{Diskussion}
\label{sec:discussion}

Aus den in Kapitel~\ref{sec:regression_co2} und 
\ref{sec:reg_other_indicators} vorgestellten Regressionsmodellen 
ergibt sich ein differenziertes Bild. Einige Indikatoren lassen sich 
sehr gut, andere nur eingeschränkt durch Gewicht, Stromverbrauch und der benutzten 
Materialien erklären. Dieses Kapitel diskutiert die Ergebnisse und 
bereitet damit die spätere Ableitung einer einfachen Heuristik zur Abschätzung der Indikatoren 
vor.

\section{Einordnung der CO$_2$-Regression}
\label{sec:disc_co2_main}

Die Diskussion bezieht sich auf das in Abschnitt~\ref{sec:regression_co2}
eingeführte Regressionsmodell, in dem die Zielvariable als
$\text{log\_cc} = \log(1 + \text{cc\_total})$ definiert ist. Als
Prädiktoren gehen das log-transformierte Produktgewicht
($\text{log\_w}$), der Stromverbrauch ($\text{log\_e}$) sowie die aus
dem Materialblock abgeleiteten PCA-Hauptkomponenten ein. Die Ergebnisse
basieren auf rund 173 PEPs und einer SGD-basierten linearen Regression
mit mehrfach wiederholten äußeren Train/Test-Splits.

\paragraph{Interpretationsebene}
Die Kombination aus Gewicht, Stromverbrauch und
Material-Hauptkomponenten erklärt mit 88\% einen großen Anteil der
Varianz der ausgewiesenen PEP-CO$_2$-Werte.
Praktisch bedeutet dies, dass das Modell auf Basis weniger,
relativ einfach zugänglicher Merkmale typischerweise die
Größenordnung der CO$_2$-Äquivalente trifft.
12\% der Varianz liegt damit noch außerhalb des Modells und kommt vermutlich
aus verschiedenen Rechenmethoden innerhalb der PEPs und feineren Einflussfaktoren wie
die Lebensdauer und das genaue benutzte Energiemodell. 
Der RMSE liegt grob im Bereich eines Faktors zwei bis drei
auf der Originalskala und erlaubt daher eher eine heuristische
Abschätzung als eine exakte Bilanzierung einzelner Geräte.

\begin{itemize}
\item \textbf{Bereiche mit guter bzw.\ schwächerer Modellpassung}
  \begin{itemize}
    \item Mittlere CO$_2$-Bereiche:
      \begin{itemize}
        \item Im Bereich typischer CO$_2$-Werte liegen die Punkte im
              Streudiagramm dicht um die Diagonale, sowohl für
              Trainings- als auch für Testdaten. Es ist keine
              ausgeprägte systematische Über- oder Unterschätzung
              erkennbar.
        \item Die Streuung der Testpunkte ist dabei sehr ähnlich zur
              Streuung der Trainingspunkte, was darauf hindeutet, dass
              das Modell den Zusammenhang in diesem Wertebereich gut
              erfasst und vernünftig generalisiert.
      \end{itemize}

    \item Extreme bzw.\ sehr große Produkte:
      \begin{itemize}
        \item Einige wenige PEPs mit sehr hohem Gewicht (z.\,B.\ 
              $\geq 720$\,kg) treten als deutliche Ausreißer auf und
              liegen sichtbar abseits der Diagonalen. Diese Beobachtungen
              tragen überproportional zur Gesamtsumme der Fehler bei und
              verzerren die Fit-Kennzahlen etwas.
        \item Naheliegende Ursachen sind abweichende Systemgrenzen oder
              besondere Anwendungsfälle (z.\,B.\ Spezialanlagen),
              zusätzliche Prozessschritte wie aufwendige Installation
              oder Transporte sowie mögliche Inkonsistenzen in den
              zugrunde liegenden PEP-Daten.
        \item Insgesamt spricht dies dafür, dass sich das Modell primär
              für typische Produkte der Stichprobe eignet und
              sehr große, außergewöhnliche Geräte nur eingeschränkt
              abbilden kann, weil sich deren Berechnung der Emission noch
              einmal deutlich vom Rest der PEPs unterscheidet.
      \end{itemize}

    \item Sehr kleine CO$_2$-Werte:
      \begin{itemize}
        \item Im Bereich sehr kleiner Emissionswerte ist die
              relative Streuung auf der log1p-Skala zwar sichtbarer,
              die absoluten Abweichungen bleiben jedoch gering.
        \item Für die Gesamtbewertung der Modellgüte sind diese
              Unterschiede daher weniger kritisch als Fehlerschätzungen
              im mittleren und hohen CO$_2$-Bereich, in dem der Großteil
              der betrachteten PEPs liegt und der Beitrag zum
              Gesamtemissionsniveau am größten ist.
      \end{itemize}
  \end{itemize}
\end{itemize}

\section{Grenzen und Unsicherheiten des Modells}

Die Aussagekraft der CO$_2$-Regression wird wesentlich durch die Qualität und
Homogenität der zugrunde liegenden PEP-Daten bestimmt. 
Die PEPs selbst beruhen
Hintergrunddatensätzen, Annahmen, unterschiedlichen Berechnungsmodellen und Systemgrenzen, womit sie keine
ultimative Wahrheitsquelle darstellen können. Das Modell lernt
damit nicht nur physikalische Zusammenhänge zwischen Gewicht, Stromverbrauch,
Materialmix und Emissionen, sondern auch diese Heterogenität mit. Ein Teil der
beobachteten Streuung ist daher als Unsicherheit zu
verstehen und dürfte selbst mit komplexeren Modellen kaum vollständig
eliminierbar sein.

Hinzu kommt, dass die Stichprobe zwar 173 PEPs umfasst, diese aber
unterschiedliche Produktkategorien abdecken. Positiv ist, dass das Modell der Zielsetzung entsprechend
geräteübergreifende Muster erkennen kann und damit nicht nur für einen einzelnen
Gerätetyp gültig ist. Gleichzeitig bedeutet die Produktvielfalt, dass für
spezielle Nischenprodukte ohne vergleichbare Vertreter in der Stichprobe mit
deutlich größeren Prognosefehlern gerechnet werden muss. Das Modell ist somit
vor allem als Näherung für typische Produkte innerhalb des betrachteten
Datenraums zu verstehen. 

Auch methodisch ergeben sich Einschränkungen. Die verwendete
\texttt{log1p}-Transformation stabilisiert die Regression, reduziert den
Einfluss extremer Werte und führt zu einem weitgehend linearisierbaren
Zusammenhang. Dadurch wird allerdings die direkte Interpretation der Fehler in kg
CO$_2$-Äquivalenten weniger intuitiv, da Fehler auf der
Log-Skala grob multiplikativen Abweichungen auf der Originalskala entsprechen.
Zudem kann das lineare Modell auf der Log-Skala theoretisch negative Werte
vorhersagen, die nach der Rücktransformation zu sehr kleinen, aber dennoch
positiven Emissionen führen. In der praktischen Anwendung könnte daher eine
Untergrenze der Vorhersagen sinnvoll sein.

Ein weiterer wichtiger Punkt ist die Indikatorabhängigkeit der Ergebnisse.
Die höchste und stabilste Modellgüte wird für den Indikator \emph{Climate
change (total)} erreicht. Für einige andere Indikatoren wie
\emph{Acidification}, \emph{Hazardous waste disposed}, \emph{Water use},
\emph{Photochemical ozone formation}, \emph{Resource use (fossils)} oder
\emph{Eutrophication (terrestrial)} liefert die gleiche Pipeline zwar noch
brauchbare Ergebnisse, die Test-R$^2$-Werte und die Streuung über die Splits
sind jedoch durchweg schwächer als im CO$_2$-Fall. Für mehrere weitere
Indikatoren mit geringer Modellgüte (vgl.\ Tab.~\ref{tab:weak_indicators})
gelingt es kaum, wenn überhaupt einen moderaten Anteil der Varianz zu erklären. Hier
scheinen die betrachteten Eingangsgrößen (Gewicht, Strom, Materialmix) nur
einen begrenzten Einfluss zu haben oder die Datenlage ist zu verrauscht.

Insgesamt kann die CO$_2$-Regression daher als obere Schranke der erreichbaren
Genauigkeit unter den gegebenen Daten- und Featureeinschränkungen gelesen
werden. Die im Mittel beobachtete Fehlerspanne bildet die Grundlage für die in
Kapitel~7 abgeleitete Heuristik zur Abschätzung von CO$_2$-Äquivalenten ohne
PEP. Auch in dieser Abschätzung kann man keine genauen Ergebnisse erwarten. 
Das Modell kann nur die Größenordnung zuverlässig einschätzen.
Außerdem sollte diese Heuristik ausdrücklich als auf den Indikator
\emph{Climate change (total)} beschränkt verstanden werden. Eine direkte
Übertragung auf andere Umweltindikatoren ist aufgrund der beschriebenen
Unterschiede in der Modellgüte nur eingeschränkt möglich.
