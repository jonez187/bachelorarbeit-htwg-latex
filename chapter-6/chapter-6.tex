\chapter{Diskussion}
\label{sec:discussion}

Aus den in Kapitel~\ref{sec:regression_co2} und 
\ref{sec:reg_other_indicators} vorgestellten Regressionsmodellen 
ergibt sich ein differenziertes Bild. Einige Indikatoren lassen sich 
sehr gut, andere nur eingeschränkt durch Gewicht, Stromverbrauch und der benutzten 
Materialien erklären. Dieses Kapitel ordnet die Ergebnisse ein und diskutiert Unsicherheiten und Grenzen.


\section{Einordnung der Ergebnisse}
\label{sec:disc_interpretation}

Die höchste und stabilste Modellgüte wird für den Indikator
\emph{Climate change (total)} erreicht. 
Insgesamt kann die CO$_2$ Regression als obere Schranke der erreichbaren
Genauigkeit unter den gegebenen Featureeinschränkungen interpretiert werden.

Das gewählte Feature-Set kann etwas 90\% der Vaianz der CO$_2$-Werte erklären,
was eine hohe Modellgüte beschreibt.
Da die geschätzten Werte im Mittel trotzdem um 120\% abweichen, ist auch dieses Modell 
nicht präzise. 
Für die anderen Indikatoren wurde die Pipeline im Wesentlichen nur über die
Transformation der Zielvariablen angepasst. 
Eine ähnlich ausführliche
explorative Entwicklung wie für CO$_2$ fand dort nicht statt.

Trotzdem lässt sich das Modell auf mehrere weitere Indikatoren wie
\emph{Acidification}, \emph{Resource use (fossils)} oder \emph{Water use} übertragen und
liefert brauchbare Ergebnisse, die Streuung ist jedoch größer. 
Für Indikatoren mit geringer Modellgüte
(vgl.\ Tab.~\ref{tab:weak_indicators}) gelingt es dagegen nur, einen
begrenzten Anteil der Varianz zu erklären. Dies deutet darauf hin, dass die
verwendeten Eingangsgrößen entweder nur einen schwachen Einfluss haben oder
dass die Datenlage für diese Zielgrößen stark verrauscht ist.

\section{Fehlerstruktur und Modellannahmen}
\label{sec:disc_residuals}

Ein zentrales Diagnoseergebnis ist, dass die Schätzfehler der Modelle nicht
normalverteilt sind. Dies zeigt sich in den QQ Plots durch systematische
Abweichungen von der Referenzgeraden, insbesondere in den äußeren Quantilen.
Die Residuen weisen damit schwere Verteilungsschwänze und Ausreißer auf.

Für die Interpretation bedeutet dies Folgendes. Klassische
Schlussfolgerungen der OLS Theorie, etwa Standardfehler und p Werte sind nur
eingeschränkt zuverlässig. Zudem ist die Fehlerverteilung trotzdem plausibel, 
da die Indikatoren über mehrere Größenordnungen streuen und einzelne Produkte
sehr große Umweltwirkungen aufweisen. 
Eine Transformation wie log1p oder Box Cox die Fehlerstruktur typischerweise, sie führt aber
nicht zu perfekter Normalität, weil Datenheterogenität und
Ausreißer weiterhin bestehen.

Für die Zielsetzung dieser Arbeit, nämlich robuste Vorhersagen für neue
Produkte mit einem simplen Modell, ist daher eine testbasierte Bewertung zentral. 
Ergänzend zu RMSE werden robuste Kennzahlen wie Median absoluter Fehler und Median relativer
Fehler berichtet. Dadurch wird die Modellleistung weniger durch
Extremwerte verzerrt und über verschiedene Größenordnungen hinweg besser
interpretierbar.


\section{Grenzen des Modells}

Die Aussagekraft der CO$_2$-Regression wird wesentlich durch die Qualität und
Homogenität der zugrunde liegenden PEP-Daten bestimmt. 
Die PEPs selbst beruhen auf
Hintergrunddatensätzen, Annahmen, unterschiedlichen Berechnungsmodellen und Systemgrenzen.
Damit können sie keine
ultimative Wahrheitsquelle darstellen. Das Modell lernt
damit nicht nur physikalische Zusammenhänge zwischen Gewicht, Stromverbrauch,
Materialmix und Emissionen, sondern auch diese Heterogenität mit. Ein Teil der
beobachteten Streuung ist daher als Unsicherheit zu
verstehen und dürfte selbst mit komplexeren Modellen kaum vollständig
eliminierbar sein.

Hinzu kommt, dass die Stichprobe zwar 173 PEPs umfasst, diese aber
unterschiedliche Produktkategorien abdecken. Positiv ist, dass das Modell der Zielsetzung entsprechend
geräteübergreifende die Muster und Trends erkennen kann und damit nicht nur für 
einen einzelnen Gerätetyp gültig ist. 
Gleichzeitig bedeutet die Produktvielfalt, dass vor allem
für Nischenprodukte ohne vergleichbare Vertreter in der Stichprobe mit
deutlich größeren Prognosefehlern gerechnet werden muss. Das Modell ist somit
vor allem als Näherung für typische Produkte innerhalb des betrachteten
Datenraums zu verstehen.