\section{Lineare Regression der anderen Indikatoren}
\label{sec:reg_other_indicators}

Die für CO$_2$ aufgebaute Regressions-Pipeline wird nun auf 
weitere Umweltindikatoren angewendet, um zu prüfen, wie gut 
sich diese mit denselben Produktmerkmalen erklären lassen.

\subsection{Regression des Indikators Acidification}
\label{sec:reg_acidification}

Neben den CO$_2$-Äquivalenten wurde das lineare Regressionsmodell auch auf den
Indikator \emph{Acidification} angewendet. Die Zielvariable ist dabei der
log-transformierte Gesamtwert des Indikators
$\text{log\_acid} = \log(1 + \text{acidification}_\text{total})$.
Es konnten 177 PEPs benutzt werden. 
Tabelle~\ref{tab:reg_acidification} fasst die Testleistung zusammen.

\begin{table}[h]
  \centering
  \caption{Gütekennzahlen der Regression für den Indikator
           \emph{Acidification} (Zielvariable auf der Skala $\log(1+\text{acidification})$)
           über $R = 100$ äußere Train/Test-Splits.}
  \label{tab:reg_acidification}
  \begin{tabular}{lcc}
    \toprule
    Größe & Mittelwert (Test) $\pm$ Std. & Bester Lauf (Test) \\
    \midrule
    $R^2_{\text{Test}}$           & $0.800 \pm 0.158$ & $0.923$ \\
    $\mathrm{RMSE}_{\text{Test}}$ & $0.975 \pm 0.281$ & $0.640$ \\
    \bottomrule
  \end{tabular}
\end{table}

Im Mittel erklärt das Modell damit rund 80\,\% der Varianz der
log-transformierten Acidification-Werte, bei zugleich moderater Streuung
über die verschiedenen Train/Test-Aufteilungen. Der beste Lauf erreicht ein
Test-$R^2$ von 0{,}923 bei einem RMSE von 0{,}640 (jeweils in
log1p-Einheiten der Zielvariable) und zeigt damit, dass das lineare Modell
mit Gewicht, Stromverbrauch und Material-PCs auch für diesen Indikator eine
sehr hohe Vorhersagegüte erreichen kann.

Zur Veranschaulichung zeigt Abbildung~\ref{fig:reg_acidification_scatter}
für den besten Lauf ein Streudiagramm der vorhergesagten gegenüber den
tatsächlichen Werten von \texttt{log\_acid}.

\begin{figure}[H]
  \centering
  \includegraphics[width=0.9\textwidth]{images/regression_acid.png}
  \caption{Vorhersagte gegenüber tatsächlichen Acidification-Werten
           (log1p-Skala) für den besten Lauf der Regression.}
  \label{fig:reg_acidification_scatter}
\end{figure}

Die meisten Punkte liegen in der Nähe der Diagonalen, insbesondere im
Bereich mittlerer bis hoher Acidification-Werte, was auf eine gute
Abbildung des allgemeinen Trends hinweist. Im Bereich sehr kleiner Werte
($\text{log1p} \approx 0$) ist die Streuung relativ größer, die absoluten
Fehler bleiben dort jedoch gering. Das Fehlerverhalten von Trainings- und
Testdaten ist vergleichbar, so dass keine starke Überanpassung erkennbar
ist. Insgesamt bestätigt die Analyse, dass das aus dem CO$_2$-Fall
übernommene Modell auch für den Indikator \emph{Acidification} robuste und
plausible Vorhersagen liefert.



\subsection{Regression des Indikators Hazardous Waste Disposed}
\label{sec:reg_hazardous_waste}

Für den Indikator Hazardous Waste Disposed wurden 168 PEP benutzt.

\begin{table}[h]
  \centering
  \caption{Gütekennzahlen der Regression für den Indikator
           \emph{Hazardous waste disposed} (Zielvariable auf der Skala
           $\log(1+\text{hazardous\_waste\_disposed})$) über
           $R = 100$ äußere Train/Test-Splits.}
  \label{tab:reg_hazardous_waste}
  \begin{tabular}{lcc}
    \toprule
    Größe & Mittelwert (Test) $\pm$ Std. & Bester Lauf (Test) \\
    \midrule
    $R^2_{\text{Test}}$           & $0.627 \pm 0.343$ & $0.906$ \\
    $\mathrm{RMSE}_{\text{Test}}$ & $1.780 \pm 0.622$ & $1.015$ \\
    \bottomrule
  \end{tabular}
\end{table}

Im Mittel erklärt das Modell rund 63\,\% der Varianz der
log-transformierten Werte, bei deutlich größerer Streuung der
Gütemaße als bei \emph{Climate change} oder \emph{Acidification}.
Der beste Lauf erreicht ein Test-$R^2$ von 0{,}906 bei einem
RMSE von 1{,}015 (log1p-Einheiten), einzelne Splits fallen jedoch
deutlich schwächer aus, was auf eine geringere Stabilität der
Regression hinweist.

\begin{figure}[H]
  \centering
  \includegraphics[width=0.9\textwidth]{images/regression_h_w_d.png}
  \caption{Vorhersagte gegenüber tatsächlichen Werten des Indikators
           \emph{Hazardous waste disposed} (log1p-Skala) für den besten
           Lauf der Regression.}
  \label{fig:reg_hazardous_waste_scatter}
\end{figure}

Das Streudiagramm zeigt, dass die Punkte insgesamt entlang der
Diagonalen $y=x$ liegen und der generelle Trend gut erfasst wird.
Gegenüber den Indikatoren \emph{Climate change} und \emph{Acidification}
ist der Streubereich jedoch sichtbar größer, insbesondere im mittleren
Wertebereich. Über- und Unterschätzungen treten in der Schätzung häufiger auf.
Insgesamt liefert das Modell sinnvolle, aber weniger präzise und
weniger robuste Vorhersagen als in den Fällen CO$_2$ und Acidification.


\subsection{Regression des Indikators Water Use}
\label{sec:reg_water_use}

Für den Indikator \emph{Water use} wird ebenfalls das lineare
Regressionsmodell mit Gewicht, Stromverbrauch und Material-PCs
verwendet. Die Zielvariable ist der log-transformierte Gesamtwert
$\log(1+\text{water\_use}_\text{total})$.
Tabelle~\ref{tab:reg_water_use} fasst die Testleistung über
$R = 100$ äußere Train/Test-Splits zusammen.

\begin{table}[h]
  \centering
  \caption{Gütekennzahlen der Regression für den Indikator
           \emph{Water use} (Zielvariable auf der Skala
           $\log(1+\text{water\_use})$) über
           $R = 100$ äußere Train/Test-Splits.}
  \label{tab:reg_water_use}
  \begin{tabular}{lcc}
    \toprule
    Größe & Mittelwert (Test) $\pm$ Std. & Bester Lauf (Test) \\
    \midrule
    $R^2_{\text{Test}}$           & $0.603 \pm 0.588$ & $0.872$ \\
    $\mathrm{RMSE}_{\text{Test}}$ & $2.196 \pm 0.753$ & $1.431$ \\
    \bottomrule
  \end{tabular}
\end{table}

Im Mittel erklärt das Modell damit rund 60\,\% der Varianz der
log-transformierten Water-use-Werte, allerdings mit einer sehr hohen
Streuung zwischen den einzelnen Train/Test-Aufteilungen. Der beste
Lauf erreicht ein Test-$R^2$ von 0{,}872 bei einem RMSE von 1{,}431
(log1p-Einheiten), liegt damit aber klar unter der Güte, die für
\emph{Climate change} und \emph{Acidification} erreicht wurde.

\begin{figure}[H]
  \centering
  \includegraphics[width=0.9\textwidth]{images/regression_water_use.png}
  \caption{Vorhersagte gegenüber tatsächlichen Werten des Indikators
           \emph{Water use} (log1p-Skala) für den besten Lauf der Regression.}
  \label{fig:reg_water_use_scatter}
\end{figure}

Das Streudiagramm in Abbildung~\ref{fig:reg_water_use_scatter} zeigt, dass die
meisten der 171 Produkte nahe der Diagonalen $y=x$ liegen und der
Wasserverbrauch damit insgesamt moderat getroffen wird. Für kleine bis mittlere
Werte (unter etwa 6 log1p-Einheiten) ist die Streuung relativ groß und es gibt
sowohl Über- als auch Unterschätzungen. Im Bereich höherer Werte liegen die Werte dagegen 
enger an der Diagonalen, so dass besonders wasserintensive
Produkte vergleichsweise zuverlässig vorhergesagt werden.
Insgesamt ist die Modellgüte für \emph{Water use}
damit als moderat und deutlich weniger stabil und präzise einzuschätzen
als für die besser erklärbaren Indikatoren.


\subsection{Regression des Indikators Photochemical Ozone Formation (HH)}
\label{sec:reg_pof_hh}

Für den Indikator \emph{Photochemical ozone formation, human health}
(\texttt{photochemical\_ozone\_formation\_hh}) wird dasselbe Modell wie in
Abschnitt~\ref{sec:regression_co2} auf insgesamt $n = 171$ PEPs angewendet.
Tabelle~\ref{tab:reg_pof_hh} zeigt die Testgüte über
$R = 100$ äußere Train/Test-Splits.

\begin{table}[h]
  \centering
  \caption{Gütekennzahlen der Regression für den Indikator
           \emph{Photochemical ozone formation, human health}
           (Zielvariable auf der Skala
           $\log(1+\text{photochemical\_ozone\_formation\_hh})$) über
           $R = 100$ äußere Train/Test-Splits.}
  \label{tab:reg_pof_hh}
  \begin{tabular}{lcc}
    \toprule
    Größe & Mittelwert (Test) $\pm$ Std. & Bester Lauf (Test) \\
    \midrule
    $R^2_{\text{Test}}$           & $0.626 \pm 0.182$ & $0.880$ \\
    $\mathrm{RMSE}_{\text{Test}}$ & $1.102 \pm 0.211$ & $0.717$ \\
    \bottomrule
  \end{tabular}
\end{table}

Im Mittel werden damit rund 63\,\% der Varianz der log-transformierten
Indikatorwerte erklärt. Die Streuung der Testgüte ist deutlich größer als
bei \emph{Climate change} und \emph{Acidification}, aber ähnlich wie bei
\emph{Hazardous waste disposed}. Der beste Lauf erreicht ein
Test-$R^2$ von 0{,}880 bei einem RMSE von 0{,}717 (log1p-Einheiten).

\begin{figure}[H]
  \centering
  \includegraphics[width=0.9\textwidth]{images/regression_rof.png}
  \caption{Vorhergesagte gegenüber tatsächlichen Werten des Indikators
           \emph{Photochemical ozone formation, human health}
           (log1p-Skala) für den besten Lauf der Regression
           ($n = 171$ PEPs).}
  \label{fig:reg_pof_hh_scatter}
\end{figure}

Das Streudiagramm in Abbildung~\ref{fig:reg_pof_hh_scatter} zeigt, dass für
sehr kleine Werte (nahe $0$ auf der $\log(1+x)$-Skala) eine relativ hohe
Streuung vorliegt und der Indikator ungefähr gleich häufig zu hoch wie zu
niedrig geschätzt wird. Im mittleren Bereich (etwa $0{,}5$ bis $2$
log1p-Einheiten) liegen die Punkte tendenziell oberhalb der Diagonalen und
das Modell überschätzt die Werte leicht. Für größere Werte ab ungefähr
$3$ log1p-Einheiten liegen die Punkte überwiegend unterhalb der Diagonalen,
so dass hohe Werte eher unterschätzt werden.


\subsection{Regression des Indikators Resource Use, Fossils}
\label{sec:reg_ruf}

\begin{table}[h]
  \centering
  \caption{Gütekennzahlen der Regression für den Indikator
           \emph{Resource Use, Fossils}
           (Zielvariable auf der Skala
           $\log(1+\text{photochemical\_ozone\_formation\_hh})$) über
           $R = 100$ äußere Train/Test-Splits.}
  \label{tab:reg_ru_f}
  \begin{tabular}{lcc}
    \toprule
    Größe & Mittelwert (Test) $\pm$ Std. & Bester Lauf (Test) \\
    \midrule
    $R^2_{\text{Test}}$           & $0.790 \pm 0.145$ & $0.940$ \\
    $\mathrm{RMSE}_{\text{Test}}$ & $1.669 \pm 0.463$ & $0.990$ \\
    \bottomrule
  \end{tabular}
\end{table}


Für den Indikator \emph{Resource use, fossils} erreicht das Modell über
alle im Mittel ein Test-$R^2$ von
$0{,}790 \pm 0{,}145$ bei einem $\mathrm{RMSE}_{\text{Test}}$ von
$1{,}669 \pm 0{,}463$ (log1p-Skala). Im besten Lauf werden sogar
ein Test-$R^2$ von 0{,}940 und ein RMSE von 0{,}990 erreicht. 
Der fossile Ressourcenverbrauch ähnlich gut geschätzt wie \emph{Acidification}.


\begin{figure}[H]
  \centering
  \includegraphics[width=0.9\textwidth]{images/regression_rof.png}
  \caption{Vorhergesagte gegenüber tatsächlichen Werten des Indikators
           \emph{Resource Use, fossils}
           (log1p-Skala) für den besten Lauf der Regression
           ($n = 171$ PEPs).}
  \label{fig:reg_resource_use_fossils_scatter}
\end{figure}


Das Streudiagramm in Abbildung~\ref{fig:reg_resource_use_fossils_scatter}
zeigt, dass die meisten der 171 Produkte dicht entlang der Diagonalen
$y=x$ liegen. Kleine und mittlere Werte weisen lediglich eine moderate
Streuung mit geringfügigen Über- und Unterschätzungen auf, während hohe
Werte besonders gut getroffen werden. Eine deutliche systematische Abweichung
ist hier nicht erkennbar.


\subsection{Regression des Indikators Eutrophication (terrestrial)}
\label{sec:reg_eutrophication_terr}

\begin{table}[h]
  \centering
  \caption{Gütekennzahlen der Regression für den Indikator
           \emph{Eutrophication, terrestrial}
           (Zielvariable auf der Skala
           $\log(1+\text{photochemical\_ozone\_formation\_hh})$) über
           $R = 100$ äußere Train/Test-Splits.}
  \label{tab:reg_eutrophication_t}
  \begin{tabular}{lcc}
    \toprule
    Größe & Mittelwert (Test) $\pm$ Std. & Bester Lauf (Test) \\
    \midrule
    $R^2_{\text{Test}}$           & $0.773 \pm 0.154$ & $0.932$ \\
    $\mathrm{RMSE}_{\text{Test}}$ & $1.155 \pm 0.312$ & $0.788$ \\
    \bottomrule
  \end{tabular}
\end{table}

Für den diesen Indikator wurde das
Regressionsmodell wie zuvor auf $n = 107$ PEPs angewendet. Über
$R = 100$ äußere Train/Test-Splits ergibt sich ein mittleres
Test-$R^2$ von $0{,}773 \pm 0{,}154$ bei einem
$\mathrm{RMSE}_{\text{Test}}$ von $1{,}155 \pm 0{,}312$
(log1p-Skala). Im besten Lauf werden ein
Test-$R^2$ von $0{,}932$ und ein RMSE von $0{,}788$ erreicht und liegen
damit auf einem ähnlich hohen Niveau wie bei \emph{Resource use,
fossils}.

\begin{figure}[H]
  \centering
  \includegraphics[width=0.9\textwidth]{images/regression_et.png}
  \caption{Vorhergesagte gegenüber tatsächlichen Werten des Indikators
           \emph{Eutrophication (terrestrial)}
           (log1p-Skala) für den besten Lauf der Regression
           ($n = 171$ PEPs).}
  \label{fig:reg_et}
\end{figure}

Das Streudiagramm in Abbildung~\ref{fig:reg_et}
zeigt, dass der Großteil der Punkte nahe der Diagonalen $y=x$ liegt.
Sehr kleine Werte (nahe $\log(1+x)=0$) streuen relativ stark, die
absoluten Fehler bleiben dort jedoch gering. Im mittleren Bereich um
2–3 log-Einheiten treten etwas stärkere Über- und Unterschätzungen auf,
während hohe Eutrophierungswerte ab etwa 4 log-Einheiten überwiegend gut
getroffen und nur leicht unterschätzt werden.


\subsection{Indikatoren mit geringer Modellgüte}
\label{sec:weak_indicators}

Neben den oben beschriebenen Indikatoren mit moderater bis hoher Modellgüte wurden
alle weiteren Umweltindikatoren mit derselben Regressionspipeline
geschätzt. Für einige Zielgrößen bleibt das erreichte
Test-$R^2$ jedoch deutlich unter 0{,}5, so dass hier nicht von einem
zuverlässigen Vorhersagemodell gesprochen werden kann. Tabelle~\ref{tab:weak_indicators}
fasst diese Indikatoren zusammen.

!TODO: Die Tabelle passt Layout technisch noch nicht!

\begin{table}[h]
  \centering
  \caption{Indikatoren mit geringer Modellgüte
           ($R^2_{\text{Test, mean}} < 0{,}5$) in der log1p-Skala
           der Zielvariablen über
           $R = 100$ äußere Train/Test-Splits.}
  \label{tab:weak_indicators}
  \resizebox{\textwidth}{!}{
  \begin{tabular}{lccc}
    \toprule
    Indikator &
    $R^2_{\text{Test}}$ (Mean $\pm$ Std.) &
    RMSE\textsubscript{Test} (Mean $\pm$ Std.) & Anzahl analysierter PEPs \\
    \midrule
    Eutrophication (freshwater)      & $0.323 \pm 0.276$ & $0.618 \pm 0.137$ & 133\\
    Eutrophication (marine)          & $0.239 \pm 1.395$ & $1.242 \pm 0.552$ & 107\\
    Ozone depletion                  & $-0.100 \pm 1.070$ & $0.008 \pm 0.004$ & 170\\
    Radioactive waste disposed       & $-0.018 \pm 2.604$ & $1.263 \pm 0.445$ & 158\\
    Resource use (minerals \& metals) & $0.401 \pm 0.311$ & $0.271 \pm 0.057$ & 175\\
    \bottomrule
  \end{tabular}
  }
\end{table}


Als besonders kritisches Beispiel zeigt Abbildung~\ref{fig:reg_ozone_depl_scatter}
den Indikator \emph{Ozone depletion}. Obwohl die absoluten Fehler aufgrund
der sehr kleinen Werte gering bleiben, liegen die Werte weit von der
Diagonalen entfernt und die Streuung ist hoch. Entsprechend ist das
durchschnittliche Test-$R^2$ sogar leicht negativ. 

\begin{figure}[H]
  \centering
  \includegraphics[width=0.9\textwidth]{images/regression_od.png}
  \caption{Vorhergesagte gegenüber tatsächlichen Werten des Indikators
           \emph{Ozone depletion}
           (log1p-Skala) für den besten Lauf der Regression.}
  \label{fig:reg_ozone_depl_scatter}
\end{figure}

Ähnliche Muster zeigen sich bei weiteren Indikatoren. 
Weitere Streudiagramme für alle hier 
aufgeführten Indikatoren sind im Anhang~\ref{app:scatter_weak_indicators} dargestellt.
! TODO: Anhang (?) !
