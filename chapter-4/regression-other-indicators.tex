\section{Lineare Regression der anderen Indikatoren}
\label{sec:reg_other_indicators}

Die für CO$_2$ aufgebaute Regressions-Pipeline wird nun auf 
weitere Umweltindikatoren angewendet, um zu prüfen, wie gut 
sich diese mit denselben Produktmerkmalen erklären lassen.

\subsection{Regression des Indikators Acidification}
\label{sec:reg_acidification}
\FloatBarrier

Neben den CO$_2$-Äquivalenten wurde das lineare Regressionsmodell auch auf den
Indikator \emph{Acidification} angewendet. 
Es wurden keine Transformation, \texttt{log1p} und 
eine Box-Cox Transformation verglichen.
Die beste Testgüte wird mit \texttt{log1p} erreicht, 
daher wird im Folgenden dieses Modell berichtet.
Daher wird im Folgenden dieses Modell berichtet.
Die Zielvariable ist dabei der
log-transformierte Gesamtwert des Indikators
$\text{log\_acid} = \log(1 + \text{acidification}_\text{total})$.
Es konnten 177 PEPs benutzt werden. 
Tabelle~\ref{tab:reg_acidification} fasst die Testleistung nach Rücktransformation 
zusammen.

\begin{table}[h]
  \centering
  \caption{Gütekennzahlen des linearen Regressionsmodells
           (\texttt{Acidification} als Zielvariable).}
  \label{tab:reg_acidification}
  \begin{tabular}{lc}
    \toprule
    Größe & Wert (Test) \\
    \midrule
    $R^2_{\text{Test}}$   & $0.845$ \\
    $\mathrm{RMSE}_{\text{Test}}$ & $490.87\,\mathrm{kg\ SO_2}$ \\
    \midrule
    Median absoluter Fehler & $0.2864\,\mathrm{kg\ SO_2}$ \\
    $\mathrm{MdARE}_{\text{Test}}$ (Median rel. Fehler) & $0.9579$ \\
    $\mathrm{MARE}_{\text{Test}}$ (Mittelwert rel. Fehler) & $3.8258$ \\
    \bottomrule
  \end{tabular}
\end{table}



Das Modell erklärt rund $85\%$ der Varianz mit einem RMSE von
$490{,}87\,\mathrm{kg\ SO_2}$, was auf eine insgesamt gute 
Vorhersagegüte für den Indikator \texttt{Acidification} hinweist.
Die robusten Fehlermaße ergänzen dieses Bild.
Der Median des absoluten Fehlers liegt weit unter dem RMSE 
bei etwa $0{,}29\,\mathrm{kg\ SO_2}$ und beschreibt damit die typische 
Abweichung eines repräsentativen Produkts.
Der Median der relativen Fehler ($\mathrm{MdARE} \approx 0{,}96$) zeigt, dass die 
Vorhersage für ein typisches Produkt häufig in der Größenordnung des wahren Werts liegt.
Der deutlich größere Mittelwert der relativen Fehler ($\mathrm{MARE} \approx 3{,}83$) weist 
zugleich auf eine stark schiefe Fehlerverteilung mit einzelnen sehr großen 
relativen Abweichungen hin.
Dies ist insbesondere bei kleinen Zielwerten plausibel, 
da dort bereits kleine absolute Fehler zu sehr großen relativen Fehlern führen.
Insgesamt ist das Modell damit für \texttt{Acidification} gut geeignet, 
einzelne Produkte können jedoch deutlich schlechter getroffen werden.


Zur Veranschaulichung zeigt Abbildung~\ref{fig:reg_acidification_scatter}
ein Streudiagramm der vorhergesagten gegenüber den
tatsächlichen Werten von \texttt{Acidification}.

\begin{figure}[h]
  \centering
  \includegraphics[width=0.9\textwidth]{images/regression_acid.png}
  \caption{Vorhergesagte gegenüber tatsächlichen Acidification-Werten. Beide Achsen sind 
  logarithmisch skaliert.}
  \label{fig:reg_acidification_scatter}
\end{figure}

Die meisten Punkte liegen in der Nähe der Diagonalen, insbesondere im
Bereich mittlerer Acidification-Werte, was auf eine gute
Abbildung des allgemeinen Trends hinweist. Hohe Werte werden tendenziell
leicht unterschätzt und im Bereich von $10^1$ gibt es ein Cluster, der überschätzt wird.
Im Bereich der sehr kleinen Werte schätzt das Modell relativ betrachtet sehr ungenau, 
die absoluten
Fehler bleiben dort jedoch gering. Das Fehlerverhalten von Trainings- und
Testdaten ist vergleichbar, so dass keine starke Überanpassung erkennbar
ist. Insgesamt bestätigt die Analyse, dass das aus dem CO$_2$-Fall
übernommene Modell auch für den Indikator \emph{Acidification} robuste und
plausible Vorhersagen liefert.

Abbildung~\ref{fig:regression_acid_qq} zeigt
einen QQ Plot der Schätzfehler des Modells im Vergleich zu einer Normalverteilung.
Wie beim CO$_2$ Indikator zeigen sich auf der Originalskala der Fehler 
typischerweise sehr starke Abweichungen, die eine schwache Interpretationsebene bereitstellen.
Daher wird hier nur die Transformationsskala dargestellt.
Im mittleren Bereich liegen die Punkte nah an der Referenzgeraden, was darauf hindeutet,
dass der Großteil der Fehler näherungsweise normalverteilt ist.
In den Randbereichen sind jedoch Abweichungen erkennbar.
Insgesamt ist die Fehlerverteilung im Zentrum gut durch eine Normalverteilung approximierbar,
während die Extrembereiche durch schwerere Verteilungsschwänze
geprägt sind.

\begin{figure}[h]
  \centering
  \includegraphics[width=0.95\textwidth]{images/regression_acid_qq.png}
  \caption{QQ Plot der Schätzfehler des Acidification Modells (Test- und Trainingsset).}
  \label{fig:regression_acid_qq}
\end{figure}


\FloatBarrier 
\subsection{Regression des Indikators Hazardous Waste Disposed}
\label{sec:reg_hazardous_waste}
\FloatBarrier

Für den Indikator \emph{Hazardous waste disposed} konnten $n = 168$ PEPs verwendet werden.
Es wurden keine Transformation, \texttt{log1p} und eine Box Cox Transformation verglichen.
Die beste Testgüte wird mit \texttt{log1p} erreicht, daher wird im Folgenden dieses Modell betrachtet.
Die Zielvariable ist damit
$\text{log\_hwd} = \log(1 + \text{hazardous\_waste\_disposed}_{\text{total}})$.
Tabelle~\ref{tab:reg_hazardous_waste} fasst die Testleistung nach Rücktransformation zusammen.

\begin{table}[h]
  \centering
  \caption{Gütekennzahlen des linearen Regressionsmodells
           (\texttt{Hazardous waste disposed} als Zielvariable).}
  \label{tab:reg_hazardous_waste}
  \begin{tabular}{lc}
    \toprule
    Größe & Wert (Test) \\
    \midrule
    $R^2_{\text{Test}}$   & $0.813$ \\
    $\mathrm{RMSE}_{\text{Test}}$ & $18489.74\,\mathrm{kg}$ \\
    \midrule
    Median absoluter Fehler & $176.7890\,\mathrm{kg}$ \\
    $\mathrm{MdARE}_{\text{Test}}$ (Median rel. Fehler) & $0.6639$ \\
    $\mathrm{MARE}_{\text{Test}}$ (Mittelwert rel. Fehler) & $13.1431$ \\
    \bottomrule
  \end{tabular}
\end{table}

Das Modell erklärt damit rund $81\%$ der Varianz auf dem Testset.
Der RMSE auf Originalskala ist vergleichsweise hoch, was auf einzelne sehr große Abweichungen hinweist.
Der Median des absoluten Fehlers liegt bei etwa $177\,\mathrm{kg}$ und somit erneut weit unter dem RMSE.
Der Median der relativen Fehler ($\mathrm{MdARE}\approx 66\%$) zeigt eine vergleichsweise gut Genauigkeit.
Der deutlich größere Mittelwert der relativen Fehler ($\mathrm{MARE}\approx 1314\%$) weist allerdings 
auf eine stark schiefe Fehlerverteilung mit wenigen extremen relativen Abweichungen hin, 
was insbesondere bei sehr kleinen Zielwerten plausibel ist.

Zur Veranschaulichung zeigt Abbildung~\ref{fig:reg_hazardous_waste_scatter}
ein Streudiagramm der vorhergesagten gegenüber den tatsächlichen Werten von \texttt{Hazardous waste disposed}.
Beide Achsen sind logarithmisch skaliert.

\begin{figure}[h]
  \centering
  \includegraphics[width=0.9\textwidth]{images/regression_h_w_d.png}
  \caption{Vorhergesagte gegenüber tatsächlichen Werten des Indikators
           \emph{Hazardous waste disposed}. Beide Achsen sind logarithmisch skaliert.}
  \label{fig:reg_hazardous_waste_scatter}
\end{figure}

Die meisten Punkte liegen im mittleren bis hohen Wertebereich nahe 
der Diagonalen, so dass der generelle Trend erfasst wird.
Bei sehr kleinen tatsächlichen Werten ist die Streuung deutlich größer 
und es treten ausgeprägte Über und Unterschätzungen auf.
Dies passt zu den hohen relativen Fehlern, da bereits kleine absolute 
Abweichungen dort sehr große relative Fehler erzeugen.

Abbildung~\ref{fig:reg_hazardous_waste_qq} zeigt den QQ Plot der Residuen auf der Transformationsskala.
Im Zentrum folgen die Punkte der Referenzgeraden weitgehend, in den Randbereichen sind Abweichungen sichtbar.
Damit ist die Normalitätsannahme näherungsweise im mittleren Bereich erfüllt, 
während einzelne große Fehler auf schwerere Verteilungsschwänze hinweisen.

\begin{figure}[h]
  \centering
  \includegraphics[width=0.95\textwidth]{images/regression_hwd_qq_new.png}
  \caption{QQ Plot der Schätzfehler des \emph{Hazardous waste disposed} Modells auf der Transformationsskala.}
  \label{fig:reg_hazardous_waste_qq}
\end{figure}

\FloatBarrier 
\subsection{Regression des Indikators Water Use}
\label{sec:reg_water_use}
\FloatBarrier

Für den Indikator \emph{Water use} wird dieselbe Regressionspipeline wie zuvor verwendet.
Es wurden keine Transformation, \texttt{log1p} und eine Box Cox Transformation verglichen.
Die beste Testgüte wird mit der Box Cox Transformation erreicht, daher wird im Folgenden dieses Modell berichtet.
Tabelle~\ref{tab:reg_water_use} fasst die Testleistung nach Rücktransformation zusammen.

\begin{table}[h]
  \centering
  \caption{Gütekennzahlen des linearen Regressionsmodells
           (\texttt{Water use} als Zielvariable).}
  \label{tab:reg_water_use}
  \begin{tabular}{lc}
    \toprule
    Größe & Wert (Test) \\
    \midrule
    $R^2_{\text{Test}}$   & $0{,}726$ \\
    $\mathrm{RMSE}_{\text{Test}}$ & $28682{,}36$ \\
    \midrule
    Median absoluter Fehler & $1353{,}5625$ \\
    $\mathrm{MdARE}_{\text{Test}}$ (Median rel. Fehler) & $0{,}9297$ \\
    $\mathrm{MARE}_{\text{Test}}$ (Mittelwert rel. Fehler) & $2{,}3278$ \\
    \bottomrule
  \end{tabular}
\end{table}


Das Modell erklärt damit rund $73\%$ der Varianz auf dem Testset.
Der RMSE auf Originalskala beträgt $28682{,}36$ und wird durch 
größere Abweichungen einzelner Produkte geprägt.
Der Median des absoluten Fehlers liegt bei $1353{,}56$ und beschreibt
die typische Abweichung eines repräsentativen Produkts.
Der Median der relativen Fehler ($\mathrm{MdARE}\approx 93\%$) zeigt, dass 
die Vorhersage für ein typisches Produkt häufig in der Größenordnung des wahren Werts liegt.
Der Mittelwert der relativen Fehler ($\mathrm{MARE}\approx 233\%$) ist deutlich höher und weist 
auf eine schiefe Fehlerverteilung mit einzelnen deutlich schlechter vorhergesagten Beobachtungen hin.

Zur Veranschaulichung zeigt Abbildung~\ref{fig:reg_water_use_scatter}
ein Streudiagramm der vorhergesagten gegenüber den tatsächlichen Werten von \texttt{Water use}.
Beide Achsen sind logarithmisch skaliert.

\begin{figure}[h]
  \centering
  \includegraphics[width=0.9\textwidth]{images/regression_water_use.png}
  \caption{Vorhergesagte gegenüber tatsächlichen Werten des Indikators
           \emph{Water use}. Beide Achsen sind logarithmisch skaliert.}
  \label{fig:reg_water_use_scatter}
\end{figure}

Im mittleren Wertebereich liegen viele Punkte in der Nähe der Diagonalen.
Der generelle Zusammenhang wird somit erfasst.
Bei sehr großen tatsächlichen Werten liegen mehrere Punkte unterhalb der 
Diagonalen, was auf eine Unterschätzung besonders wasserintensiver Produkte hindeutet.
Bei sehr kleinen tatsächlichen Werten ist die Streuung auf der Log Darstellung relativ groß, 
wodurch sich sehr große relative Fehler ergeben können.
Allgemein wird eine größere Streunng beobachtet als bei den vorherigen Indikatoren, welche sich in den 
Gütekennzahlen widerspiegelt.

Abbildung~\ref{fig:reg_water_use_qq} zeigt den QQ Plot der Residuen auf der Transformationsskala.
Im Zentrum liegen die Punkte näher an der Referenzgeraden, 
in den Randbereichen treten deutliche Abweichungen auf.
Besonders in dem größten dargestellten Quantil kann kein Zusammenhang mit einer Normalverteilung
beobachtet werden.
Dies weist auf schwerere Verteilungsschwänze und einzelne extreme Residuen hin.

\begin{figure}[h]
  \centering
  \includegraphics[width=0.95\textwidth]{images/regression_water_use_qq.png}
  \caption{QQ Plot der Residuen des \emph{Water use} Modells auf der Transformationsskala.}
  \label{fig:reg_water_use_qq}
\end{figure}


\FloatBarrier 
\subsection{Regression des Indikators Photochemical Ozone Formation (HH)}
\label{sec:reg_pof_hh}
\FloatBarrier

Für den Indikator \emph{Photochemical ozone formation, human health}
(\texttt{photochemical\_ozone\_formation\_hh}) wird dieselbe Regressionspipeline wie zuvor verwendet.
Es wurden keine Transformation, \texttt{log1p} und eine Box-Cox Transformation verglichen.
Die beste Testgüte wird mit der Box-Cox Transformation erreicht, daher wird im 
Folgenden dieses Modell berichtet.

Es konnten $n = 171$ PEPs verwendet werden.
Tabelle~\ref{tab:reg_pof_hh} fasst die Testleistung nach Rücktransformation zusammen.

\begin{table}[h]
  \centering
  \caption{Gütekennzahlen des linearen Regressionsmodells
           (\texttt{Photochemical ozone formation, human health} als Zielvariable).}
  \label{tab:reg_pof_hh}
  \begin{tabular}{lc}
    \toprule
    Größe & Wert (Test) \\
    \midrule
    $R^2_{\text{Test}}$   & $0{,}802$ \\
    $\mathrm{RMSE}_{\text{Test}}$ & $109{,}1708\,\mathrm{kg\ C_2H_4}$ \\
    \midrule
    Median absoluter Fehler & $0{,}0255\,\mathrm{kg\ C_2H_4}$ \\
    $\mathrm{MdARE}_{\text{Test}}$ (Median rel. Fehler) & $0{,}6665$ \\
    $\mathrm{MARE}_{\text{Test}}$ (Mittelwert rel. Fehler) & $1{,}2811$ \\
    \bottomrule
  \end{tabular}
\end{table}

Das Modell erklärt damit rund $80\%$ der Varianz auf dem Testset.
Der Median des absoluten Fehlers liegt bei $0{,}0255\,\mathrm{kg\ C_2H_4}$ und beschreibt 
die typische Abweichung eines repräsentativen Produkts.
Der Median der relativen Fehler ($\mathrm{MdARE}\approx 67\%$) deutet 
auf eine moderate typische relative Genauigkeit hin.
Der höhere Mittelwert der relativen Fehler ($\mathrm{MARE}\approx 128\%$) 
weist auf eine schiefe Fehlerverteilung mit einigen deutlich schlechter getroffenen Beobachtungen hin.

Zur Veranschaulichung zeigt Abbildung~\ref{fig:reg_pof_hh_scatter}
ein Streudiagramm der vorhergesagten gegenüber den tatsächlichen Werten von \texttt{POF (HH)}.
Beide Achsen sind logarithmisch skaliert.

\begin{figure}[h]
  \centering
  \includegraphics[width=0.9\textwidth]{images/regression_pof.png}
  \caption{Vorhergesagte gegenüber tatsächlichen Werten des Indikators
           \emph{Photochemical ozone formation, human health}. Beide Achsen sind logarithmisch skaliert.}
  \label{fig:reg_pof_hh_scatter}
\end{figure}

Im mittleren Werte\-be\-reich liegen viele Punkte nahe der Diago\-nalen, 
während bei sehr kleinen Werten eine deutlich größere Streuung sichtbar ist.
Für größere tatsächliche Werte liegen mehrere Punkte unterhalb der Diagonalen, was auf eine tendenzielle Unterschätzung hoher Werte hindeutet.
Einige Punkte liegen deutlich oberhalb der Diagonalen, was auf wenige starke Überschätzungen einzelner Beobachtungen hinweist.

Abbildung~\ref{fig:reg_pof_hh_qq} zeigt den QQ Plot der Residuen auf der Transformationsskala.
Einige Punkte liegen nahe an der Referenzgeraden,
allerdings weichen die meisten Punkte, vor allem in den Randbereichen, deutliche ab.
Dies spricht für schwerere Verteilungsschwänze und einzelne extreme Residuen.

\begin{figure}[h]
  \centering
  \includegraphics[width=0.95\textwidth]{images/regression_pof_qq.png}
  \caption{QQ Plot der Residuen des \emph{Photochemical ozone formation, human health} Modells auf der Transformationsskala.}
  \label{fig:reg_pof_hh_qq}
\end{figure}


\FloatBarrier 
\subsection{Regression des Indikators Resource Use, Fossils}
\label{sec:reg_ruf}

Für den Indikator \emph{Resource use, fossils} wird dieselbe Regressionspipeline wie zuvor verwendet.
Es wurden keine Transformation, \texttt{log1p} und eine Box-Cox Transformation verglichen.
Die beste Testgüte wird mit \texttt{log1p} erreicht, daher wird dieses Modell benutzt.
Tabelle~\ref{tab:reg_ru_f} fasst die Testleistung nach Rücktransformation zusammen.

\begin{table}[h]
  \centering
  \caption{Gütekennzahlen des linearen Regressionsmodells
           (\texttt{Resource use, fossils} als Zielvariable).}
  \label{tab:reg_ru_f}
  \begin{tabular}{lc}
    \toprule
    Größe & Wert (Test) \\
    \midrule
    $R^2_{\text{Test}}$   & $0{,}871$ \\
    $\mathrm{RMSE}_{\text{Test}}$ & $1119662{,}28\,\mathrm{MJ}$ \\
    \midrule
    Median absoluter Fehler & $14460{,}95\,\mathrm{MJ}$ \\
    $\mathrm{MdARE}_{\text{Test}}$ (Median rel. Fehler) & $0{,}6370$ \\
    $\mathrm{MARE}_{\text{Test}}$ (Mittelwert rel. Fehler) & $11{,}1394$ \\
    \bottomrule
  \end{tabular}
\end{table}

Das Modell erklärt damit rund $87\%$ der Varianz auf dem Testset und erreicht eine insgesamt hohe Vorhersagegüte.
Der RMSE auf Originalskala wird durch einzelne sehr große Abweichungen dominiert.
Der Median des absoluten Fehlers liegt bei $14460{,}95\,\mathrm{MJ}$ und beschreibt die typische 
Abweichung eines repräsentativen Produkts.
Der Median der relativen Fehler ($\mathrm{MdARE}\approx 64\%$) deutet auf eine moderate typische 
relative Genauigkeit hin.
Der deutlich größere Mittelwert der relativen Fehler ($\mathrm{MARE}\approx 1114\%$) 
weist auf eine stark schiefe Fehlerverteilung mit wenigen extremen relativen Abweichungen hin, 
was insbesondere bei sehr kleinen Zielwerten plausibel ist.

Zur Veranschaulichung zeigt Abbildung~\ref{fig:reg_ruf_scatter}
ein Streudiagramm der vorhergesagten gegenüber den tatsächlichen Werten von \emph{Resource use, fossils}.
Beide Achsen sind logarithmisch skaliert.

\begin{figure}[h]
  \centering
  \includegraphics[width=0.9\textwidth]{images/regression_ruf.png}
  \caption{Vorhergesagte gegenüber tatsächlichen Werten des Indikators
           \emph{Resource use, fossils}. Beide Achsen sind logarithmisch skaliert.}
  \label{fig:reg_ruf_scatter}
\end{figure}

Die meisten Punkte liegen über einen breiten Wertebereich nahe der Diagonalen, so dass der generelle Trend gut erfasst wird.
Im mittleren Bereich ist eine moderate Streuung sichtbar, während sehr große Werte überwiegend in der Nähe der Diagonalen liegen.
Einzelne Punkte liegen deutlich oberhalb der Diagonalen, was auf wenige starke Überschätzungen hinweist und zur Schiefe der Fehlerverteilung beiträgt.

Abbildung~\ref{fig:reg_ruf_qq} zeigt den QQ Plot der Residuen auf der Transformationsskala.
Im Zentrum folgen die Punkte der Referenzgeraden weitgehend, in den Randbereichen treten Abweichungen auf.
Dies weist auf schwerere Verteilungsschwänze und einzelne extreme Residuen hin.

\begin{figure}[h]
  \centering
  \includegraphics[width=0.95\textwidth]{images/regression_ruf_qq.png}
  \caption{QQ Plot der Residuen des \emph{Resource use, fossils} Modells auf der Transformationsskala.}
  \label{fig:reg_ruf_qq}
\end{figure}


\FloatBarrier 
\subsection{Regression des Indikators Eutrophication (terrestrial)}
\label{sec:reg_eutrophication_terr}

Für den Indikator \emph{Eutrophication, terrestrial} konnten $n = 107$ PEPs
verwendet werden.
Es wurden keine Transformation, \texttt{log1p} und eine Box Cox Transformation
verglichen.
Die beste Testgüte wird mit der Box Cox Transformation erreicht, daher wird
im Folgenden dieses Modell berichtet.

Tabelle~\ref{tab:reg_eutrophication_t} fasst die Testleistung nach
Rücktransformation zusammen.

\begin{table}[h]
  \centering
  \caption{Gütekennzahlen des linearen Regressionsmodells
           (\texttt{Eutrophication, terrestrial} als Zielvariable).}
  \label{tab:reg_eutrophication_t}
  \begin{tabular}{lc}
    \toprule
    Größe & Wert (Test) \\
    \midrule
    $R^2_{\text{Test}}$   & $0{,}793$ \\
    $\mathrm{RMSE}_{\text{Test}}$ & $99{,}33\,\mathrm{mol\ N}$ \\
    \midrule
    Median absoluter Fehler & $0{,}0854\,\mathrm{mol\ N}$ \\
    $\mathrm{MdARE}_{\text{Test}}$ (Median rel. Fehler) & $0{,}5611$ \\
    $\mathrm{MARE}_{\text{Test}}$ (Mittelwert rel. Fehler) & $1{,}1653$ \\
    \bottomrule
  \end{tabular}
\end{table}

Das Modell erklärt damit rund $79\%$ der Varianz auf dem Testset.
Der RMSE beträgt $99{,}33\,\mathrm{mol\ N}$ und wird durch größere Abweichungen
einzelner Produkte geprägt.
Der Median des absoluten Fehlers liegt bei $0{,}0854\,\mathrm{mol\ N}$ und
beschreibt die typische Abweichung eines repräsentativen Produkts.
Der Median der relativen Fehler ($\mathrm{MdARE}\approx 56\%$) deutet auf eine
moderate typische relative Genauigkeit hin.
Der höhere Mittelwert der relativen Fehler ($\mathrm{MARE}\approx 117\%$) weist
auf eine schiefe Fehlerverteilung mit einigen deutlich schlechter getroffenen
Beobachtungen hin.

Zur Veranschaulichung zeigt Abbildung~\ref{fig:reg_et_scatter} ein
Streudiagramm der vorhergesagten gegenüber den tatsächlichen Werten.
Beide Achsen sind logarithmisch skaliert.

\begin{figure}[h]
  \centering
  \includegraphics[width=0.9\textwidth]{images/regression_et.png}
  \caption{Vorhergesagte gegenüber tatsächlichen Werten des Indikators
           \emph{Eutrophication, terrestrial}. Beide Achsen sind logarithmisch
           skaliert.}
  \label{fig:reg_et_scatter}
\end{figure}

Die meisten Punkte liegen nahe der Diagonalen, insbesondere im mittleren
Wertebereich.
Bei sehr kleinen tatsächlichen Werten ist die Streuung größer, was zu höheren
relativen Fehlern führen kann.
Einige größere Werte liegen unterhalb der Diagonalen, was auf eine leichte
Unterschätzung im oberen Bereich hindeutet.

Abbildung~\ref{fig:reg_et_qq} zeigt den QQ Plot der Residuen auf der
Transformationsskala.
Im Zentrum folgen die Punkte der Referenzgeraden weitgehend, in den
Randbereichen sind Abweichungen sichtbar.
Dies weist auf schwerere Verteilungsschwänze und einzelne extreme Residuen hin.

\begin{figure}[h]
  \centering
  \includegraphics[width=0.95\textwidth]{images/regression_et_qq.png}
  \caption{QQ Plot der Residuen des \emph{Eutrophication, terrestrial} Modells
           auf der Transformationsskala.}
  \label{fig:reg_et_qq}
\end{figure}


\FloatBarrier 
\subsection{Regression des Indikators Ozone Depletion}
\label{sec:reg_ozone_depletion}

Das lineare Regressionsmodell wird analog zu den vorherigen Indikatoren
auf \texttt{ozone\_depletion} angewendet.
Es wurden keine Transformation, \texttt{log1p} und eine Box-Cox Transformation
verglichen.
Die beste Testgüte ergibt sich mit der Box-Cox Transformation.
Daher wird im Folgenden dieses Modell berichtet.
Es konnten $n = 170$ PEPs benutzt werden.
Tabelle~\ref{tab:reg_ozone_depletion} fasst die Testleistung nach
Rücktransformation zusammen.

\begin{table}[h]
  \centering
  \caption{Gütekennzahlen des linearen Regressionsmodells
           (\texttt{Ozone depletion} als Zielvariable).}
  \label{tab:reg_ozone_depletion}
  \begin{tabular}{lc}
    \toprule
    Größe & Wert (Test) \\
    \midrule
    $R^2_{\text{Test}}$ & $0.858$ \\
    $\mathrm{RMSE}_{\text{Test}}$ & $0.0029$ \\
    \midrule
    Median absoluter Fehler & $0.0000$ \\
    $\mathrm{MdARE}_{\text{Test}}$ (Median rel. Fehler) & $0.8635$ \\
    $\mathrm{MARE}_{\text{Test}}$ (Mittelwert rel. Fehler) & $1.0411$ \\
    \bottomrule
  \end{tabular}
\end{table}

Das Modell erklärt auf dem Testset rund $86\%$ der Varianz.
Der RMSE auf Originalskala ist klein, was zur sehr kleinen Größenordnung des
Indikators passt.
Der Median des absoluten Fehlers wird in der Ausgabe auf vier Dezimalstellen
gerundet, daher erscheint er als $0.0000$.
Die relativen Fehlermaße zeigen dennoch eine deutlich heterogene Fehlerstruktur.
Ein typisches Produkt weist einen relativen Fehler von etwa
$\mathrm{MdARE}\approx 0.86$ auf.
Der größere Mittelwert $\mathrm{MARE}\approx 1.04$ deutet auf einzelne
Beobachtungen mit sehr großen relativen Abweichungen hin.
Dies ist bei \texttt{Ozone depletion} plausibel, da bei sehr kleinen Zielwerten
bereits kleine absolute Fehler zu großen relativen Fehlern führen.

Zur Veranschaulichung zeigt Abbildung~\ref{fig:reg_ozone_depletion_scatter}
ein Streudiagramm der vorhergesagten gegenüber den tatsächlichen Werten von
\texttt{Ozone depletion} auf logarithmischen Achsen.

\begin{figure}[h]
  \centering
  \includegraphics[width=0.9\textwidth]{images/regression_od.png}
  \caption{Vorhergesagte gegenüber tatsächlichen Werten von
           \texttt{Ozone depletion}. Beide Achsen sind logarithmisch skaliert.}
  \label{fig:reg_ozone_depletion_scatter}
\end{figure}

Das Streudiagramm zeigt eine grobe Ausrichtung entlang der Diagonalen.
Bei großen Zielwerten liegen viele Punkte unterhalb der Referenzlinie, was auf
eine Tendenz zur Unterschätzung im oberen Wertebereich hinweist.
Gleichzeitig existieren einzelne Fälle mit starker Überschätzung bei sehr
kleinen Zielwerten, die auf der Log Skala als große vertikale Abweichungen
erscheinen und die relativen Fehler stark erhöhen.

Abbildung~\ref{fig:reg_ozone_depletion_qq} zeigt den QQ Plot der Residuen auf der
Transformationsskala.
Im Zentrum liegen die Punkte näher an der Referenzgeraden, während die oberen
Quantile deutlich nach oben abweichen.
Dies weist auf eine rechtsschiefe Fehlerverteilung mit wenigen sehr großen
positiven Fehlern hin, konsistent mit den extremen Überschätzungen bei kleinen
Zielwerten.

\begin{figure}[h]
  \centering
  \includegraphics[width=0.95\textwidth]{images/regression_od_qq.png}
  \caption{QQ Plot der Schätzfehler des Ozone depletion Modells
           auf der Transformationsskala.}
  \label{fig:reg_ozone_depletion_qq}
\end{figure}


\FloatBarrier 
\subsection{Regression des Indikators Resource Use, Minerals and Metals}
\label{sec:reg_ru_mm}

Für den Indikator \emph{Resource use, minerals and metals} wurde das
lineare Regressionsmodell mit Gewicht, Stromverbrauch und Material-PCs
angewendet.
Es wurden keine Transformation, \texttt{log1p} und Box-Cox verglichen.
Die beste Testgüte wurde mit Box-Cox erreicht, daher wird dieses Modell
im Folgenden berichtet.

Tabelle~\ref{tab:reg_ru_mm} fasst die Testleistung nach Rücktransformation
auf die Originalskala zusammen.

\begin{table}[h]
  \centering
  \caption{Gütekennzahlen des linearen Regressionsmodells
           (\texttt{Resource use, minerals and metals} als Zielvariable).}
  \label{tab:reg_ru_mm}
  \begin{tabular}{lc}
    \toprule
    Größe & Wert (Test) \\
    \midrule
    $R^2_{\text{Test}}$ & $0.866$ \\
    $\mathrm{RMSE}_{\text{Test}}$ (Originalskala) & $0.7218$ \\
    \midrule
    Median absoluter Fehler & $0.0010$ \\
    $\mathrm{MdARE}_{\text{Test}}$ (Median rel. Fehler) & $0.7798$ \\
    $\mathrm{MARE}_{\text{Test}}$ (Mittelwert rel. Fehler) & $1.0003$ \\
    \bottomrule
  \end{tabular}
\end{table}

Das Modell erreicht eine hohe Testgüte ($R^2 \approx 0{,}87$) und bildet
den Indikator insgesamt gut ab.
Der Median der relativen Fehler zeigt, dass ein typisches Produkt oft in
der richtigen Größenordnung getroffen wird
($\mathrm{MdARE} \approx 0{,}78$).
Der größere Mittelwert der relativen Fehler
($\mathrm{MARE} \approx 1{,}00$) weist auf einzelne Fälle mit deutlich
größerer relativer Abweichung hin, bleibt aber insgesamt in einem
moderaten Rahmen.

Abbildung~\ref{fig:reg_ru_mm_scatter} zeigt die vorhergesagten gegenüber
den tatsächlichen Werten auf logarithmischen Achsen.

\begin{figure}[h]
  \centering
  \includegraphics[width=0.9\textwidth]{images/regression_rumm.png}
  \caption{Vorhergesagte gegenüber tatsächlichen Werten des Indikators
           \emph{Resource use, minerals and metals}.
           Beide Achsen sind logarithmisch skaliert.}
  \label{fig:reg_ru_mm_scatter}
\end{figure}

Der Hauptteil der Punkte liegt nahe der Diagonalen, während einzelne
Beobachtungen deutlich abweichen.
Das passt zur Kombination aus gutem $R^2$ und einer Fehlerverteilung mit
Ausreißern.

Abbildung~\ref{fig:reg_ru_mm_qq} zeigt einen QQ Plot der Residuen auf der
Transformationsskala.
Im Zentrum folgen die Residuen näherungsweise der Normalverteilung,
während die äußeren Quantile schwere Schwänze zeigen.

\begin{figure}[h]
  \centering
  \includegraphics[width=0.95\textwidth]{images/regression_rumm_qq.png}
  \caption{QQ Plot der Residuen des Modells für
           \emph{Resource use, minerals and metals} auf der
           Transformationsskala.}
  \label{fig:reg_ru_mm_qq}
\end{figure}


\FloatBarrier
\subsection{Indikatoren mit geringer Modellgüte}

Neben den oben beschriebenen Indikatoren mit moderater bis hoher Modellgüte wurden
alle weiteren Umweltindikatoren mit derselben Regressionspipeline geschätzt.
Für einige Zielgrößen bleibt das erreichte Test-$R^2$ jedoch unter $0{,}5$, so dass hier
nicht von einem zuverlässigen Vorhersagemodell gesprochen werden kann.
Tabelle~\ref{tab:weak_indicators} fasst diese Indikatoren zusammen.

Für diese schwächer erklärbaren Zielgrößen wurden ebenfalls verschiedene
Transformationen der Zielvariable verglichen. Box-Cox kann die Testgüte für alle 
restlichen Indikatoren leicht
verbessern, die Zugewinne bleiben jedoch insgesamt begrenzt und reichen nicht aus,
um die Indikatoren in den Bereich stabiler Vorhersagegüte zu überführen.

\begin{table}[h]
  \centering
  \caption{Indikatoren und Gütemaße mit geringer Modellgüte
           ($R^2_{\text{Test, mean}} < 0{,}5$).}
  \label{tab:weak_indicators}
  \resizebox{\textwidth}{!}{
  \begin{tabular}{lccc}
    \toprule
    Indikator &
    $R^2_{\text{Test}}$  &
    RMSE\textsubscript{Test} & Anzahl analysierter PEPs \\
    \midrule
    Eutrophication (freshwater)      & $0.434$ & $1.3165$ & 133 \\
    Eutrophication (marine)          & $0.322$ & $25.0197$ & 107 \\
    Radioactive waste disposed       & $0.492$ & $176.8942$ & 158 \\
    \bottomrule
  \end{tabular}
  }
\end{table}

Als besonders kritisches Beispiel zeigt Abbildung~\ref{fig:reg_rwd}
den Indikator \emph{Radioactive Waste Disposed}. Obwohl die absoluten Fehler aufgrund
der sehr kleinen Werte gering bleiben, liegen viele Punkte weit von der
Diagonalen entfernt und die Streuung ist hoch. Entsprechend ist das
durchschnittliche Test-$R^2$ klein. Eine Box-Cox Transformation kann die Verteilung
der Fehler nicht glätten.

\begin{figure}[H]
  \centering
  \includegraphics[width=0.9\textwidth]{images/regression_rwd.png}
  \caption{Vorhergesagte gegenüber tatsächlichen Werten des Indikators
           \emph{Radioactive Waste Disposed}.}
  \label{fig:reg_rwd}
\end{figure}

Ähnliche Muster zeigen sich bei den weiteren in Tabelle~\ref{tab:weak_indicators}
aufgezählten Indikatoren.

\FloatBarrier 