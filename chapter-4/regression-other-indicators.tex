\section{Lineare Regression der anderen Indikatoren}
\label{sec:reg_other_indicators}

Die für \emph{Climate change (total)} aufgebaute Regressionspipeline wird im Folgenden auf weitere
Umweltindikatoren angewendet, um zu prüfen, in welchem Umfang sich diese mit denselben
Produktmerkmalen erklären lassen. Die zentralen Gütemaße aller Indikatoren sind in
Tabelle~\ref{tab:reg_other_overview} zusammengefasst. Ergänzende Streudiagramme und QQ-Plots
für die einzelnen Indikatoren sind in Anhang~\ref{app:other_regressions} dargestellt.

\begin{table}[ht]
  \centering
  \caption{Übersicht der Testgüte der Regressionsmodelle für weitere Indikatoren.
  RMSE und Median absoluter Fehler sind auf der Originalskala angegeben.}
  \label{tab:reg_other_overview}
  \resizebox{\textwidth}{!}{
  \begin{tabular}{lccccccc}
    \toprule
    Indikator & $n$ & Transform. & $R^2_{\text{Test}}$ & RMSE\textsubscript{Test} & Median abs. Fehler & MdARE & MARE \\
    \midrule
    Acidification & 177 & log1p & 0.845 & $490.87\,\mathrm{kg\ SO_2}$ & $0.2864\,\mathrm{kg\ SO_2}$ & 0.9579 & 3.8258 \\
    Hazardous waste disposed & 168 & log1p & 0.813 & $18489.74\,\mathrm{kg}$ & $176.7890\,\mathrm{kg}$ & 0.6639 & 13.1431 \\
    Water use & 169 & Box-Cox & 0.726 & $28682.36\,\mathrm{m^3}$ & $1353.56\,\mathrm{m^3}$ & 0.9297 & 2.3278 \\
    Photochemical ozone formation (HH) & 171 & Box-Cox & 0.802 & $109.1708\,\mathrm{kg\ C_2H_4}$ & $0.0255\,\mathrm{kg\ C_2H_4}$ & 0.6665 & 1.2811 \\
    Resource use, fossils & 171 & log1p & 0.871 & $1119662.28\,\mathrm{MJ}$ & $14460.95\,\mathrm{MJ}$ & 0.6370 & 11.1394 \\
    Eutrophication (terrestrial) & 107 & Box-Cox & 0.793 & $99.33\,\mathrm{mol\ N}$ & $0.0854\,\mathrm{mol\ N}$ & 0.5611 & 1.1653 \\
    Ozone depletion & 170 & Box-Cox & 0.858 & $0.0029\,\mathrm{kg\ CFC-11}$ & $0.00001\,\mathrm{kg\ CFC-11}$ & 0.8635 & 1.0411 \\
    Resource use, minerals and metals & 175 & Box-Cox & 0.866 & $0.7218\,\mathrm{kg\ Sb}$ & $0.0010\,\mathrm{kg\ Sb}$ & 0.7798 & 1.0003 \\
    \bottomrule
  \end{tabular}
  }
\end{table}

\FloatBarrier
\subsection{Regression des Indikators Acidification}
\label{sec:reg_acidification}
\FloatBarrier

Als repräsentatives Beispiel für Indikatoren mit hoher Modellgüte wird im Folgenden
\emph{Acidification} detaillierter dargestellt. Es wurden keine Transformation, \texttt{log1p} und
eine Box-Cox Transformation verglichen. Die beste Testgüte wird mit \texttt{log1p} erreicht.
Die Zielvariable ist damit $\log(1+\text{acidification}_{\text{total}})$, und es konnten $n=177$ PEPs
verwendet werden.
Die beste Testgüte wird mit \texttt{log1p} erreicht, 
daher wird im Folgenden dieses Modell berichtet.
Die Zielvariable ist dabei der
log-transformierte Gesamtwert des Indikators
$\text{log\_acid} = \log(1 + \text{acidification}_\text{total})$.
Tabelle~\ref{tab:reg_acidification} fasst die Testleistung nach Rücktransformation 
zusammen.

\begin{table}[h]
  \centering
  \caption{Gütekennzahlen des linearen Regressionsmodells
           (\texttt{Acidification} als Zielvariable).}
  \label{tab:reg_acidification}
  \begin{tabular}{lc}
    \toprule
    Größe & Wert (Test) \\
    \midrule
    $R^2_{\text{Test}}$   & $0.845$ \\
    $\mathrm{RMSE}_{\text{Test}}$ & $490.87\,\mathrm{kg\ SO_2}$ \\
    \midrule
    Median absoluter Fehler & $0.2864\,\mathrm{kg\ SO_2}$ \\
    $\mathrm{MdARE}_{\text{Test}}$ (Median rel. Fehler) & $0.9579$ \\
    $\mathrm{MARE}_{\text{Test}}$ (Mittelwert rel. Fehler) & $3.8258$ \\
    \bottomrule
  \end{tabular}
\end{table}



Das Modell erklärt rund $85\%$ der Varianz mit einem RMSE von
$490{,}87\,\mathrm{kg\ SO_2}$, was auf eine insgesamt gute 
Vorhersagegüte für den Indikator \texttt{Acidification} hinweist.
Die robusten Fehlermaße ergänzen dieses Bild.
Der Median des absoluten Fehlers liegt weit unter dem RMSE 
bei etwa $0{,}29\,\mathrm{kg\ SO_2}$ und beschreibt damit die typische 
Abweichung eines repräsentativen Produkts.
Der Median der relativen Fehler ($\mathrm{MdARE} \approx 0{,}96$) zeigt, dass die 
Vorhersage für ein typisches Produkt häufig in der Größenordnung des wahren Werts liegt.
Der deutlich größere Mittelwert der relativen Fehler ($\mathrm{MARE} \approx 3{,}83$) weist 
zugleich auf eine stark schiefe Fehlerverteilung mit einzelnen sehr großen 
relativen Abweichungen hin.
Dies ist insbesondere bei kleinen Zielwerten plausibel, 
da dort bereits kleine absolute Fehler zu sehr großen relativen Fehlern führen.
Insgesamt ist das Modell damit für \texttt{Acidification} gut geeignet, 
einzelne Produkte können jedoch deutlich schlechter getroffen werden.


Zur Veranschaulichung zeigt Abbildung~\ref{fig:reg_acidification_scatter}
ein Streudiagramm der vorhergesagten gegenüber den
tatsächlichen Werten von \texttt{Acidification}.

\begin{figure}[h]
  \centering
  \includegraphics[width=0.9\textwidth]{images/regression_acid.png}
  \caption{Vorhergesagte gegenüber tatsächlichen Acidification-Werten. Beide Achsen sind 
  logarithmisch skaliert.}
  \label{fig:reg_acidification_scatter}
\end{figure}

Die meisten Punkte liegen in der Nähe der Diagonalen, insbesondere im
Bereich mittlerer Acidification-Werte, was auf eine gute
Abbildung des allgemeinen Trends hinweist. Hohe Werte werden tendenziell
leicht unterschätzt und im Bereich von $10^1$ gibt es ein Cluster, der überschätzt wird.
Im Bereich der sehr kleinen Werte schätzt das Modell relativ betrachtet sehr ungenau, 
die absoluten
Fehler bleiben dort jedoch gering. Das Fehlerverhalten von Trainings- und
Testdaten ist vergleichbar, so dass keine starke Überanpassung erkennbar
ist. Insgesamt bestätigt die Analyse, dass das aus dem CO$_2$-Fall
übernommene Modell auch für den Indikator \emph{Acidification} robuste und
plausible Vorhersagen liefert.

Abbildung~\ref{fig:regression_acid_qq} zeigt
einen QQ Plot der Schätzfehler des Modells im Vergleich zu einer Normalverteilung.
Wie beim CO$_2$ Indikator zeigen sich auf der Originalskala der Fehler 
typischerweise sehr starke Abweichungen, die eine schwache Interpretationsebene bereitstellen.
Daher wird hier nur die Transformationsskala dargestellt.
Im mittleren Bereich liegen die Punkte nah an der Referenzgeraden, was darauf hindeutet,
dass der Großteil der Fehler näherungsweise normalverteilt ist.
In den Randbereichen sind jedoch Abweichungen erkennbar.
Insgesamt ist die Fehlerverteilung im Zentrum gut durch eine Normalverteilung approximierbar,
während die Extrembereiche durch schwerere Verteilungsschwänze
geprägt sind.

\begin{figure}[h]
  \centering
  \includegraphics[width=0.95\textwidth]{images/regression_acid_qq.png}
  \caption{QQ Plot der Schätzfehler des Acidification Modells (Test- und Trainingsset).}
  \label{fig:regression_acid_qq}
\end{figure}

Eine ausführliche Visualisierung der übrigen Indikatoren, einschließlich QQ-Plots, ist in
Anhang~\ref{app:other_regressions} dokumentiert.


\FloatBarrier
\subsection{Indikatoren mit geringer Modellgüte}

Neben den in Tabelle~\ref{tab:reg_other_overview} aufgeführten Indikatoren mit moderater bis hoher Modellgüte wurden
alle weiteren Umweltindikatoren mit derselben Regressionspipeline geschätzt.
Für einige Zielgrößen bleibt das erreichte Test-$R^2$ jedoch unter $0{,}5$, sodass hier
nicht von einem zuverlässigen Vorhersagemodell gesprochen werden kann.
Tabelle~\ref{tab:weak_indicators} fasst diese Indikatoren zusammen.

Für diese schwächer erklärbaren Zielgrößen wurden ebenfalls verschiedene
Transformationen der Zielvariable verglichen. Box-Cox kann die Testgüte für alle
restlichen Indikatoren leicht verbessern, die Zugewinne bleiben jedoch insgesamt begrenzt und reichen nicht aus,
um die Indikatoren in den Bereich stabiler Vorhersagegüte zu überführen.

\begin{table}[h]
  \centering
  \caption{Indikatoren und Gütemaße mit geringer Modellgüte.}
  \label{tab:weak_indicators}
  \resizebox{\textwidth}{!}{
  \begin{tabular}{lccc}
    \toprule
    Indikator &
    $R^2_{\text{Test}}$  &
    RMSE\textsubscript{Test} & Anzahl analysierter PEPs \\
    \midrule
    Eutrophication (freshwater)      & $0.434$ & $1.3165$ & 133 \\
    Eutrophication (marine)          & $0.322$ & $25.0197$ & 107 \\
    Radioactive waste disposed       & $0.492$ & $176.8942$ & 158 \\
    \bottomrule
  \end{tabular}
  }
\end{table}

Abbildung~\ref{fig:reg_rwd} zeigt den Indikator \emph{Radioactive waste disposed} als ein
Beispiel. Der Indikator ist nur eingeschränkt erklärbar
($R^2_{\text{Test}}\approx 0{,}49$), und sowohl der RMSE im Verhältnis zum Mittelwert
(Rel.\ RMSE $\approx 3{,}31$) als auch die relativen Fehlermaße
($\mathrm{MdARE}\approx 1{,}85$, $\mathrm{MARE}\approx 19{,}01$) weisen auf eine stark
instabile Vorhersage hin.
Gleichzeitig bleibt der Median des absoluten Fehlers mit $0{,}5193$ vergleichsweise klein,
was zur sehr kleinen Größenordnung des Indikators passt und zeigt, dass die extremen relativen Fehler
vor allem bei kleinen Zielwerten entstehen.
Insgesamt wird die Streuung im Streudiagramm nicht ausreichend abgebildet, und eine Box-Cox-Transformation
kann die Fehlerstruktur nicht stabilisieren.

\begin{figure}[H]
  \centering
  \includegraphics[width=0.9\textwidth]{images/regression_rwd.png}
  \caption{Vorhergesagte gegenüber tatsächlichen Werten des Indikators
           \emph{Radioactive waste disposed}.}
  \label{fig:reg_rwd}
\end{figure}

Der QQ Plot in Abbildung~\ref{fig:reg_rwd_qq} zeigt die Residuen auf der Transformationsskala.
Im Zentrum liegen die Quantile nur näherungsweise auf der Referenzgeraden, während die äußeren Bereiche
deutlich abweichen. Dies weist auf schwere Verteilungsschwänze und systematische Modellfehler hin.

\begin{figure}[h]
  \centering
  \includegraphics[width=0.95\textwidth]{images/regression_rwd_qq.png}
  \caption{QQ Plot der Schätzfehler des Modells für \emph{Radioactive waste disposed}
           auf der Transformationsskala.}
  \label{fig:reg_rwd_qq}
\end{figure}

Ähnliche Muster zeigen sich auch bei den beiden Eutrophication-Indikatoren in
Tabelle~\ref{tab:weak_indicators}. Sie sind im Anhang~\ref{app:other_regressions_weak} 
ebenfalls visualisiert.