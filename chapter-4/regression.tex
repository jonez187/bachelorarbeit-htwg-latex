\section{Lineare Regression der CO$_2$-Äquivalente}
\label{sec:regression_co2}

Ziel der folgenden Analyse ist es zu untersuchen, inwieweit sich die in den PEPs
ausgewiesenen Treibhausgasemissionen (\emph{Climate Change, total})
durch wenige, aus den Dokumenten verfügbare Produktmerkmale erklären lassen,
die grundsätzlich auch für Produkte ohne PEP messbar sind.
Im Fokus steht in diesem Kapitel ausschließlich der CO$_2$-Indikator und ein
lineares Regressionsmodell. Weitere Umweltindikatoren und alternative
Modellklassen werden in späteren Abschnitten betrachtet.

\subsection{Datenbasis und Transformation}

Für die Regression werden nur Datensätze berücksichtigt, bei denen
\texttt{cc\_total}, \texttt{total\_weight} und
\texttt{electricity\_consumption} vorhanden und positiv sind.
Nach dieser Filterung verbleiben insgesamt n = 171 PEPs.
Die Materialinformationen liegen als Massenanteile in Spalten der Form
\texttt{m\_*} vor (z.\,B.\ \texttt{m\_steel}, \texttt{m\_aluminium},
\texttt{m\_copper}, \texttt{m\_wood}).
Um sehr seltene Materialien auszublenden, werden nur Materialspalten
verwendet, die in mindestens 10 Produkten einen positiven Anteil
aufweisen.

Die Verteilungen der CO$_2$-Äquivalente, des Produktgewichts und des
Stromverbrauchs sind stark rechtsschief und decken mehrere Größenordnungen ab.
Um den Einfluss extremer Werte zu verringern und die Größenordnungen besser
vergleichbar zu machen, werden diese Variablen mit der Funktion
\texttt{log1p} transformiert.
Es werden die folgenden Größen definiert:
\[
  \text{log\_cc} = \log(1 + \text{CO2}_{\text{total}}), \quad
  \text{log\_w}  = \log(1 + \text{weight}), \quad
  \text{log\_e}  = \log(1 + \text{electricity}).
\]
Die lineare Regression wird auf der Transformationsskala von
$\text{log\_cc}$ durchgeführt.
Bei Bedarf lassen sich die Vorhersagen über die inverse Funktion
\texttt{expm1} wieder auf die Originalskala der Emissionen zurückführen.

\subsection{Modellformulierung}

Das endgültig betrachtete Modell nutzt drei Arten von erklärenden Variablen:
das log-transformierte Gesamtgewicht, den log-transformierten, über die
Lebensdauer aggregierten Stromverbrauch und verdichtete Materialinformationen
aus einer PCA der Materialien.
In der log-transformierten Skala hat das Modell die Form
\[
  \text{log\_cc} =
  \beta_0 +
  \beta_1 \cdot \text{log\_w} +
  \beta_2 \cdot \text{log\_e} +
  \sum_{j=1}^{k} \gamma_j \cdot \text{PC\_mat}_j +
  \varepsilon,
\]
wobei $\text{PC\_mat}_j$ die Material-Hauptkomponenten aus der PCA bezeichnen
und k die Anzahl der verwendeten Komponenten ist.
Diese Hauptkomponenten fassen jeweils ein charakteristisches Muster aus
Materialanteilen zusammen (vgl.\ref{sub:mat_pc}) und
fungieren als verdichtete Materialindikatoren im Regressionsmodell.
Der Fehlerterm $\varepsilon$ umfasst alle nicht modellierten Einflüsse sowie
Mess- und Rundungsfehler.

Die Schätzung der Koeffizienten erfolgt nicht über eine geschlossene
OLS-Lösung, sondern mit einem iterativen, gradientenbasierten Verfahren:
dem \texttt{SGDRegressor} aus \texttt{scikit-learn}. Die Parameter werden
schrittweise aktualisiert, indem in jeder Epoche sukzessive über die
Trainingsbeispiele (oder kleine Mini-Batches) iteriert wird. Die Anzahl
der Epochen wurde auf E = 100 festgelegt. Die Lernrate wird von
\texttt{scikit-learn} gemäß dem gewählten Lernraten-Schema
(\texttt{learning\_rate = '...'}, Anfangsschrittweite
$\eta_0 = \dots$) automatisch skaliert.

\subsection{Schätzverfahren, Validierung und Ergebnisse}

Zur Bewertung der Modellgüte wird ein äußerer Train/Test-Split mit einem
Testanteil von 15\,\% verwendet.
Der Split wird R = 100-mal mit unterschiedlichen Zufallskeimen
durchgeführt, so dass jeweils unterschiedliche Train- und Testsets entstehen.
Für jeden dieser äußeren Läufe wird die Pipeline neu auf dem Trainingsset
gefitttet und anschließend auf dem unabhängigen Testset ausgewertet.
Auf diese Weise erhält man Gütemaße über viele plausible
Train/Test-Aufteilungen und kann die Stabilität des Modells gegenüber einer
zufälligen Stichprobenziehung beurteilen.

\iffalse
Zusätzlich wird auf dem Trainingsset eines äußeren Splits ein innerer
Wiederholungssplit durchgeführt:
In jeweils $200$ Wiederholungen wird das Trainingsset erneut in Trainings-
und Validierungsdaten aufgeteilt, die Pipeline wird gefittet und die
Gütemaße werden auf beiden Teilmengen berechnet.
Dieser innere Loop dient ausschließlich der Stabilitätsanalyse der linearen
Regression und der gewählten Features; es werden keine Hyperparameter
optimiert.
\fi

\paragraph{Ergebnisse der CO$_2$-Regression}

Die Modelle werden über das Bestimmtheitsmaß $R^2$ und den
Root-Mean-Square-Error (RMSE) bewertet.
Beide Größen beziehen sich auf die log-transformierte Zielvariable
$\text{log\_cc}$.
Tabelle~\ref{tab:reg_co2_mainmodel} fasst die Testgüte über alle äußeren
Wiederholungen zusammen und zeigt zusätzlich die Kennzahlen des besten Laufs
(d.\,h.\ des Runs mit maximalem $R^2_{\text{Test}}$).

\begin{table}[h]
  \centering
  \caption{Gütekennzahlen des linearen Regressionsmodells
           (\texttt{log\_cc} als Zielvariable) über $R=\dots$ äußere
           Train/Test-Splits.}
  \label{tab:reg_co2_mainmodel}
  \begin{tabular}{lcc}
    \toprule
    Größe & Mittelwert (Test) $\pm$ Std.\ & Bester Lauf (Test) \\
    \midrule
    $R^2_{\text{Test}}$   & $0.882 \pm 0.054$ & $0.963$ \\
    $\mathrm{RMSE}_{\text{Test}}$ & $1.095 \pm 0.199$ & $0.874$ \\
    \bottomrule
  \end{tabular}
\end{table}

Im Mittel erklärt das Modell etwa 88\% der Varianz von \texttt{log\_cc}
auf dem Testset, der mittlere RMSE liegt bei ungefähr einer
log-Einheit.
Der beste Lauf erreicht ein Test-$R^2$ von 96\% bei einer
Test-RMSE von 0,874.
Damit liefert das lineare Modell eine robuste und über 
verschiedene zufällige Stichprobenaufteilungen hinweg stabile
Approximation der CO$_2$-Äquivalente.

Eine Auswertung der größten Abweichungen der Schätzungen zeigt, dass
sie bei besonders schweren Produkten auftreten.
Die fünf stärksten Ausreißer stammen aus PEPs mit einem Gesamtgewicht von
mindestens 720\,kg.


\paragraph{Visualisierung der Vorhersagequalität}

Zur Veranschaulichung der Modellgüte zeigt
Abbildung~\ref{fig:regression_scatter} für einen exemplarischen Lauf
(z.\,B.\ den Run mit dem besten Test-$R^2$) ein Streudiagramm der
vorhergesagten gegenüber den tatsächlichen Werten von \texttt{log\_cc}.
Auf beiden Achsen ist die Transformationsskala
$\log(1 + \text{CO2}_{\text{total}})$ verwendet.
Trainings- und Testdaten werden im Plot getrennt dargestellt, und eine
gestrichelte Diagonale $y = x$ markiert die ideale Übereinstimmung von
Vorhersage und Realität.

\begin{figure}[H]
  \centering
  \includegraphics[width=1.0\textwidth]{images/regression_co2.png}
  \caption{Vorhersagte und tatsächliche Werte von \texttt{log\_cc}
           für einen exemplarischen Train/Test-Split des CO$_2$-Modells.}
  \label{fig:regression_scatter}
\end{figure}

Die meisten Punkte liegen in der Nähe der Diagonalen, und die Streuung ist
für Trainings- und Testdaten ähnlich.
Es ist keine systematische Über- oder Unterschätzung erkennbar.
Dies passt zu den ausgewiesenen Gütemaßen und spricht dafür, dass das
Modell die Daten gut abbildet, ohne stark zu überanpassen.

Die in diesem Abschnitt beschriebene Pipeline, bestehend aus
log-transformierter Zielgröße, technischen Basismerkmalen (Gewicht,
Stromverbrauch), verdichteten Materialinformationen (PCA) sowie äußerem und
innerem Wiederholungssplit, wird im nächsten Schritt auf weitere
Umweltindikatoren übertragen.
Dabei ändert sich die verfügbare Datenbasis (bedingt durch
unterschiedlen Anteile der Fehlwerte der Indikatoren) und die Erklärbarkeit der
Indikatoren.
Ein systematischer Vergleich mit weiteren Modellklassen (z.\,B.\
regularisierten linearen Modellen oder Baumverfahren) folgt anschließend in
einem separaten Kapitel.
