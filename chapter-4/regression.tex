\section{Lineare Regression des Indikators \emph{Climate change (total)}}
\label{sec:regression_co2}

Ziel der folgenden Analyse ist es entsprechend der
Zielsetzung dieser Arbeit zu untersuchen, inwieweit sich die in den PEPs
ausgewiesenen Treibhausgasemissionen (\emph{Climate Change, total})
durch wenige, aus den Dokumenten verfügbare Produktmerkmale erklären lassen,
die grundsätzlich auch für Produkte ohne PEP messbar sind.
Im Fokus steht in diesem Kapitel ausschließlich der CO$_2$-Indikator und ein
lineares Regressionsmodell. Weitere Umweltindikatoren werden in späteren Abschnitten betrachtet.

\subsection{Datenbasis und Transformation}

Für die Regres\-sion werden nur Daten\-sätze berück\-sich\-tigt, bei welchen
\texttt{cc\_total}, \texttt{total\_weight} und
\texttt{electricity\_consumption} vor\-handen und positiv sind.
Nach dieser Filterung verbleiben insgesamt $n = 173$ PEPs.
Die Materialinformationen liegen als Massenanteile vor.

Die Verteilungen der CO$_2$-Äquivalente, des Produktgewichts und des
Stromverbrauchs sind stark rechtsschief und decken mehrere Größenordnungen ab.
Um den Einfluss extremer Werte zu verringern und die Größenordnungen besser
vergleichbar zu machen, werden diese Variablen mit der Funktion
\texttt{log1p} transformiert.
Es werden die folgenden Größen definiert:
\[
  \text{log\_cc} = \log(1 + \text{CO2}_{\text{total}}), \quad
  \text{log\_w}  = \log(1 + \text{weight}), \quad
  \text{log\_e}  = \log(1 + \text{electricity\_consumption})
\]
Die lineare Regression wird auf der Transformationsskala von
$\text{log\_cc}$ durchgeführt.
Bei Bedarf lassen sich die Vorhersagen über die inverse Funktion
\texttt{expm1} wieder auf die Originalskala der Emissionen zurückführen.

\subsection{Modellformulierung}

Das endgültig betrachtete Modell nutzt drei Arten von erklärenden Variablen:
das log-transformierte Gesamtgewicht, den log-transformierten, über die
Lebensdauer aggregierten Stromverbrauch und verdichtete Materialinformationen
aus einer PCA der Materialien.
In der log-transformierten Skala hat das Modell die Form
\[
  \text{log\_cc} =
  \beta_0 +
  \beta_1 \cdot \text{log\_w} +
  \beta_2 \cdot \text{log\_e} +
  \sum_{j=1}^{k} \gamma_j \cdot \text{PC\_mat}_j +
  \varepsilon,
\]
wobei $\text{PC\_mat}_j$ die Material-Hauptkomponenten aus der PCA bezeichnen
und k die Anzahl der verwendeten Komponenten ist.
Die Materialanteile selbst werden nicht log transformiert.
Sie liegen als Massenanteile vor, werden skaliert und anschließend per PCA zu
Hauptkomponenten zusammengefasst.
Diese Hauptkomponenten fassen jeweils ein charakteristisches Muster aus
Materialanteilen zusammen (vgl.~\ref{sub:mat_pc}) und
fungieren als verdichtete Materialindikatoren im Regressionsmodell.
Der Fehlerterm $\varepsilon$ umfasst alle nicht modellierten Einflüsse sowie
Mess- und Rundungsfehler.

\FloatBarrier
\subsection{Schätzverfahren, Validierung und Ergebnisse}

Zur Bewertung der Modellgüte wird ein Train/Test-Split mit einem
Testanteil von 10\,\% verwendet.
Das Modell wird ausschließlich durch die 90\% Trainingsdaten 
angepasst und anschließend auf dem unabhängigen Testset ausgewertet.
Als Regressor wird \texttt{Ridge} (L2-Regularisierung) verwendet, 
wobei der Regularisierungsparameter \(\lambda\) ausschließlich auf den Trainingsdaten durch 
Cross-Validation bestimmt wird.
Auf diese Weise erhält man Gütemaße, die angeben, wie gut das Modell
auf die nie zuvor gesehenen Testdaten generalisieren kann.

Die Modelle werden über das Bestimmtheitsmaß $R^2$ und den
Root-Mean-Square-Error (RMSE) bewertet.
Der RMSE wird nach Rücktransformation der Vorhersagen 
auf der Originalskala berichtet.
Tabelle~\ref{tab:reg_co2_mainmodel} fasst die Testgüte 
des CO$_2$ Regressionsmodells zusammen.

\begin{table}[ht]
  \centering
  \caption{Gütekennzahlen des linearen Regressionsmodells
           (\texttt{Climate Change (total)} als Zielvariable).}
  \label{tab:reg_co2_mainmodel}
  \begin{tabular}{lcc}
    \toprule
    Größe & Wert (Test)  \\
    \midrule
    $R^2_{\text{Test}}$   & $0.896$  \\
    $\mathrm{RMSE}_{\text{Test}}$ & $25116.20\,\mathrm{kg\ CO_2}$ \\
    \bottomrule
  \end{tabular}
\end{table}

Das Modell erklärt damit etwa 90\,\% der Varianz von \texttt{log\_cc}
auf dem Testset und lässt nur etwa 10\,\% unerklärt. Dieser Wert spricht
für eine hohe Modellgüte.

Der absolute RMSE-Wert von $25116.20\,\mathrm{kg\ CO_2}$ erscheint hoch.
Dies ist vor allem eine Folge der größten Produkte im Datensatz.
Abweichungen der Schätzung bei großen Produkten dominieren den RMSE,
da dieser einzelne große Fehler quadratisch stärker gewichtet.

Zur Illustration, wie stark einzelne Produkte die Fehlermaße beeinflussen können,
wird der größte Fehler im Testset separat betrachtet. Das Produkt mit der größten
Abweichung ist:

\begin{itemize}
  \item \textbf{Produkt:} \texttt{Daikin Applied Europe SpA Wärmepumpe \cite{daikin-pumpe}}
  \item \textbf{Tatsächlicher Wert:} $266000.0\,\mathrm{kg\ CO_2}$
  \item \textbf{Vorhersage:} $37447.76\,\mathrm{kg\ CO_2}$
  \item \textbf{Absoluter Fehler:} $228552.24\,\mathrm{kg\ CO_2}$
  \item \textbf{Relativer Fehler:} $85.9\%$
\end{itemize}

Dieses Beispiel verdeutlicht, dass einzelne sehr große Produkte
einen großen absoluten Fehler verursachen können und damit einen 
überproportionalen Einfluss auf RMSE-basierte Kennzahlen haben, obwohl sich
der relative Fehler in Grenzen hält.
Daher werden im Folgenden neben $R^2$ und RMSE auch robuste und relative
Fehlermaße berichtet, um die Modellgüte über verschiedene Größenordnungen
hinweg nachvollziehbar zu interpretieren.

Folgende Maße werden definiert:
\begin{itemize}
  \item \textbf{Median abs.\ Fehler (MdAE):} \(\mathrm{MdAE}=\mathrm{median}\!\left(|y_i-\hat{y}_i|\right)\).
  \item \textbf{MdARE (Median relative Errors):} \(\mathrm{MdARE}=\mathrm{median}\!\left(\frac{|y_i-\hat{y}_i|}{y_i}\right)\) (für \(y_i>0\)).
  \item \textbf{MARE (Mean relative Errors):} \(\mathrm{MARE}=\frac{1}{n}\sum_{i=1}^{n}\frac{|y_i-\hat{y}_i|}{y_i}\) (für \(y_i>0\)).
\end{itemize}
Der MdAE weird auf Originalskala in der jeweiligen Einheit angegeben. MdARE und MARE sind einheitenlos und können als 
Anteil interpretiert werden (z.\,B.\ \(0.66\approx66\,\%\)).

Der MdAE liegt bei $2220.80\,\mathrm{kg\ CO_2}$ und beschreibt die
typische absolute Abweichung eines repräsentativen Produkts.
Der MdARE liegt bei $51.49\%$.
Der MARE beträgt $120.27\%$ und fällt deutlich höher aus,
was auf einzelne sehr große relative Abweichungen hinweist.
Diese Kombination aus Median und Mittelwert zeigt, dass die Fehlerverteilung
durch Ausreißer geprägt ist und eine rein RMSE-basierte Interpretation
die typische Modellleistung verzerren kann.

Eine mittlere relative Abweichung von $120\%$ bedeutet, dass das Modell die Größenordnung
der verursachten CO$_2$-Äquivalente im Mittel korrekt einordnet.
Eine präzise Abschätzung über alle Produkte hinweg ist mit dem verwendeten,
bewusst kompakten Feature-Set nur eingeschränkt möglich.

Eine weitere Auswertung der größten absoluten Abweichungen zeigt, dass diese vor allem bei
besonders schweren Produkten auftreten.
Die fünf stärksten Ausreißer stammen aus PEPs mit einem Gesamtgewicht
von mindestens $720\,\mathrm{kg}$.

\FloatBarrier
\subsection{Visualisierung der Vorhersagequalität}

Zur Veranschaulichung der Modellgüte zeigt Abbildung~\ref{fig:regression_scatter}
ein Streudiagramm der vorhergesagten gegenüber den tatsächlichen Werten von
\texttt{Climate Change total}.
Trainings- und Testdaten werden getrennt dargestellt, und eine Diagonale $y=x$
markiert die ideale Übereinstimmung zwischen Vorhersage und Realität.

\begin{figure}[ht]
  \centering
  \includegraphics[width=1.0\textwidth]{images/regression_co2.png}
  \caption{Vorhergesagte und tatsächliche Werte von \texttt{Climate Change total} für einen
           exemplarischen Train und Test Split des CO$_2$-Modells. [Eigene Darstellung]}
  \label{fig:regression_scatter}
\end{figure}

Die meisten Punkte liegen nahe der Diagonalen, und die Streuung ist für Trainings-
und Testdaten ähnlich.
Dies passt zu den ausgewiesenen Gütemaßen und spricht dafür, dass das Modell die Daten
abbildet, ohne stark zu überanpassen.
Gleichzeitig ist in einigen Wertebereichen sichtbar, dass der 
Zielwert tendenziell
unter- (z.\,B.\ Cluster bei $10^4$ bis $10^5\,\mathrm{kg}$) bzw. überschätzt 
(z.\,B.\ Cluster bei $10^3,\mathrm{kg}$) wird.


Für die OLS Theorie wird angenommen, 
dass die Fehler auf der Modellskala näherungsweise normalverteilt sind. 
Für die hier verfolgte Vorhersage und Fehleranalyse ist diese Annahme jedoch 
nicht zwingend erforderlich, dient aber als diagnostischer Hinweis.

Abbildung~\ref{fig:regression_co2_qq_orig} zeigt einen QQ Plot der Residuen auf der 
Originalskala, nach Rücktransformation verglichen mit den theoretischen Quantilen der Normalverteilung.
Hier sind deutliche Abweichungen von der Referenzgeraden sichtbar, 
insbesondere in den Rändern, was auf Schiefe und schwere 
Verteilungsschwänze hindeutet.
Dies ist plausibel, da auf Originalskala einzelne sehr große Produkte die Fehler dominieren
und Fehler häufig multiplikativ wirken, was auf Originalskala zu stark asymmetrischen Residuen führt.

\begin{figure}[ht]
  \centering
  \includegraphics[width=0.95\textwidth]{images/regression_co2_qq_orig.png}
  \caption{QQ Plot der Residuen des CO$_2$-Modells auf Originalskala im Testset. [Eigene Darstellung]}
  \label{fig:regression_co2_qq_orig}
\end{figure}

Abbildung~\ref{fig:regression_co2_qq_trans} zeigt den QQ Plot der Residuen auf der Transformationsskala.
Hier liegen die Punkte deutlich näher an der Referenzgeraden als auf Originalskala,
was eine näherungsweise Normalität im mittleren Bereich unterstützt.
In den äußeren Quantilen bleiben Abweichungen sichtbar, was auf eine erhöhte Wahrscheinlichkeit
großer Fehler hinweist.

\begin{figure}[ht]
  \centering
  \includegraphics[width=0.95\textwidth]{images/regression_co2_qq_trans.png}
  \caption{QQ Plot der Residuen des CO$_2$-Modells im Testset auf der Transformationsskala \texttt{log\_cc}. [Eigene Darstellung]}
  \label{fig:regression_co2_qq_trans}
\end{figure}

Die Normalitätsannahme ist auf Originalskala für die CO$_2$-Daten klar verletzt,
was aufgrund der großen Wertebandbreite und der dominierenden Ausreißer
bei sehr großen Produkten erwartbar ist.
Auf der Transformationsskala liegen die Fehler
deutlich näher an einer Normalverteilung, während in den Rändern weiterhin
Abweichungen verbleiben.

Für die Zielsetzung dieser Arbeit, nämlich robuste Vorhersagen für neue Produkte,
ist diese Diagnose dennoch konsistent mit einem brauchbaren Modell.
Die Modellgüte wird ausschließlich auf strikt getrennten Testdaten berichtet,
und zusätzlich werden robuste und relative Fehlermaße verwendet, um die Leistung
über verschiedene Größenordnungen hinweg fair zu bewerten.
Die verbleibenden Abweichungen von der Normalität werden deshalb nicht als Ausschlusskriterium
interpretiert, jedoch als Hinweis, dass klassische Annahmen der OLS-Theorie
nur eingeschränkt auf diesen Datensatz übertragbar sind und Fehlermaße auf 
der Originalskala stark durch wenige
große Produkte geprägt werden.

Die in diesem Abschnitt beschriebene Pipeline, bestehend aus trans\-for\-mierter Ziel\-größe,
technischen Basis\-merk\-malen (Gewicht, Stromverbrauch) und 
verdichteten Material\-in\-for\-mationen (PCA)
wird im nächsten Schritt auf weitere Um\-welt\-indikatoren übertragen.
Dabei ändert sich die verfügbare Datenbasis, bedingt durch unterschiedliche Fehlwerteanteile,
und die Erklärbarkeit der jeweiligen Indikatoren.
Für die übrigen Indikatoren werden die getesteten Transformationen (keine Transformation, \texttt{log1p}, Box-Cox)
und die resultierende Auswahl jeweils knapp zusammengefasst, und die Fehlerdiagnostik wird auf die wichtigsten
Befunde reduziert.
Für jeden Indikator werden diese Transformationsoptionen systematisch verglichen.
Die Auswahl erfolgt datengetrieben anhand der Testgüte, wobei Fehlermaße auf Originalskala
nach Rücktransformation berichtet werden, damit sie in der Einheit interpretierbar bleiben.

\FloatBarrier