\section{Lineare Regression der CO$_2$-Äquivalente}
\label{sec:regression_co2}

Ziel der folgenden Analyse ist es zu untersuchen, inwieweit sich die in den PEPs
ausgewiesenen Treibhausgasemissionen (\emph{Climate Change, total})
durch einfache technische Produkteigenschaften erklären lassen.
Als erklärende Variablen werden das Gesamtgewicht des Produkts,
der über die Nutzungsdauer aggregierte Stromverbrauch sowie die
Materialzusammensetzung in Form von Hauptkomponenten aus der PCA verwendet.
Da der Stromverbrauch in den PEPs nur als Gesamtwert über die gesamte
Lebensdauer und nicht phasenspezifisch angegeben wird, erfolgt die
Modellierung auf Basis der gesamten CO$_2$-Äquivalente pro Produkt.

\paragraph{Datenbasis}

Für die Regression werden nur Datensätze berücksichtigt, bei denen
\texttt{cc\_total}, \texttt{total\_weight} und
\texttt{electricity\_consumption} vorhanden und positiv sind.
Nach dieser Filterung verbleiben insgesamt $n = 171$ PEPs.
Die Materialinformationen liegen als Massenanteile in Spalten der Form
\texttt{m\_*} vor, zum Beispiel \texttt{m\_steel}, \texttt{m\_aluminium},
\texttt{m\_copper} oder \texttt{m\_wood}.
Um sehr seltene Materialien auszublenden, werden nur solche Materialien
verwendet, die in mindestens 15 Produkten vorkommen.
Auf diese Weise gehen 25 Materialspalten in die weiteren Auswertungen ein.

\paragraph{Transformation der Variablen}

Die Verteilungen der CO$_2$-Äquivalente, des Produktgewichts und des
Stromverbrauchs sind stark rechtsschief und decken einen großen Wertebereich ab.
Um diese Größen besser handhabbar zu machen und den Einfluss extremer Werte
zu verringern, werden sie mit der Funktion \texttt{log1p} transformiert.
Dabei werden die folgenden Variablen definiert:
\[
  \text{log\_cc} = \log(1 + \text{CO2}_{\text{total}}), \quad
  \text{log\_w}  = \log(1 + \text{weight}), \quad
  \text{log\_e}  = \log(1 + \text{electricity}).
\]
Die lineare Regression wird auf der Transformationsskala von $\text{log\_cc}$
durchgeführt.
Falls nötig, lassen sich die Ergebnisse über die inverse Funktion
\texttt{expm1} wieder auf die Originalskala der Emissionen zurückführen.

\paragraph{Regressionsmodell und Schätzung}

In der log-transformierten Skala hat das Regressionsmodell die Form
\[
  \text{log\_cc} =
  \beta_0 +
  \beta_1 \cdot \text{log\_w} +
  \beta_2 \cdot \text{log\_e} +
  \sum_{j=1}^{k} \gamma_j \cdot \text{PC\_mat} +
  \varepsilon,
\]
wobei \text{PC\_mat} die Material-Hauptkomponenten aus der PCA bezeichnen und
$k = 16$ die Anzahl der verwendeten Komponenten ist.
Der Fehlerterm $\varepsilon$ steht für nicht modellierte Einflüsse und
Messfehler.

Die Schätzung der Koeffizienten erfolgt mit gewöhnlichen kleinsten Quadraten
(OLS) auf Basis der SciPy-Funktion \texttt{linalg.lstsq}.
Vor der OLS-Schätzung werden alle erklärenden Variablen
(also \text{log\_w}, \text{log\_e}{} und die PCA-Komponenten) per Min--Max-Skalierung in den
Bereich $[0,1]$ gebracht.
Die dafür benötigten Minimal- und Maximalwerte werden ausschließlich aus dem
Trainingsdatensatz bestimmt und anschließend auf das Testset übertragen.

Zur Bewertung der Modellgüte wird ein äußerer Train/Test-Split mit einem
Testanteil von 15\,\% verwendet.
Auf dem Trainingsdatensatz wird zusätzlich ein innerer Wiederholungssplit
durchgeführt: In 200 Wiederholungen wird der Trainingsdatensatz jeweils
erneut in Train- und Validierungsdaten aufgeteilt.
Damit lässt sich prüfen, wie stabil die Schätzungen und Gütemaße gegenüber
der zufälligen Aufteilung der Daten sind.
Hyperparameter werden dabei nicht optimiert, sondern es geht nur um eine
Stabilitätsanalyse.

\paragraph{Modellvarianten}

Um den Beitrag der Materialinformationen und die Wirkung der PCA zu bewerten,
werden drei Modellvarianten mit identischer Train/Test-Aufteilung betrachtet:

\begin{enumerate}
  \item \textbf{Basismodell}: \\
        In dieser Variante wird $\text{log\_cc}$ nur auf \text{log\_w}{} und \text{log\_e}{}
        (sowie \texttt{lifetime}, falls vorhanden) regressiert.
        Materialinformationen werden nicht verwendet.

  \item \textbf{Basis + Rohmaterialien}: \\
        Hier wird das Basismodell um die 25 ausgewählten Materialspalten
        \texttt{m\_*} in roher Form erweitert.
        Jede Materialart geht also als eigene Regressorvariable in das
        Modell ein.

  \item \textbf{Basis + PCA-Materialien (PCR)}: \\
        In dieser Variante wird das Basismodell statt mit den einzelnen
        Materialspalten mit 16 Material-Hauptkomponenten erweitert.
        Diese Komponenten stammen aus einer PCA auf dem Materialblock
        und bilden gemeinsam mindestens 90\,\% der Varianz der 25
        Ausgangsvariablen ab.
        Dieses Modell entspricht einer Principal Component Regression
        (PCR) auf Basis der Materialinformationen.
\end{enumerate}

\paragraph{Gütekennzahlen der Modelle}

Die Modelle werden über das Bestimmtheitsmaß $R^2$ und den
Root-Mean-Square-Error (RMSE) auf dem Testdatensatz bewertet.
Beide Kennzahlen beziehen sich auf die log-transformierte Zielvariable
$\text{log\_cc}$.
Tabelle~\ref{tab:reg_co2_models} zeigt die Ergebnisse für die drei
Modellvarianten.

\begin{table}[h]
  \centering
  \caption{Gütekennzahlen der linearen Regressionsmodelle für den Indikator
  \emph{Climate Change (total)} in der Transformationsskala $\log(1+\text{CO2})$.}
  \label{tab:reg_co2_models}
  \begin{tabular}{lccc}
    \toprule
    Modellvariante                     & $R^2_{\text{Train}}$ & $R^2_{\text{Test}}$ & RMSE\textsubscript{Test} \\
    \midrule
    Basis (Gewicht, Stromverbrauch)    & 0.817 & 0.730 & 1.904 \\
    Basis + Rohmaterialien             & 0.904 & 0.727 & 1.914 \\
    Basis + PCA-Materialien (16 PCs)   & 0.851 & 0.822 & 1.325 \\
    \bottomrule
  \end{tabular}
\end{table}

Das Basismodell erklärt bereits einen großen Teil der Varianz von $\text{log\_cc}$,
erreicht aber auf dem Testdatensatz nur ein $R^2$ von etwa 0{,}73.
Die Erweiterung um Rohmaterialien erhöht das Trainings-$R^2$
deutlich, verbessert das Test-$R^2$ jedoch nicht und führt zu einem sehr
ähnlichen RMSE.
Dies deutet darauf hin, dass das Modell mit Rohmaterialspalten eher zur
Überanpassung neigt.
Das PCA-basierte Modell mit 16 Material-Hauptkomponenten erreicht dagegen
ein Test-$R^2$ von rund 0{,}82 und einen deutlich geringeren RMSE von etwa
1{,}33 in der $\text{log\_cc}$-Skala.
Damit zeigt sich, dass die Materialinformationen in komprimierter Form zur
Verbesserung der Vorhersagegüte beitragen.

\subsection*{Visualisierung der Vorhersagequalität}
\begin{figure}[H]
  \centering
  \includegraphics[width=1.0\textwidth]{images/regression_co2.png}
  \caption{Schematischer Aufbau der Pipeline: von der PEP-Erfassung bis zur
  strukturierten Datenbasis.}
  \label{fig:regression_scatter}
\end{figure}

Zur Veranschaulichung der Modellgüte zeigt \ref{fig:regression_scatter} für das PCA-basierte Modell ein
Streudiagramm der vorhergesagten gegenüber den tatsächlichen Werten von
$\text{log\_cc}$ erstellt.
Auf beiden Achsen ist die Transformationsskala $\log(1 + \text{CO2}_{\text{total}})$
verwendet.
Trainings- und Testdaten werden im Plot getrennt dargestellt, und eine
gestrichelte Diagonale $y = x$ markiert die ideale Übereinstimmung von
Vorhersage und Realität.
Die meisten Punkte liegen in der Nähe dieser Diagonalen, und die Streuung ist
für Trainings- und Testdaten ähnlich.
Dies passt zu den ausgewiesenen Gütemaßen und spricht dafür, dass das
PCA-Modell die Daten gut abbildet, ohne stark zu überanpassen.

\paragraph{Ausreißer}

Zusätzlich werden die Werte betrachtet, die auf der Originalskala der
CO$_2$-Äquivalente betrachtet.
Dabei zeigt sich, dass die größten Abweichungen bei besonders großen
Produkten auftreten.
Die stärksten Ausreißer stammen aus PEPs mit einem Gesamtgewicht von
mindestens 720\,kg.
Eine inhaltliche Einordnung dieser Ausreißer und mögliche Ursachen werden im
anschließenden Diskussionsabschnitt aufgegriffen.
