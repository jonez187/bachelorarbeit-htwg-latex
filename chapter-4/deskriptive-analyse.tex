\section{Deskriptive Annäherung an die PEP-Daten}

Wie bereits in Kapitel~\ref{sec:stat_grundlagen} erläutert, ist eine
deskriptive Betrachtung der Daten ein notwendiger erster Schritt, um die
Qualität und Aussagekraft des Datensatzes zu beurteilen. 
Ziel dieses Abschnitts ist es, einen Überblick über die Vollständigkeit der
vorliegenden Daten sowie über zentrale Eingangs- und Ausgangsvariablen zu geben. 
Dazu werden zunächst die Anteile der fehlenden Werte analysiert,
gefolgt von einer Beschreibung der
Input-Variablen Gesamtgewicht, Stromverbrauch, Materialzusammensetzung und
Energiemodelle. 
Abschließend werden die Verteilungen der Umweltindikatoren
untersucht, um erste strukturelle Muster und Auffälligkeiten innerhalb des
Datensatzes zu identifizieren.


\subsection{Vollständigkeit der Werte}

Zur Bewertung der Datenvollständigkeit wurde der Anteil fehlender Werte
pro Variable berechnet und in einem Balkendiagramm
dargestellt (Abb.~\ref{fig:missingness}). Die Missingness umfasst sowohl Nullwerte
aus der Datenpipeline als auch Indikatoren, die in den PEPs nicht berichtet werden.

Die Analyse zeigt deutliche Unterschiede zwischen den Indikatoren: Für
\emph{Wasserknappheit} fehlen rund 78~\% der Werte, während mehrere weitere
Indikatoren wie \emph{Eutrophierung marines Gewässer}, \emph{Klimawandel (fossil,
total)} und \emph{Eutrophierung terrestrisch} Fehlstände von etwa 30~\% aufweisen.

Für den in der Regression verwendeten Stromverbrauch (\texttt{electricity\_consumption})
fehlen knapp 25~\% der Werte. In diesen PEPs wird zwar ein Energienutzungsmodell
beschrieben und eine Formel zur Berechnung des Verbrauchs angegeben, der
tatsächliche Gesamtstromverbrauch über die Lebensdauer wird jedoch nicht als
konkreter Zahlenwert ausgewiesen, sondern basiert auf externen Katalogdaten
(z.\,B.\ Verlustleistung $P_{\text{use}}$). Diese Informationen stehen in der
vorliegenden Datenpipeline nicht zur Verfügung und können daher
nicht automatisch in \texttt{electricity\_consumption} überführt werden. In der
weiteren Analyse stehen somit nur die PEPs mit explizit angegebenem
Stromverbrauch zur Verfügung, was den Stichprobenumfang für die Regression
reduziert.

Der Indikator \emph{Wasserknappheit} wird aufgrund der hohen Ausfallrate von der Auswertung
ausgeschlossen, da keine belastbaren statistischen Aussagen getroffen werden können.

\begin{figure}[H]
  \centering
  \includegraphics[width=\textwidth]{images/missingness_bar.png}
  \caption{Anteil fehlender Werte pro Umweltindikator (\%, \(N=233\) Produkte).}
  \label{fig:missingness}
\end{figure}



Zusätzlich wurden die Erscheinungsjahre der PEP-Ecopassports untersucht. Die
Veröffentlichungen reichen von 2020 bis 2025 und verteilen sich wie in
Abb.~\ref{fig:erscheinungsjahr} dargestellt. Der deutliche Anstieg ab 2022 zeigt die
zunehmende Etablierung des Formats und eine stärkere Datenverfügbarkeit in den
letzten Jahren.

\begin{figure}[h]
  \centering
  \includegraphics[width=0.8\linewidth]{images/dateien_pro_jahr.png}
  \caption{Erscheinungsjahre der analysierten PEP-Dokumente (\(N=233\) Produkte).}
  \label{fig:erscheinungsjahr}
\end{figure}


\subsection{Überblick der \emph{Input}-Variablen}

Für die Input-Variablen werden robuste Kennzahlen (\textbf{Median}, \textbf{IQR})
gezeigt und durch den \textbf{Mittelwert} ergänzt, um die Wirkung der Schiefe
(v.\,a.\ Rechts­schiefe) zu verdeutlichen. Ausreißer werden nicht entfernt, ihre
Einflüsse spiegeln sich im Mittelwert wider.


\begin{table}[h]
  \centering
  \small
  \begin{tabular}{l r r r r r r}
    \hline
    \textbf{Variable} & \textbf{Einheit} & \textbf{Min} & \textbf{Median} &
    \textbf{Max} & \textbf{IQR} & \textbf{Mittelwert} \\
    \hline
    Gesamtgewicht & kg & 0.0395 & 2.178 & 13022.6 & 125.210 & 278.023 \\
    Stromverbrauch & kWh & 0.026 & 326.511 & 8203569.5 & 86147.1 & 228061.654 \\
    \hline
  \end{tabular}
  \caption{Robuste deskriptive Kennzahlen der Basisvariablen.}
  \label{tab:basis}
\end{table}

Die in Tab.~\ref{tab:basis} dargestellten Basisvariablen zeigen deutlich
\textbf{rechtsschiefe Verteilungen} mit großen Interquartilsabständen (IQR). Beim
\emph{Gesamtgewicht} reicht die Spannweite von 0.04~kg bis über 13\,000~kg, was die
starke Heterogenität der betrachteten Produkte verdeutlicht. Das kleinste Produkt ist
ein leichtes elektronisches Gerät, ein \emph{Connected dimmer mit Bluetooth interface}
(\href{https://register.pep-ecopassport.org/pep/consult/mbesqrsCBZbWbKJq6-kJ3gQki33tF6v8mGqIbNl9ZBA/mbesqrsCBZbWbKJq6-kJ3lQmBuGvAHsLUfQU9idjOpk}{PEP-Link}),
während das größte Produkt, ein \emph{Flüssigkeitskühler mit drehzahlgeregeltem
Schraubenverdichter und Greenspeed\textsuperscript{TM}-Technologie}
(\href{https://register.pep-ecopassport.org/pep/consult/mbesqrsCBZbWbKJq6-kJ3i_N0-YkMDn5wDN9lr3hiPE/mbesqrsCBZbWbKJq6-kJ3lQmBuGvAHsLUfQU9idjOpk}{PEP-Link}),
mehr als 13~t erreicht. Der Mittelwert liegt mit 278~kg weit über dem Median
(2.18~kg), was die ausgeprägte Rechtsschiefe bestätigt.


Auch der \emph{Stromverbrauch} weist eine extreme Streuung auf (ca.~86147~kWh), 
mit Werten zwischen 0.026~kWh und über 8.2e6~kWh.
Damit ist das kleinste Produkt nahezu stromlos im Betrieb, 
während das größte Produkt eine mehrjährige oder großtechnische Nutzung abbildet.
Der Mittelwert (228000~kWh) übersteigt den Median (327~kWh) um mehrere Größenordnungen, was 
die starke Rechtsverschiebung der Verteilung verdeutlicht.


Die Zusammenhänge zwischen Stromverbrauch und Umweltauswirkungen hängen 
maßgeblich von der Art der Stromerzeugung ab. 
Da sich die Strommixe regional unterscheiden, variieren auch 
die resultierenden Emissionen je nach Herkunftsland des Energiebezugs.

Im Datensatz zeigt sich, dass der Großteil der verwendeten Energiemodelle auf allgemeine europäische Strommixe 
(EU27) und Frankreich entfällt. Besonders in den Phasen Nutzung und End-of-Life ist der Anteil
europäischer Modelle deutlich höher. Dies liegt vermutlich daran, dass die Produkte häufig europaweit 
vertrieben und verwendet werden. Daher ist es schwierig, den tatsächlichen Energiebezug eines spezifischen Landes 
realistisch abzubilden, weshalb in der Regel ein repräsentativer europäischer Durchschnitt angenommen wird.

Der hohe Anteil von Frankreich ist auf eine große Anzahl an PEP-Dokumenten aus Frankreich zurückzuführen,
die zu national vermarkteten Produkten gehören.
Von dort stammt die Association P.E.P und das Format ist dort am meisten etabliert.

Auch in der Herstellungsphase dominiert ein europäischer Energiemix, ergänzt durch einzelne Modelle 
aus Deutschland und China, was auf internationale Produktionsketten hinweist. 
Insgesamt verdeutlicht die Verteilung, dass die meisten PEP-Deklarationen von europäischen 
Strommixen ausgehen, wodurch die berechneten Umweltauswirkungen tendenziell niedrigere fossile Anteile aufweisen, als es bei 
stärker kohleabhängigen Regionen (z. B. China) der Fall wäre.

Aufgrund der europäischen Prägung des Datensatzes ist die Aussagekraft der anschließenden Regression 
für Produkte außerhalb Europas eingeschränkt.
Entsprechende Auswertungen werden mit erhöhter Unsicherheit und geringerer Datenqualität verbunden sein.
  
\begin{figure}[h] 
  \centering
  \includegraphics[width=0.8\linewidth]{images/energy_iso_grouped_count.png}
  \caption{Verteilung der verwendeten Energiemodelle (ISO-Regionen) über die Lebenszyklusphasen}
  \label{fig:meinbild}
\end{figure}


Eine weitere Variable, die die Umweltindikatoren stark beeinflussen, und damit in der zu entwickelnden Heuristik eine Rolle spielen muss,
ist die Zusammensetzung des Produkts aus den verschiedenen Materialien.
In der Tabelle \ref{tab:materials_threecols} wird aufgeführt, aus welchen Materialien das 
durchschnittliche PEP-Produkt aus dem Datensatz besteht.
Dazu wurde der Mittelwert jedes Materials über alle Produkte hinweg berechnet.
\emph{N} gibt zusätzlich die Anzahl der Produkte an, in welchen das aufgeführte Material auftaucht. 
Wie in den meisten PEP-Dokumenten werden die modularen Materialien in
die Gruppen \emph{Metalle, Plastik und andere} gegliedert. 


\begin{table}
  \centering
  {\captionsetup{font=small}
   \caption{Durchschnittliche Materialanteile nach Hauptkategorien (Mittelwert in \%).}
   \label{tab:materials_threecols}
   \scriptsize
   \setlength{\tabcolsep}{3pt}
   \renewcommand{\arraystretch}{1.03}

   \begin{adjustbox}{max width=\textwidth}
   \begin{minipage}{\textwidth}
     % ==================== METALLE ====================
     \begin{subtable}[t]{0.315\textwidth}
       \centering
       \caption{Metalle}
       \begin{tabular}{@{}L{0.60\linewidth} S r@{}}
         \toprule
         Material & {Mittelwert} & {n} \\
         \midrule
         Stahl & 26.46 & 199 \\
         Aluminium & 6.13 & 140 \\
         Kupfer & 5.41 & 161 \\
         Messing & 0.86 & 82 \\
         Zamak & 0.45 & 15 \\
         Nickel & 0.10 & 12 \\
         Zinn & 0.06 & 20 \\
         Zink & 0.05 & 9 \\
         Bronze & 0.01 & 5 \\
         Neodym & 0.01 & 5 \\
         Hartlot & 0.00 & 7 \\
         \bottomrule
       \end{tabular}
     \end{subtable}\hfill
     % ==================== KUNSTSTOFFE ====================
     \begin{subtable}[t]{0.315\textwidth}
       \centering
       \caption{Kunststoffe}
       \begin{tabular}{@{}L{0.60\linewidth} S r@{}}
         \toprule
         Material & {Mittelwert} & {n} \\
         \midrule
         Polycarbonat (PC) & 8.23 & 113 \\
         ABS & 2.84 & 105 \\
         Polyamid (PA) & 2.37 & 118 \\
         PVC & 2.08 & 86 \\
         PS & 1.13 & 62 \\
         PP & 0.80 & 75 \\
         Gummi & 0.69 & 70 \\
         PMMA & 0.66 & 19 \\
         Epoxidharz & 0.60 & 42 \\
         Polyesterharz & 0.57 & 23 \\
         PE & 0.44 & 68 \\
         PU & 0.40 & 40 \\
         PBT & 0.23 & 17 \\
         PET & 0.13 & 25 \\
         POM & 0.09 & 16 \\
         TBBPA & 0.07 & 9 \\
         HIPS & 0.06 & 4 \\
         Silikon & 0.04 & 6 \\
         EPDM & 0.02 & 5 \\
         PPS & 0.02 & 5 \\
         Sonstige & 0.03 & --\\ 
         \bottomrule
       \end{tabular}
     \end{subtable}\hfill
     % ==================== ANDERE ====================
     \begin{subtable}[t]{0.315\textwidth}
       \centering
       \caption{Andere}
       \begin{tabular}{@{}L{0.60\linewidth} S r@{}}
         \toprule
         Material & {Mittelwert} & {n} \\
         \midrule
         Papier & 15.73 & 197 \\
         Elektronik & 3.69 & 102 \\
         Holz & 3.12 & 81 \\
         Glas & 2.90 & 67 \\
         PCBA & 1.79 & 24 \\
         PCB & 1.27 & 49 \\
         Kabel & 0.35 & 38 \\
         Kältemittel & 0.35 & 53 \\
         Ferrit & 0.28 & 38 \\
         Elektromotoren & 0.27 & 9 \\
         Lack / Farbe & 0.15 & 39 \\
         Tinte & 0.08 & 15 \\
         Silizium & 0.08 & 9 \\
         Batterie & 0.08 & 10 \\
         Thionylchlorid & 0.08 & 5 \\
         Öl & 0.07 & 8 \\
         Mineralwolle & 0.06 & 13 \\
         Bitumen & 0.04 & 7 \\
         Titandioxid & 0.04 & 14 \\
         Quarz & 0.02 & 7 \\
         Flussmittel & 0.02 & 6 \\
         Filz & 0.01 & 11 \\
         Aluminiumoxid & 0.01 & 5 \\
         Haftkleber & 0.01 & 4 \\
         Sonstige & 0.48 & -- \\
         \bottomrule
       \end{tabular}
     \end{subtable}
   \end{minipage}
   \end{adjustbox}
  }
\end{table}

Die in Tabelle~\ref{tab:materials_threecols} dargestellten Materialanteile zeigen eine
sehr heterogene Zusammensetzung der untersuchten Produkte. Mit durchschnittlich
rund 26~\% ist \emph{Stahl} das mengenmäßig dominierende Einzelmaterial, gefolgt von
\emph{Papier} (15.7~\%), welches vor allem für Verpackungen verwendet wird,
und \emph{Polycarbonat (PC)} (8.2~\%).
Während Metalle in nahezu allen PEPs vertreten sind, treten bestimmte
Kunststoffe und Spezialmaterialien (z.\,B.\ PMMA, PBT, PPS) nur in wenigen Fällen auf.
Die Kategorie \emph{Andere} enthält zahlreiche kleinvolumige Komponenten, deren
summierter Anteil jedoch nicht vernachlässigbar ist. Insgesamt spiegelt sich in der
Verteilung die Diversität der erfassten Produktgruppen wider.


\subsection{Überblick der Umweltindikatoren}

Die Umweltindikatoren bilden die Output-Variablen, auf deren Basis später die
Heuristik entwickelt wird. Eine deskriptive Betrachtung verdeutlicht bereits die
Verteilungsstruktur der Daten.

% TODO: Übersetzen, einheitlich, Layout anpassen.

\begin{table}[tbp]
  \centering
  \small
  \resizebox{\textwidth}{!}{
  \begin{tabular}{
      p{5cm}
      S[table-format=1.2e3]
      S[table-format=1.2e3]
      S[table-format=1.2e3]
      S[table-format=1.2e3]
      S[table-format=1.2e3]
      p{2.6cm}
  }
  \hline
  \textbf{Indikator (total)} & \textbf{Min} & \textbf{Median} & \textbf{Max} &
  \textbf{IQR} & \textbf{Mean} & \textbf{Einheit} \\
  \hline
  Acidification & 0.000017 & 0.4295 & 3650 & 10.31 & 110.78 & kg SO$_2$ eq \\
  Climate change (total) & 0.0031 & 86.75 & 1040000 & 1979.43 & 22740.20 &
  kg CO$_2$ eq \\
  Eutrophication (freshwater) & 0.000001 & 0.0266 & 236 & 0.314 & 2.624 &
  kg P eq \\
  Hazardous waste disposed & 0.0001 & 39.3 & 489000 & 596.69 & 6438.71 & kg \\
  Ozone depletion & 0 & 0.000007 & 0.192 & 0.000286 & 0.00323 &
  kg CFC-11 eq \\
  Photochemical ozone formation (HH) & 0.000002 & 0.181 & 1410 & 3.126 &
  40.60 & kg C$_2$H$_4$ eq \\
  Resource use (fossils) & 0.0326 & 1620 & 106000000 & 95076 & 1583968.96 &
  MJ \\
  Resource use (minerals/metals) & 0.000001 & 0.00392 & 5.87 & 0.0495 &
  0.2279 & kg Sb eq \\
  Radioactive waste disposed & 0 & 0.0656 & 3260 & 0.5381 & 22.56 & kg \\
  Water use & 0.000093 & 42.4 & 5770000 & 383.85 & 98836.32 & m$^{3}$ \\
  \hline
  \end{tabular}}
  \caption{Gesamtindikatoren (Total) mit Median, IQR und Mittelwert
  (gerundet auf zwei Nachkommastellen).}
  \label{tab:desc_indicators}
\end{table}

\begin{figure}[tbp]
  \centering
  \includegraphics[width=\linewidth]{images/boxplots_indicators.png}
  \caption{Boxplots der Total-Werte je Indikator auf logarithmischer Skala.}
  \label{fig:boxplot_indicators}
\end{figure}

Wie Tabelle~\ref{tab:desc_indicators} zeigt, weisen alle Umweltindikatoren deutlich
rechtsschiefe Verteilungen auf. Der Mittelwert liegt bei allen Größen um ein
Vielfaches über dem Median. Besonders ausgeprägt ist die Schiefe bei
\emph{Climate change (total)}, \emph{Resource use (fossils)}, \emph{Water use} und
\emph{Hazardous waste disposed}.

Abbildung~\ref{fig:boxplot_indicators} ergänzt diese Zusammenfassung, indem sie die
Streuung innerhalb der Indikatoren sowie die Ausreißerstruktur in Boxplots auf
logarithmischer Skala zeigt. Bei den meisten Indikatoren erstreckt sich die
Verteilung über mehrere Größenordnungen und es treten zahlreiche Ausreißer nach
oben auf. Die logarithmische
Skala komprimiert große Werte, weshalb extrem hohe Beobachtungen im Plot
optisch näher zusammenrücken, obwohl sie sich in den Originaleinheiten stark
unterscheiden.

Insgesamt bestätigt sich eine heterogene, rechtsschiefe Datenbasis mit
einigen Produkten, die sehr hohe Umweltwirkungen aufweisen.


