\section{Deskriptive Annäherung an die PEP-Daten}
Wie im Kapitel \ref{chapter-2} angesprochen ist es sinnvoll die erarbeiteten Daten deskriptiv anzuschauen, 
um unter anderem auch die Datenqualität zu gewährleisten. 

\subsection{Vollständigkeit der Werte}

Zur Bewertung der Datenvollständigkeit wurde der Anteil fehlender Werte (\emph{missingness}) pro Umweltindikator berechnet und in einem Balkendiagramm dargestellt (Abb.~\ref{fig:missingness}). 
Die Analyse zeigt deutliche Unterschiede zwischen den Indikatoren: 
Während für \emph{Wasserknappheit} über 78~\% der Werte fehlen, weisen mehrere weitere Indikatoren
wie \emph{Eutrophierung marines Gewässer}, \emph{Klimawandel (fossil, total)} und \emph{Eutrophierung terrestrisch} Lücken von rund 30~\% auf. 
Andere Kernfelder wie \emph{Gesamtgewicht} oder \emph{Stromverbrauch} sind deutlich vollständiger. 
Für die weitere Analyse müssen die fehlenden Werte beachtet werden. Regressiv auf die Wasserknappheit zu schließen wird 
dementsprechend nicht möglich sein, da der Datensatz mit Wasserknappheit Werten zu klein ist.

\begin{figure}[h!]
    \centering
    \includegraphics[width=\textwidth]{images/missingness_bar.png}
    \caption{Anteil fehlender Werte pro Umweltindikator (\%).}
    \label{fig:missingness}
\end{figure}

\subsection{Ziel und Überblick}
Die robusten Kennzahlen (\textbf{Median}, \textbf{IQR}) werden berichtet und ergänzt mit dem \textbf{Mittelwert},
um die Wirkung von Schiefe (v.\,a.\ Rechts­schiefe) zu verdeutlichen. 
Ausreißer werden nicht entfernt. Ihre Effekte spiegeln sich im Mittelwert wider.

Zusätzlich wurden die Erscheinungsjahre der PEP-Ecopassports analysiert. 
Die Veröffentlichungen reichen von 2020 bis 2025 und setzt sich folgendermaßen zusammen: 

\begin{figure}[h!] 
  \centering
  \includegraphics[width=0.8\linewidth]{images/dateien_pro_jahr.png} % Dateiname ohne Pfad, wenn in \graphicspath
  \caption{Erscheinungsjahr der analysierten PEP-Dokumente}
  \label{fig:meinbild}
\end{figure}


% ------------------------------------------------------------
% T1: Basisvariablen (inkl. Mean)
% ------------------------------------------------------------
\begin{table}[h!]
\centering
\small
\begin{tabular}{l r r r r r r r l}
\hline
\textbf{Variable} & \textbf{Einheit} & \textbf{Min} & \textbf{Median} & \textbf{Max} & \textbf{IQR} & \textbf{Mittelwert}  \\
\hline
Gesamtgewicht & kg  & 0.0395 & 2.178 & 13022.6 & 125.210 & 278.023 \\
Stromverbrauch & kWh & 0.026 & 326.511 & 8203569.5 & 86147.1 & 228061.654 \\
\hline
\end{tabular}
\caption{Robuste deskriptive Kennzahlen der Basisvariablen.}
\label{tab:basis}
\end{table}

Die in Tabelle~\ref{tab:basis} dargestellten Basisvariablen zeigen deutlich \textbf{rechtsschiefe Verteilungen} mit sehr großen Interquartilsabständen (IQR).
Beim \emph{Gesamtgewicht} reicht die Spannweite von nur 0.04~kg bis zu über 13000~kg, was die starke Heterogenität der betrachteten Produkte verdeutlicht.
Das \emph{kleinste Produkt} ist ein sehr leichtes elektronisches Gerät, ein \emph{Connected dimmer mit Bluetooth interface}
(\href{https://register.pep-ecopassport.org/pep/consult/mbesqrsCBZbWbKJq6-kJ3gQki33tF6v8mGqIbNl9ZBA/mbesqrsCBZbWbKJq6-kJ3lQmBuGvAHsLUfQU9idjOpk}{PEP-Link}) im Grammbereich,
während das \emph{größte Produkt}, ein \emph{Flüssigkeits\-kühler mit drehzahlgeregeltem Schrauben\-verdichter und intelligenter Greenspeed\textsuperscript{TM}\mbox{}-Technologie}
(\href{https://register.pep-ecopassport.org/pep/consult/mbesqrsCBZbWbKJq6-kJ3i_N0-YkMDn5wDN9lr3hiPE/mbesqrsCBZbWbKJq6-kJ3lQmBuGvAHsLUfQU9idjOpk}{PEP-Link}),
mehr als 13~t wiegt.
Der Mittelwert liegt mit rund 278~kg weit über dem Median (2.18~kg), was die ausgeprägte Rechtsschiefe bestätigt.

Auch der \emph{Stromverbrauch} weist eine extreme Streuung auf (ca.~86147~kWh), mit Werten zwischen 0.026~kWh und über 8.2e6~kWh.
Damit ist das kleinste Produkt nahezu stromlos im Betrieb, während das größte Produkt eine mehrjährige oder großtechnische Nutzung abbildet.
Der Mittelwert (228000~kWh) übersteigt den Median (327~kWh) um mehrere Größenordnungen, was die starke Rechtsverschiebung der Verteilung verdeutlicht.


Eine weitere Variable, die die Umweltindikatoren stark beeinflussen, und damit in der zu entwickelnden Heuristik eine Rolle spielen muss,
ist die Zusammensetzung des Produkts aus den verschiedenen Materialien.
In der Tabelle \ref{tab:materials_threecols} wird aufgeführt, aus welchen Materialien das 
durchschnittliche PEP-Produkt aus dem Datensatz besteht (Mittelwert).
\emph{N} gibt die Anzahl der Produkte an, in welchen das aufgeführte Material auftaucht. 
Wie in den meisten PEP-Dokumenten werden die modularen Materialien in
die Gruppen \emph{Metalle, Plastik und andere} gegliedert. 


\begin{table}
  \centering
  {\captionsetup{font=small}
   \caption{Durchschnittliche Materialanteile nach Hauptkategorien (Mittelwert in \%).}
   \label{tab:materials_threecols}
   \scriptsize
   \setlength{\tabcolsep}{3pt}
   \renewcommand{\arraystretch}{1.03}

   \begin{adjustbox}{max width=\textwidth}
   \begin{minipage}{\textwidth}
     % ==================== METALLE ====================
     \begin{subtable}[t]{0.315\textwidth}
       \centering
       \caption{Metalle}
       \begin{tabular}{@{}L{0.60\linewidth} S r@{}}
         \toprule
         Material & {Mittelwert} & {n} \\
         \midrule
         Stahl & 26.46 & 199 \\
         Aluminium & 6.13 & 140 \\
         Kupfer & 5.41 & 161 \\
         Messing & 0.86 & 82 \\
         Zamak & 0.45 & 15 \\
         Nickel & 0.10 & 12 \\
         Zinn & 0.06 & 20 \\
         Zink & 0.05 & 9 \\
         Bronze & 0.01 & 5 \\
         Neodym & 0.01 & 5 \\
         Hartlot & 0.00 & 7 \\
         \bottomrule
       \end{tabular}
     \end{subtable}\hfill
     % ==================== KUNSTSTOFFE ====================
     \begin{subtable}[t]{0.315\textwidth}
       \centering
       \caption{Kunststoffe}
       \begin{tabular}{@{}L{0.60\linewidth} S r@{}}
         \toprule
         Material & {Mittelwert} & {n} \\
         \midrule
         Polycarbonat (PC) & 8.23 & 113 \\
         ABS & 2.84 & 105 \\
         Polyamid (PA) & 2.37 & 118 \\
         PVC & 2.08 & 86 \\
         PS & 1.13 & 62 \\
         PP & 0.80 & 75 \\
         Gummi & 0.69 & 70 \\
         PMMA & 0.66 & 19 \\
         Epoxidharz & 0.60 & 42 \\
         Polyesterharz & 0.57 & 23 \\
         PE & 0.44 & 68 \\
         PU & 0.40 & 40 \\
         PBT & 0.23 & 17 \\
         PET & 0.13 & 25 \\
         POM & 0.09 & 16 \\
         TBBPA & 0.07 & 9 \\
         HIPS & 0.06 & 4 \\
         Silikon & 0.04 & 6 \\
         EPDM & 0.02 & 5 \\
         PPS & 0.02 & 5 \\
         Sonstige & 0.03 & --\\ 
         \bottomrule
       \end{tabular}
     \end{subtable}\hfill
     % ==================== ANDERE ====================
     \begin{subtable}[t]{0.315\textwidth}
       \centering
       \caption{Andere}
       \begin{tabular}{@{}L{0.60\linewidth} S r@{}}
         \toprule
         Material & {Mittelwert} & {n} \\
         \midrule
         Papier & 15.73 & 197 \\
         Elektronik & 3.69 & 102 \\
         Holz & 3.12 & 81 \\
         Glas & 2.90 & 67 \\
         PCBA & 1.79 & 24 \\
         PCB & 1.27 & 49 \\
         Kabel & 0.35 & 38 \\
         Kältemittel & 0.35 & 53 \\
         Ferrit & 0.28 & 38 \\
         Elektromotoren & 0.27 & 9 \\
         Lack / Farbe & 0.15 & 39 \\
         Tinte & 0.08 & 15 \\
         Silizium & 0.08 & 9 \\
         Batterie & 0.08 & 10 \\
         Thionylchlorid & 0.08 & 5 \\
         Öl & 0.07 & 8 \\
         Mineralwolle & 0.06 & 13 \\
         Bitumen & 0.04 & 7 \\
         Titandioxid & 0.04 & 14 \\
         Quarz & 0.02 & 7 \\
         Flussmittel & 0.02 & 6 \\
         Filz & 0.01 & 11 \\
         Aluminiumoxid & 0.01 & 5 \\
         Haftkleber & 0.01 & 4 \\
         Sonstige & 0.48 & -- \\
         \bottomrule
       \end{tabular}
     \end{subtable}
   \end{minipage}
   \end{adjustbox}
  }
\end{table}

Die Materialien sind sehr sehr unterschiedlich....



% ------------------------------------------------------------
% T2: Umweltindikatoren (Total) mit Mean und Schiefen-Indikatoren
% ------------------------------------------------------------
\begin{table}
\centering
\small
{\setlength{\tabcolsep}{3pt}%
\begin{tabular}{
  p{5cm}
  S[table-format=1.2e2]  % Min
  S[table-format=3.2e2]  % Median
  S[table-format=3.2e2]  % Max
  S[table-format=3.2e2]  % IQR
  S[table-format=3.2e2]  % Mean
  p{2.6cm}
}
\hline
\textbf{Indikator (total)} & \textbf{Min} & \textbf{Median} & \textbf{Max} & \textbf{IQR} & \textbf{Mean} & \textbf{Einheit} \\
\hline
Acidification & 0.000017 & 0.4295 & 3650 & 10.31 & 110.78 & kg SO$_2$ eq \\
Climate change (total) & 0.0031 & 86.75 & 1040000 & 1979.43 & 22740.20 & kg CO$_2$ eq \\
Eutrophication (freshwater) & 0.000001 & 0.0266 & 236 & 0.314 & 2.624 & kg P eq \\
Hazardous waste disposed & 0 & 39.3 & 489000 & 596.69 & 6438.71 & kg \\
Ozone depletion & 0 & 0.000007 & 0.192 & 0.000286 & 0.00323 & kg CFC-11 eq \\
Photochemical ozone formation (HH) & 0.000002 & 0.181 & 1410 & 3.126 & 40.60 & kg C$_2$H$_4$ eq \\
Resource use (fossils) & 0.0326 & 1620 & 106000000 & 95076 & 1583968.96 & MJ \\
Resource use (minerals/metals) & 0.000001 & 0.00392 & 5.87 & 0.0495 & 0.2279 & kg Sb eq \\
Radioactive waste disposed & 0 & 0.0656 & 3260 & 0.5381 & 22.56 & kg \\
Water use & 0.000093 & 42.4 & 5770000 & 383.85 & 98836.32 & m$^{3}$ \\
\hline
\end{tabular}}
\caption{Gesamtindikatoren (Total) mit Median/IQR und Mittelwert (gerundet auf zwei Nachkommastellen).}
\label{tab:desc_indicators}
\end{table}


Wie in Tabelle~\ref{tab:desc_indicators} aufgezeigt weisen alle Umweltindikatoren eine stark rechtsschiefe Verteilungen auf, 
bei denen der Mittelwert um ein Vielfaches über dem Median liegt.  
Eine deutliche Rechtsschiefe zeigt sich insbesondere bei den Indikatoren 
\emph{Climate change (total)}, \emph{Resource use (fossils)}, \emph{Water use} und
\emph{Hazardous waste disposed}, bei denen der Mittelwert den Median teilweise um mehrere Größenordnungen übersteigt.  
Lediglich \emph{Ozone depletion} und \emph{Resource use (minerals/metals)} zeigen 
etwas geringere Unterschiede zwischen Mittelwert und Median, bleiben aber ebenfalls rechtsschief.  
Insgesamt bestätigt dieses Muster eine stark heterogene Datenbasis, 
bei der wenige Produkte mit extrem hohen Umweltbelastungen die Verteilungen dominieren.  

