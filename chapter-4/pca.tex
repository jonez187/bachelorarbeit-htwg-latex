\section{PCA der Materialien}

Zur explorativen Analyse der hochdimensionalen Materialdaten wurde eine Hauptkomponentenanalyse
(PCA) durchgeführt.
Benutzt wurde die Python-Bibliothekt \emph{scikit-learn} \cite{scikit-learn-pca}.
Die Eingangsdaten wurden zuvor pro Spalte z-standardisiert (Mittelwert 0, Varianz 1).
Alle Materialien, die in einem Produkt nicht vorkommen, wurden als 0 interpretiert.

\begin{figure}[H]
  \centering
  \includegraphics[width=1.0\textwidth]{images/pca_kum.png}
  \caption{Kumulative erklärte Varianz}
  \label{fig:pca_kum}
\end{figure}

In Abbildung~\ref{fig:pca_kum} ist die durch die Hauptkomponenten erklärte kumulative 
Varianz dargestellt.
Die ersten 10 Hauptkomponenten erklären bereits ungefähr 55\,\% der Varianz, 20 Komponenten etwa 80\,\%
und 30 Komponenten ungefähr 90\,\%. Ab etwa 30 Komponenten nehmen zusätzliche Komponenten nur noch
geringe Varianzanteile auf (z.\,B.\ ca.\ 97\,\% bei 40 Komponenten). Für die weitere Modellierung
wurde daher ein Varianzschwellenwert von 90\,\% gewählt, was in diesem Datensatz etwa $k=30$
Hauptkomponenten entspricht. Dieser Wert ist ein Kompromiss zwischen Dimensionsreduktion
und Informationsverlust. 

Anzumerken ist, dass die hier dargestellte PCA auf dem vollständigen Datensatz basiert und nur der 
explorativen Darstellung der Materialien dient. 
Für die nachfolgende Modellschätzung wird die PCA jeweils ausschließlich
auf dem Trainingsdatensatz gefittet und anschließend auf die Testdaten angewendet, um
zu vermeiden, dass das Modell durch die PCA die Testdaten lernt.

\subsection{Interpretation der Material-Hauptkomponenten}

Jede Hauptkomponente ist eine
lineare Kombination der einzelnen Materialanteile. Die zugehörigen \emph{Loadings}
geben an, wie stark ein bestimmtes Material zur jeweiligen Komponente beiträgt.
Große Beträge der Loadings (unabhängig vom Vorzeichen) weisen auf Materialien hin,
die diese Komponente besonders prägen. Das Vorzeichen bestimmt lediglich die Richtung
der Achse (Zunahme vs.\ Abnahme eines Materials), ist für die inhaltliche Einordnung
des Musters aber weniger wichtig als die absolute Größe.

Auf Basis der Top-Loadings pro Komponente lassen sich die ersten
Hauptkomponenten wie folgt interpretieren:
\begin{itemize}
  \item \textbf{PC$_1$ (Stahl- / Kältemittel):}
        Die erste Komponente wird vor allem durch \texttt{Stahl} und
        \texttt{Kältemittel} auf der einen, sowie \texttt{Epoxid},
        \texttt{Papier/Karton} und \texttt{PC} auf der anderen Seite geprägt.
        Sie trennt damit Produkte mit hohem Stahl- und Kältemittelanteil
        (z.\,B.\ schwere, kältemittelführende Geräte) von Produkten, bei denen
        Epoxidharze, Papier und PC-Kunststoffe dominieren (typisch für Leiterplatten,
        Gehäuse und Verpackungs- bzw.\ Isolationsanteile).
        Auffällig ist, dass das Kältemittel trotz eines im Mittel sehr geringen
        Massenanteils von nur rund 0{,}35\,\% mit einem hohen Loading in der ersten
        Hauptkomponente erscheint. Das deutet darauf hin, dass Kältemittel zwar nur
        in wenigen Produkten vorkommt, dort aber gemeinsam mit hohen
        Stahlanteilen auftritt und damit ein charakteristisches Materialmuster
        schwerer, kältemittelführender Geräte beschreibt.
  \item \textbf{PC$_2$ und PC$_3$ (elektromechanische Baugruppen):}
        In PC$_2$ und PC$_3$ treten insbesondere \texttt{Hartlot},
        \texttt{Elektrische Motoren}, \texttt{Kältemittel} und \texttt{PE-HD}
        mit hohen Loadings auf, teilweise kontrastiert mit Aluminium,
        Messing oder PU. Diese Komponenten beschreiben Varianten
        von Produkten mit ausgeprägten elektromechanischen Baugruppen (Motoren,
        Lötverbindungen, Hochdichte-PE) gegenüber eher metallischen oder mit
        PU-Schaum ausgestatteten Ausführungen.
  \item \textbf{PC$_4$ und PC$_5$ (Oberflächen und Gehäusematerialien):}
        PC$_4$ wird vor allem durch \texttt{Farbe}, \texttt{Aluminium},
        \texttt{PVC} und \texttt{PU} geprägt, PC$_5$ durch \texttt{Messing},
        \texttt{Quartz} und \texttt{Polyester}. Diese Komponenten fassen
        unterschiedliche Typen von Oberflächen- und Gehäusematerialien zusammen,
        also Kombinationen aus metallischen Gehäusen, Beschichtungen, Kunststoffen
        und Füllstoffen.
  \item \textbf{Weitere Komponenten (PC$_6$\,ff.):}
        Die nachfolgenden Hauptkomponenten erfassen zunehmend speziellere
        Materialkombinationen, etwa unterschiedliche Kunststoffmischungen, 
        Isolations- und Dichtmaterialien, sowie
        Elektronik-spezifische Stoffe und Flammschutzmittel. Viele dieser
        Komponenten unterscheiden damit Varianten von elektronischen Produkten mit
        hohem Leiterplatten- und Metallanteil von Geräten mit höheren Anteilen an
        klassischen Struktur- und Dämmmaterialien.
\end{itemize}

Insgesamt zeigt die Material-PCA, dass sich der Materialmix der betrachteten PEPs auf
wenige wiederkehrende Muster zurückführen lässt: Kombinationen aus Strukturmetallen
(z.\,B.\ Stahl, Aluminium), Kunststoffen, Elektronik- und Leiterplattenmaterialien
sowie Isolations- und Spezialstoffen. Diese Hauptkomponenten werden in der folgenden
Regression als verdichtete Materialindikatoren verwendet, um den Einfluss des
Materialmixes auf die CO$_2$-Äquivalente zu erfassen, ohne alle stark korrelierten
Materialspalten einzeln berücksichtigen zu müssen.
