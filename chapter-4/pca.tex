\section{PCA der Materialien}

Zur explorativen Analyse und zur Reduktion der Di\-men\-sionali\-tät der 
Material\-daten wurde eine Haupt\-komponenten\-analyse
(PCA) durchgeführt.
Benutzt wurde die Python-Bibliothek \emph{scikit-learn} \cite{scikit-learn-pca}.
Alle Materialien, die in einem Produkt nicht vorkommen, wurden als 0 interpretiert.

Vor der PCA wurden die Verteilungen der Materialanteile 
grafisch untersucht. Die Histogramme, von welchen die 3 häufigsten Materialien 
in Abbildung~\ref{fig:material_hists}
dargestellt sind,
zeigen, dass die Daten teilweise multimodal verteilt sind und viele Beobachtungen 
im Bereich sehr kleiner Anteile liegen. Bei einer einfachen z-Standardisierung 
wäre die Verteilung durch 
Ausreißer weiterhin stark beeinflusst.

Für die PCA der Materialien wird daher der \emph{scikit-learn RobustScaler} \cite{sklearn-robust}
verwendet. Dieser zentriert jede Materialspalte um den Median und skaliert durch die
Interquartilsbreite (IQR), wobei die Quantile standardmäßig als
$Q_{0.25}$ und $Q_{0.75}$ definiert sind (\texttt{quantile\_range=(25,75)}).
Im Gegensatz zur z-Standardisierung wird damit die Skalierung deutlich weniger durch
Ausreißer beeinflusst.
Die PCA wird anschließend auf der robust skalierten Materialmatrix durchgeführt.
In den nachfolgenden Regressionsmodellen werden die Skalierungsparameter jeweils nur auf den
Trainingsdaten geschätzt und anschließend auf Validierungs- bzw.\ Testdaten angewendet, um
Data Leakage zu vermeiden.



\begin{figure}[h]
  \centering
  \includegraphics[width=1.0\textwidth]{images/material_histograms.png}
  \caption{Histogramme der Materialien mit den größten Anteilen über alle Podukte.}
  \label{fig:material_hists}
\end{figure}


\FloatBarrier
\subsection{Ergebnis der PCA}

\begin{figure}[H]
  \centering
  \includegraphics[width=1.0\textwidth]{images/pca_kum_robust.png}
  \caption{Kumulative erklärte Varianz}
  \label{fig:pca_kum}
\end{figure}

In Abbildung~\ref{fig:pca_kum} ist die durch die Hauptkomponenten erklärte kumulative 
Varianz dargestellt.
Die ersten 5 Hauptkomponenten erklären bereits ungefähr 90\,\% der Varianz. 
Ab etwa 20 Komponenten nehmen zusätzliche Komponenten nur noch
geringe Varianzanteile auf. Für die weitere Modellierung
wurde daher ein Varianzschwellenwert von 95\,\% gewählt, was in diesem Datensatz etwa $k=8$
Hauptkomponenten entspricht. Dieser Wert ist ein Kompromiss zwischen Dimensionsreduktion
und Informationsverlust. 

Anzumerken ist, dass die hier dargestellte PCA auf dem vollständigen Datensatz basiert und nur der 
explorativen Darstellung der Materialien dient. 
Für die nachfolgende Modellschätzung wird die PCA jeweils ausschließlich
auf dem Trainingsdatensatz gefittet und anschließend auf die Testdaten angewendet, um
zu vermeiden, dass das Modell durch die PCA die Testdaten lernt.

\subsection{Interpretation der Material-Hauptkomponenten}
\label{sub:mat_pc}

Jede Hauptkomponente ist eine
lineare Kombination der einzelnen Materialanteile. Die zugehörigen \emph{Loadings}
geben an, wie stark ein bestimmtes Material zur jeweiligen Komponente beiträgt.
Große Beträge der Loadings (unabhängig vom Vorzeichen) weisen auf Materialien hin,
die diese Komponente besonders prägen. Das Vorzeichen bestimmt lediglich die Richtung
der Achse (Zunahme vs.\ Abnahme eines Materials), ist für die inhaltliche Einordnung
des Musters aber weniger wichtig als die absolute Größe.

Auf Basis der Top-Loadings pro Komponente lassen sich die ersten
Hauptkomponenten wie folgt interpretieren:
\begin{itemize}
  \item \textbf{PC$_1$ (\emph{PS (Polystyrol)} dominiert):}
        Die erste Hauptkomponente wird fast vollständig durch
        \texttt{ps} geprägt, mit deutlich kleineren Beiträgen von
        \texttt{pe} und anderen Kunststoffen. Sie unterscheidet
        damit Produkte mit hohen Polystyrolanteilen
        (zum Beispiel bestimmte Gehäuse oder Schäume) von Produkten,
        bei denen PS kaum eine Rolle spielt.

  \item \textbf{PC$_2$ (\emph{PE (Polyethylen)} versus übrige Kunststoffe):}
        PC$_2$ hat ein sehr hohes positives Loading auf
        \texttt{pe}, während \texttt{other\_plastics} und
        \texttt{ps} eher negativ geladen sind. Diese Komponente
        beschreibt also eine Achse zwischen Produkten mit ausgeprägtem
        PE-Anteil und solchen, bei denen eher andere Kunststoffe oder
        unspezifische Kunststoffmischungen dominieren.

  \item \textbf{PC$_3$ (unspezifische Kunststoffmischungen):}
        In PC$_3$ dominiert \texttt{other\_plastics}, ergänzt durch
        positive Beiträge von \texttt{pe}, \texttt{electronics}
        und \texttt{other} sowie leicht negative Beiträge von
        \texttt{pvc}. Diese Komponente steht für Produkte mit einem
        breiten Kunststoffmix und einem gewissen Elektronikanteil,
        die sich von eher PVC-basierten Gehäusen abgrenzen.

  \item \textbf{PC$_4$ (Leiterplatten und Elektronik versus PVC):}
        PC$_4$ wird stark durch \texttt{pcba (Leiterplattenbestückung)} geprägt, mit
        zusätzlichen positiven Beiträgen von Glas und Elektronik
        und einem deutlich negativen Loading auf \texttt{pvc}.
        Sie trennt damit Produkte mit hohem Leiterplatten und
        Elektronikanteil von solchen, bei denen PVC-Gehäusematerial
        im Vordergrund steht.

  \item \textbf{PC$_5$ (Elektronikschwerpunkt):}
        In PC$_5$ weist \texttt{electronics} das höchste positive
        Loading auf, während \texttt{pvc}, \texttt{pcba},
        \texttt{pe} und \texttt{other\_plastics} überwiegend negativ
        geladen sind. Diese Komponente beschreibt Produkte, bei denen
        Elektronikbauteile in der Masse dominieren und klassische
        Gehäuse und Strukturkunststoffe relativ weniger Gewicht haben.

\end{itemize}

Bemerkenswert ist, dass klassische Strukturmetalle wie Stahl oder Kupfer
sowie Materialien wie Papier in den ersten
Hauptkomponenten nicht mit den höchsten Ladungen auftreten. Das liegt daran,
dass ihre Anteile über viele PEPs hinweg vergleichsweise stabil sind und
dadurch weniger zur Gesamtvarianz beitragen als die stark schwankenden
Kunststoff und Elektronikanteile. Ihr Einfluss verteilt sich daher auf
spätere Hauptkomponenten mit geringerem Varianzanteil.

Insgesamt zeigt die Material-PCA, dass sich die sehr unterschiedlichen
Materiallisten auf wenige dominante Muster verdichten lassen. Die ersten
drei Komponenten beschreiben vor allem verschiedene Kunststoffmischungen
(PS, PE und andere Kunststoffe), während PC$_4$ und
PC$_5$ Leiterplatten und Elektronik gegenüber PVC lastigen Gehäusen
abgrenzen. Diese fünf Hauptkomponenten erklären zusammen knapp
90,\% der Varianz im Materialblock und bilden damit die prägendsten
Materialmuster der PEPs ab. In den folgenden
Regressionsmodellen werden sie genutzt,
um den Einfluss des Materialmixes auf die Umweltindikatoren zu erfassen,
ohne alle einzelnen Materialien separat berücksichtigen zu müssen.
