\chapter{Theoretische Grundlagen}

Hier werden die theoretischen Grundlagen für die vorliegende Arbeit gelegt. 
Ausgangspunkt ist die Beobachtung, dass Smart-Home-/IoT-Produkte Umweltwirkungen nicht nur in der Nutzungsphase (Stromverbrauch), 
sondern ebenso durch Materialzusammensetzung, Fertigung, Distribution und Entsorgung verursachen. 
Für die standardisierte Berichterstattung solcher Wirkungen existieren deklarative Formate wie die PEP Ecopassports, 
die Indikatoren entlang des Lebenszyklus ausweisen. Damit diese Angaben für quantitative Analysen nutzbar werden, 
sind konsistente Begriffe, Einheiten und Moduldefinitionen ebenso erforderlich wie ein Verständnis zentraler 
statistischer Verfahren zur Muster- und Zusammenhangsanalyse. Dieses Kapitel führt daher zunächst in Struktur 
und Inhalte von PEP-Deklarationen ein und skizziert anschließend die methodischen Bausteine 
(u. a. Lineare Regression und PCA), die in den folgenden Kapiteln zur Reduktion von Variablen, 
zur Erklärung von Indikatorvarianz und zur Ableitung einer praxistauglichen Heuristik für Produkte ohne PEP eingesetzt werden.

\section{PEP-Ecopassport}
Was ist PEP-Ecopassport, was steht drin, was ist interessant für mich

\section{PDF-Datenextraktion (?)}
Theoretische Grundlagen PDFs und Daten (Aufbau Textlayer usw), gpt Modell für extraktion (Vor-/Nachteile)

\section{Statistische Methoden}
Welche statistischen Methoden werden eingesetzt und worauf basieren sie?

\subsection{}
